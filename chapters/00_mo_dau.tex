% ============ MỞ ĐẦU ============

\subsection*{1. Tính cấp thiết của đề tài}
\addcontentsline{toc}{subsection}{1. Tính cấp thiết của đề tài}

Trong lịch sử tư tưởng phương Tây, triết học Hy Lạp cổ đại được coi là cái nôi của nền văn minh lý tính. Trong số các trường phái triết học Hy Lạp hóa (Hellenistic philosophy), chủ nghĩa khắc kỉ nổi lên như một hệ tư tưởng có sức sống mãnh liệt nhất, kéo dài hơn 500 năm từ thế kỷ III TCN đến thế kỷ II CN và để lại dấu ấn sâu đậm trong văn hóa, tôn giáo và đời sống chính trị của Đế chế La Mã\footnote{Long, A. A. (1986), \textit{Hellenistic Philosophy: Stoics, Epicureans, Sceptics}, University of California Press, Berkeley.}.

Về bản chất, \textbf{chủ nghĩa khắc kỉ} được hiểu là một hệ thống triết học thực hành, quan niệm rằng hạnh phúc đích thực không đến từ của cải hay danh vọng bên ngoài, mà từ việc rèn luyện đức hạnh và sống thuận theo tự nhiên. Cốt lõi của nó là thái độ bình thản chấp nhận những gì không thể thay đổi và nỗ lực làm chủ những gì nằm trong tầm kiểm soát của bản thân\footnote{Inwood, B. (Ed.) (2003), \textit{The Cambridge Companion to the Stoics}, Cambridge University Press, Cambridge.}.

Chủ nghĩa khắc kỉ ra đời trong bối cảnh khủng hoảng của thế giới Hy Lạp sau cái chết của Alexander Đại đế (323 TCN). Khi các thành bang Hy Lạp (Polis) sụp đổ, con người mất đi điểm tựa cộng đồng truyền thống và rơi vào trạng thái bất an hiện sinh. Trong hoàn cảnh đó, các triết gia khắc kỉ đã xây dựng một hệ thống triết học toàn diện nhằm giúp cá nhân tìm kiếm sự bình an nội tại bất chấp hoàn cảnh bên ngoài\footnote{Sellars, J. (2006), \textit{Stoicism}, University of California Press, Berkeley.}.

Nghiên cứu chủ nghĩa khắc kỉ không chỉ có ý nghĩa khảo cổ học về tư tưởng mà còn mang tính thời đại sâu sắc. Trong thế kỷ XXI, khi con người đối mặt với những khủng hoảng hiện sinh mới như biến đổi khí hậu, đại dịch toàn cầu, và sự bất ổn xã hội, các nguyên lý của chủ nghĩa khắc kỉ về việc phân biệt những gì ta có thể kiểm soát và những gì nằm ngoài tầm kiểm soát đang được tái khám phá như một phương thức sống hiệu quả. Đặc biệt, chủ nghĩa khắc kỉ đã trở thành nền tảng lý thuyết cho Liệu pháp Nhận thức Hành vi (CBT) – một trong những phương pháp tâm lý trị liệu hiệu quả nhất hiện nay\footnote{Robertson, D. (2010), \textit{The Philosophy of Cognitive-Behavioural Therapy (CBT)}, Karnac Books, London.}.

Do đó, việc nghiên cứu \textit{"Lịch sử triết học phương Tây: Chủ nghĩa khắc kỷ, những giá trị, hạn chế và vai trò đối với đời sống xã hội"} là cần thiết về cả phương diện lý luận lẫn thực tiễn.

\subsection*{2. Mục đích và nhiệm vụ nghiên cứu}
\addcontentsline{toc}{subsection}{2. Mục đích và nhiệm vụ nghiên cứu}

\textbf{Mục đích nghiên cứu:} Làm sáng tỏ quá trình hình thành, phát triển và nội dung tư tưởng cơ bản của chủ nghĩa khắc kỉ; đồng thời đánh giá những giá trị, hạn chế và vai trò của trường phái này đối với lịch sử tư tưởng phương Tây và đời sống xã hội hiện đại.

\textbf{Nhiệm vụ nghiên cứu:}
\begin{itemize}[leftmargin=1.5cm]
    \item Phân tích bối cảnh lịch sử và các tiền đề lý luận dẫn đến sự ra đời của chủ nghĩa khắc kỉ.
    \item Hệ thống hóa quá trình phát triển của chủ nghĩa khắc kỉ qua ba giai đoạn: Sơ kỳ, Trung kỳ và Hậu kỳ.
    \item Trình bày và phân tích nội dung tư tưởng cơ bản của chủ nghĩa khắc kỉ trên ba phương diện: Bản thể luận (Vật lý học), Nhận thức luận (Logic học) và Nhân bản luận (Đạo đức học).
    \item Đánh giá những giá trị tích cực và những hạn chế lịch sử của chủ nghĩa khắc kỉ.
    \item Làm rõ vai trò và ảnh hưởng của chủ nghĩa khắc kỉ đối với triết học phương Tây hiện đại và đời sống xã hội đương đại.
\end{itemize}

\subsection*{3. Đối tượng và phạm vi nghiên cứu}
\addcontentsline{toc}{subsection}{3. Đối tượng và phạm vi nghiên cứu}

\textbf{Đối tượng nghiên cứu:} Hệ thống tư tưởng triết học của chủ nghĩa khắc kỉ, bao gồm các quan điểm về vật lý học, logic học và đạo đức học của các triết gia đại diện như Zeno thành Citium, Cleanthes, Chrysippus, Seneca, Epictetus và Marcus Aurelius.

\textbf{Phạm vi nghiên cứu:}
\begin{itemize}[leftmargin=1.5cm]
    \item Về thời gian: Tập trung vào giai đoạn từ thế kỷ III TCN (khi Zeno sáng lập trường phái) đến thế kỷ II CN (thời kỳ Marcus Aurelius). Đồng thời, mở rộng xem xét ảnh hưởng của chủ nghĩa khắc kỉ đến triết học và tâm lý học hiện đại.
    \item Về không gian: Thế giới Hy-La (Hy Lạp cổ đại và Đế chế La Mã) cùng với những liên hệ đến bối cảnh toàn cầu đương đại.
    \item Về nội dung: Tập trung phân tích ba bộ phận cấu thành chủ nghĩa khắc kỉ (Logic, Vật lý, Đạo đức) và mối quan hệ biện chứng giữa chúng.
\end{itemize}

\subsection*{4. Cơ sở lý luận và phương pháp nghiên cứu}
\addcontentsline{toc}{subsection}{4. Cơ sở lý luận và phương pháp nghiên cứu}

\textbf{Cơ sở lý luận:}
\begin{itemize}[leftmargin=1.5cm]
    \item Quan điểm duy vật biện chứng và duy vật lịch sử của triết học Mác - Lênin về sự phát triển của tư tưởng triết học trong mối liên hệ với điều kiện kinh tế - xã hội\footnote{Hội đồng Trung ương (1999), \textit{Giáo trình triết học Mác – Lênin}, Nxb. Chính trị quốc gia, Hà Nội.}.
    \item Các tác phẩm gốc của triết gia khắc kỉ: \textit{Suy tưởng} (Marcus Aurelius), \textit{Giáo khoa thư} và \textit{Cẩm nang} (Epictetus), \textit{Những bức thư đạo đức} (Seneca).
    \item Các công trình nghiên cứu chuyên sâu về lịch sử triết học Hy Lạp và La Mã của các học giả trong và ngoài nước.
\end{itemize}

\textbf{Phương pháp nghiên cứu:}
\begin{itemize}[leftmargin=1.5cm]
    \item \textit{Phương pháp lịch sử - logic:} Xem xét sự hình thành và phát triển của chủ nghĩa khắc kỉ trong tiến trình lịch sử, đồng thời rút ra những nội dung lý luận cốt lõi.
    \item \textit{Phương pháp phân tích - tổng hợp:} Phân tích các khái niệm, phạm trù triết học cụ thể và tổng hợp thành hệ thống tư tưởng nhất quán.
    \item \textit{Phương pháp so sánh:} Đối chiếu chủ nghĩa khắc kỉ với các trường phái triết học cùng thời (Epicurus, Hoài nghi) và các trào lưu tư tưởng hiện đại (Chủ nghĩa hiện sinh, CBT).
    \item \textit{Phương pháp liên ngành:} Kết hợp triết học với lịch sử, tâm lý học và xã hội học để có cái nhìn toàn diện.
\end{itemize}

\subsection*{5. Ý nghĩa lý luận và thực tiễn}
\addcontentsline{toc}{subsection}{5. Ý nghĩa lý luận và thực tiễn}

\textbf{Ý nghĩa lý luận:}
\begin{itemize}[leftmargin=1.5cm]
    \item Góp phần làm phong phú thêm hiểu biết về lịch sử triết học phương Tây, đặc biệt là giai đoạn triết học Hy Lạp hóa – một giai đoạn ít được nghiên cứu sâu tại Việt Nam so với triết học cổ điển (Socrates, Plato, Aristotle)\footnote{Nguyễn Hữu Vui (1998), \textit{Lịch sử triết học}, Nxb. Chính trị quốc gia, Hà Nội.}.
    \item Cung cấp một góc nhìn mới về mối quan hệ giữa triết học cổ đại và tư tưởng hiện đại, chứng minh sức sống vượt thời gian của di sản triết học.
    \item Làm rõ những đóng góp của chủ nghĩa khắc kỉ vào sự phát triển của logic học, đạo đức học và nhận thức luận trong lịch sử tư tưởng nhân loại.
\end{itemize}

\textbf{Ý nghĩa thực tiễn:}
\begin{itemize}[leftmargin=1.5cm]
    \item Cung cấp cơ sở triết học cho việc rèn luyện bản lĩnh, sự kiên định và khả năng đối mặt với nghịch cảnh trong cuộc sống hiện đại\footnote{Irvine, W. B. (2008), \textit{A Guide to the Good Life: The Ancient Art of Stoic Joy}, Oxford University Press, New York.}.
    \item Giúp người đọc nhận thức rõ hơn về mối quan hệ biện chứng giữa thế giới nội tâm và thế giới bên ngoài, từ đó xây dựng lối sống cân bằng và ý nghĩa hơn.
    \item Có thể ứng dụng các nguyên lý khắc kỉ trong lĩnh vực giáo dục, tâm lý trị liệu và phát triển bản thân.
\end{itemize}

\subsection*{6. Kết cấu của tiểu luận}
\addcontentsline{toc}{subsection}{6. Kết cấu của tiểu luận}

Ngoài phần Mở đầu, Kết luận và Danh mục tài liệu tham khảo, nội dung chính của tiểu luận được trình bày trong 3 chương:

\textbf{Chương 1: Những cơ sở, tiền đề và quá trình hình thành, phát triển của chủ nghĩa khắc kỉ.}

Chương này trình bày bối cảnh lịch sử - xã hội của thế giới Hy Lạp thời kỳ Hy Lạp hóa, các tiền đề tư tưởng từ triết học Hy Lạp cổ điển (Socrates, phái Khuyển nho, Heraclitus), và quá trình hình thành, phát triển của chủ nghĩa khắc kỉ qua ba giai đoạn: Sơ kỳ (Zeno, Cleanthes, Chrysippus), Trung kỳ (Panaetius, Posidonius) và Hậu kỳ (Seneca, Epictetus, Marcus Aurelius).

\textbf{Chương 2: Nội dung tư tưởng cơ bản của chủ nghĩa khắc kỉ.}

Chương này đi sâu phân tích ba bộ phận cấu thành hệ thống triết học khắc kỉ: Bản thể luận (khái niệm Logos – lý tính vũ trụ, Pneuma – khí/hơi thở, thuyết định mệnh và tương thích); Nhận thức luận (lý thuyết về phantasia – ấn tượng, katalepsis – nắm bắt, và synkatathesis – sự phê chuẩn); Nhân bản luận và Đạo đức học (khái niệm oikeiosis – sự chiếm hữu, phân loại các điều dửng dưng, tư tưởng về đức hạnh và sự tự do nội tại).

\textbf{Chương 3: Những giá trị, hạn chế và vai trò của chủ nghĩa khắc kỉ đối với đời sống xã hội.}

Chương này đánh giá những giá trị tích cực và hạn chế lịch sử của chủ nghĩa khắc kỉ; phân tích vai trò của nó đối với triết học phương Tây hiện đại (ảnh hưởng đến Kitô giáo sơ khai, triết học hiện sinh, CBT); và liên hệ với thực tiễn đời sống xã hội đương đại, đặc biệt trong bối cảnh Việt Nam.
