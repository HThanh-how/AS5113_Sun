% ============ CHƯƠNG 1 ============
\section{NHỮNG CƠ SỞ, TIỀN ĐỀ VÀ QUÁ TRÌNH HÌNH THÀNH, PHÁT TRIỂN CỦA CHỦ NGHĨA KHẮC KỈ}

\subsection{Những cơ sở hình thành và phát triển Chủ nghĩa khắc kỉ}

\subsubsection{Bối cảnh lịch sử của thế giới Hy Lạp thời kỳ Hy Lạp hóa}

Chủ nghĩa khắc kỉ ra đời trong một giai đoạn chuyển biến lớn của lịch sử phương Tây – thời kỳ Hy Lạp hóa (Hellenistic period), kéo dài từ cái chết của Alexander Đại đế năm 323 TCN đến khi Đế chế La Mã sáp nhập Ai Cập năm 30 TCN. Đây là thời kỳ mà thế giới Hy Lạp trải qua những biến động chính trị, xã hội và văn hóa sâu sắc.

Trước đó, trong thời kỳ cổ điển, đời sống của người Hy Lạp gắn bó chặt chẽ với thành bang (Polis). Công dân Hy Lạp tìm thấy ý nghĩa cuộc sống trong việc tham gia vào các hoạt động chính trị, tôn giáo và văn hóa của cộng đồng. Triết học của Plato và Aristotle đều lấy Polis làm trung tâm, coi việc xây dựng một nhà nước lý tưởng là mục tiêu tối thượng của triết học chính trị.

Tuy nhiên, các cuộc chinh phạt của Alexander Đại đế đã phá vỡ cấu trúc thành bang truyền thống. Các vương quốc Hy Lạp hóa rộng lớn (Ptolemy ở Ai Cập, Seleucid ở Syria, Antigonid ở Macedonia) thay thế các thành bang nhỏ bé. Trong các đế chế đa sắc tộc này, cá nhân trở nên nhỏ bé và bất lực trước bộ máy quyền lực khổng lồ. Họ không còn có tiếng nói trong các quyết định chính trị như trước.

Sự sụp đổ của Polis kéo theo một cuộc khủng hoảng hiện sinh sâu sắc. Khi cá nhân không còn tìm thấy ý nghĩa cuộc sống trong đời sống công cộng, họ buộc phải quay vào bên trong, tìm kiếm sự bình an trong chính tâm hồn mình. Đây chính là mảnh đất màu mỡ cho sự nảy nở của các trường phái triết học đạo đức mới như Chủ nghĩa Khoái lạc (Epicureanism), Chủ nghĩa Hoài nghi (Skepticism) và Chủ nghĩa Khắc kỉ (Stoicism). Các trường phái này có chung mục tiêu: giúp cá nhân đạt được trạng thái bình an tâm hồn (Ataraxia) bất chấp sự hỗn loạn của thế giới bên ngoài.

\subsubsection{Bối cảnh kinh tế - xã hội}

Thời kỳ Hy Lạp hóa chứng kiến sự phát triển mạnh mẽ của thương mại và giao lưu văn hóa. Các tuyến đường buôn bán kết nối Hy Lạp với Ấn Độ, Trung Á và Ai Cập. Các thành phố quốc tế lớn như Alexandria, Antioch và Pergamon trở thành trung tâm văn hóa đa dạng, nơi các tư tưởng từ phương Đông và phương Tây giao thoa.

Sự phân hóa giàu nghèo ngày càng sâu sắc. Tầng lớp thương nhân giàu có nổi lên, trong khi nhiều nông dân mất đất và trở thành lao động làm thuê. Nô lệ chiếm một tỷ lệ lớn trong dân số. Trong bối cảnh này, triết học khắc kỉ với quan điểm về sự bình đẳng căn bản của mọi người (dựa trên việc tất cả đều chia sẻ Logos) đã thu hút được nhiều tầng lớp xã hội khác nhau, từ nô lệ (Epictetus) đến hoàng đế (Marcus Aurelius).

\subsection{Những tiền đề hình thành và phát triển Chủ nghĩa khắc kỉ}

\subsubsection{Tiền đề lý luận từ triết học Hy Lạp cổ điển}

Chủ nghĩa khắc kỉ không phải là một sáng tạo hoàn toàn mới mà là sự kế thừa và phát triển sáng tạo các di sản tư tưởng của triết học Hy Lạp cổ điển.

\textbf{Ảnh hưởng của Socrates:} Socrates (470-399 TCN) được coi là "ông tổ" tinh thần của chủ nghĩa khắc kỉ. Các triết gia khắc kỉ thường xuyên trích dẫn Socrates như một hình mẫu của hiền nhân (Sage). Những đóng góp của Socrates ảnh hưởng đến Stoicism bao gồm:
\begin{itemize}[leftmargin=1.5cm]
    \item Quan niệm rằng đức hạnh là tri thức, tội lỗi là sự ngu dốt (intellectualism đạo đức).
    \item Khẳng định rằng linh hồn quan trọng hơn thể xác, đời sống nội tâm quan trọng hơn của cải bên ngoài.
    \item Thái độ bình thản trước cái chết (thể hiện qua cái chết của Socrates khi uống thuốc độc).
    \item Phương pháp đối thoại (dialectic) để tìm kiếm chân lý.
\end{itemize}

\textbf{Ảnh hưởng của phái Khuyển nho (Cynicism):} Trường phái Khuyển nho do Antisthenes (học trò của Socrates) sáng lập và được Diogenes thành Sinope đưa đến cực đoan. Zeno, người sáng lập chủ nghĩa khắc kỉ, đã từng theo học Crates – một triết gia Khuyển nho. Những ảnh hưởng từ Cynicism bao gồm:
\begin{itemize}[leftmargin=1.5cm]
    \item Lý tưởng sống thuận theo tự nhiên (kata phusin).
    \item Coi thường của cải vật chất, danh vọng và khoái lạc.
    \item Nhấn mạnh sự tự túc (autarkeia) của cá nhân.
\end{itemize}
Tuy nhiên, Zeno đã làm mềm hóa tính cực đoan và phản xã hội của Cynicism, xây dựng một hệ thống triết học có tính hệ thống hơn và chấp nhận được hơn với xã hội thượng lưu.

\textbf{Ảnh hưởng của Heraclitus:} Heraclitus (khoảng 535-475 TCN) đã đề xuất khái niệm Logos như một nguyên lý lý tính chi phối vũ trụ. Ông cũng cho rằng lửa là nguyên tố căn bản của vũ trụ. Các triết gia khắc kỉ đã kế thừa:
\begin{itemize}[leftmargin=1.5cm]
    \item Khái niệm Logos như lý tính vũ trụ.
    \item Quan niệm vũ trụ vận hành theo quy luật tất yếu.
    \item Tư tưởng về sự thống nhất của các mặt đối lập.
    \item Hình ảnh ngọn lửa như nguyên lý chủ động của vũ trụ.
\end{itemize}

\subsubsection{Tiền đề lý luận từ triết học phương Đông}

Mặc dù vấn đề này còn đang được tranh luận, nhiều học giả cho rằng chủ nghĩa khắc kỉ có thể đã chịu ảnh hưởng gián tiếp từ các tư tưởng phương Đông thông qua các tuyến đường thương mại và giao lưu văn hóa thời Hy Lạp hóa.

Một số điểm tương đồng đáng chú ý giữa Stoicism và các triết học phương Đông:
\begin{itemize}[leftmargin=1.5cm]
    \item Với Đạo gia: Khái niệm Logos có nhiều điểm tương đồng với khái niệm Đạo (Tao) của Lão Tử – một nguyên lý vũ trụ vô hình nhưng chi phối mọi sự vật.
    \item Với Phật giáo: Quan niệm về sự vô thường của vạn vật, sự cần thiết phải buông bỏ các chấp trước để đạt được sự an lạc.
    \item Với Nho giáo: Nhấn mạnh nghĩa vụ xã hội và tu dưỡng đạo đức cá nhân.
\end{itemize}

\subsection{Quá trình hình thành và phát triển của Chủ nghĩa khắc kỉ}

\subsubsection{Thời kỳ Sơ kỳ (Early Stoa): Thế kỷ III-II TCN}

\textbf{Zeno thành Citium (334-262 TCN) – Người sáng lập:}

Zeno sinh ra tại Citium, một thành phố trên đảo Cyprus có nguồn gốc Phoenicia. Ông ban đầu là một thương gia giàu có. Bước ngoặt cuộc đời ông diễn ra khi tàu buôn của ông bị đắm và ông mất toàn bộ tài sản. Đến Athens, ông tình cờ đọc được cuốn \textit{Memorabilia} của Xenophon về Socrates tại một hiệu sách. Ông hỏi người chủ: "Tôi có thể tìm những người như thế này ở đâu?" Người chủ chỉ về phía Crates thành Thebes đang đi ngang qua. Từ đó, Zeno bắt đầu con đường triết học.

Zeno theo học Crates (phái Khuyển nho) nhưng cảm thấy sự phô trương và thiếu tế nhị của phái này không phù hợp với mình. Ông tiếp tục nghiên cứu với các triết gia thuộc Học viện (Academy) và trường phái Megarian. Khoảng năm 300 TCN, ông bắt đầu giảng dạy tại Stoa Poikile (Hành lang sơn) ở Athens – từ đó có tên gọi "Stoicism".

Zeno được mô tả là người có lối sống khắc khổ, làn da ngăm đen và tính cách nghiêm nghị. Ông là người đầu tiên phân chia triết học thành ba phần: Logic, Vật lý và Đạo đức. Ông định nghĩa mục tiêu cuối cùng của cuộc sống là "sống thuận theo tự nhiên" (homologoumenos te phusei zen).

\textbf{Cleanthes thành Assos (331-232 TCN) – Người kế nhiệm:}

Cleanthes là một võ sĩ quyền anh trước khi đến với triết học. Ông nổi tiếng với sự cần cù và lòng sùng kính. Ban ngày ông đi gánh nước thuê để kiếm sống, ban đêm ông học triết. Đóng góp lớn nhất của Cleanthes là bài thơ \textit{Thánh ca dâng Zeus} (Hymn to Zeus), một kiệt tác văn học - triết học, khẳng định niềm tin tuyệt đối vào sự sắp đặt thiêng liêng của vũ trụ và kêu gọi con người hòa hợp với ý chí của Zeus (tức Logos).

\textbf{Chrysippus thành Soli (280-207 TCN) – Người hệ thống hóa:}

Chrysippus được mệnh danh là "Bộ não của Stoa" hay "Người sáng lập thứ hai của chủ nghĩa khắc kỉ". Ông là một thiên tài logic học và là tác gia đa sản nhất trong số các triết gia cổ đại, viết hơn 700 cuốn sách (hầu hết đã thất lạc). Chrysippus đã:
\begin{itemize}[leftmargin=1.5cm]
    \item Bảo vệ học thuyết Stoa khỏi sự tấn công dữ dội từ Arcesilaus và Viện Hàn lâm Hoài nghi.
    \item Phát triển logic mệnh đề (propositional logic) đến mức hoàn thiện, tạo nên xương sống lý luận cho trường phái.
    \item Củng cố vững chắc thuyết định mệnh bằng các lập luận về nhân quả.
    \item Giải quyết nhiều nghịch lý logic như nghịch lý Người nói dối.
\end{itemize}
Có thể nói, phần lớn hệ thống triết học khắc kỉ như chúng ta biết ngày nay là do công lao của Chrysippus.

\subsubsection{Thời kỳ Trung kỳ (Middle Stoa): Thế kỷ II-I TCN}

Giai đoạn này đánh dấu sự chuyển giao của chủ nghĩa khắc kỉ từ Hy Lạp sang La Mã và sự điều chỉnh học thuyết để phù hợp với tư duy thực tiễn của người La Mã.

\textbf{Panaetius (185-110 TCN):} Ông là người bạn thân thiết của Scipio Aemilianus, vị tướng La Mã lừng danh đã tiêu diệt Carthage. Panaetius đã:
\begin{itemize}[leftmargin=1.5cm]
    \item Làm mềm hóa tính khắc khổ cực đoan của Sơ kỳ.
    \item Đưa vào các yếu tố của Plato và Aristotle.
    \item Tập trung nhiều hơn vào nghĩa vụ thực tiễn (officium) của con người trong xã hội.
    \item Tác phẩm của ông ảnh hưởng trực tiếp đến cuốn \textit{De Officiis} của Cicero.
\end{itemize}

\textbf{Posidonius (135-51 TCN):} Ông là một nhà bác học đa tài, nghiên cứu cả thiên văn, địa lý và lịch sử. Posidonius am hiểu tâm lý học con người sâu sắc hơn các bậc tiền bối, thừa nhận vai trò của các yếu tố phi lý trí trong tâm hồn và tìm cách dung hòa chúng với học thuyết khắc kỉ.

\subsubsection{Thời kỳ Hậu kỳ (Late Stoa / Roman Stoa): Thế kỷ I-II CN}

Đây là thời kỳ rực rỡ nhất của chủ nghĩa khắc kỉ về mặt văn học và ảnh hưởng xã hội. Các triết gia thời kỳ này tập trung chủ yếu vào đạo đức học thực hành, coi triết học là một phương thức sống hơn là một hệ thống lý thuyết trừu tượng.

\textbf{Seneca (4 TCN – 65 CN) – Chính khách và Nhà văn:}

Lucius Annaeus Seneca là một trong những nhân vật quyền lực và giàu có nhất đế chế La Mã. Ông là gia sư và sau là cố vấn cho hoàng đế Nero. Cuộc đời ông đầy rẫy mâu thuẫn giữa lý tưởng triết học về sự giản đơn và thực tế cuộc sống xa hoa chốn cung đình. Tuy nhiên, chính trong hoàn cảnh đó, ông viết nên những tác phẩm triết học sâu sắc nhất về sự vô thường của của cải, cách quản lý cơn giận (De Ira), giá trị của thời gian (De Brevitate Vitae), và nghệ thuật đối diện với cái chết.

Trong các tác phẩm như \textit{Những bức thư đạo đức gửi Lucilius}, Seneca bàn về mọi khía cạnh của đời sống: tình bạn, thái độ với nô lệ, tuổi già, và sự lưu vong. Ông dạy rằng: "Không phải người có quá ít, mà người mong muốn nhiều hơn mới là người nghèo". Cuối cùng, ông bị Nero ép tự sát. Cái chết bình thản của Seneca được ví như cái chết của Socrates.

\textbf{Epictetus (55-135 CN) – Từ nô lệ đến bậc thầy:}

Trái ngược hoàn toàn với Seneca, Epictetus sinh ra là một nô lệ tại Hierapolis (Thổ Nhĩ Kỳ ngày nay). Tên "Epictetus" trong tiếng Hy Lạp có nghĩa là "người được mua về". Ông bị chủ đánh gãy một chân, trở thành người tàn tật suốt đời. Sau khi được giải phóng, ông mở trường dạy triết học tại Nicopolis.

Triết lý của Epictetus cực kỳ thực tế và "cơ bắp". Ông không viết sách; những lời dạy của ông được học trò Arrian ghi lại trong \textit{Discourses} (Giáo khoa thư) và \textit{Enchiridion} (Cẩm nang). Epictetus nhấn mạnh tuyệt đối vào sự phân biệt giữa những gì ta có thể kiểm soát (ý kiến, phán đoán, ham muốn) và những gì nằm ngoài tầm kiểm soát (thân thể, tài sản, danh tiếng). Câu nói nổi tiếng của ông: "Đau đớn là không thể tránh khỏi, nhưng đau khổ là sự lựa chọn."

\textbf{Marcus Aurelius (121-180 CN) – Vị vua triết gia:}

Marcus Aurelius là hoàng đế của Đế chế La Mã hùng mạnh, người nắm trong tay quyền sinh sát cả thế giới. Tuy nhiên, cuộc đời ông là chuỗi ngày dài của chiến tranh biên ải chống lại các bộ tộc German, dịch bệnh (dịch Antonine) và sự phản bội (cuộc nổi loạn của Cassius). Ông viết tác phẩm \textit{Suy tưởng} (Meditations) không phải để xuất bản, mà như một cuốn nhật ký rèn luyện tinh thần cho chính mình giữa chiến trường.

\textit{Suy tưởng} là minh chứng sống động nhất cho việc một người có thể giữ gìn phẩm hạnh và lòng nhân ái ngay cả khi nắm giữ quyền lực tuyệt đối. Ông thường tự nhắc nhở: "Sáng sớm thức dậy, hãy tự nhủ: hôm nay ta sẽ gặp những kẻ tọc mạch, vô ơn, kiêu ngạo, lừa lọc, đố kỵ... Nhưng ta không thể giận họ, vì ta và họ cùng chung một bản thể."

\subsection*{Tiểu kết Chương 1}
\addcontentsline{toc}{subsection}{Tiểu kết Chương 1}

Chủ nghĩa khắc kỉ ra đời trong bối cảnh khủng hoảng của thế giới Hy Lạp thời kỳ Hy Lạp hóa, khi cấu trúc thành bang truyền thống sụp đổ và con người buộc phải tìm kiếm ý nghĩa cuộc sống trong chính nội tâm mình. Trường phái này kế thừa sáng tạo các tư tưởng của Socrates (đạo đức học), phái Khuyển nho (lý tưởng sống thuận tự nhiên) và Heraclitus (khái niệm Logos).

Qua hơn 500 năm phát triển, chủ nghĩa khắc kỉ đã trải qua ba giai đoạn: Sơ kỳ đặt nền móng lý thuyết (Zeno, Chrysippus), Trung kỳ thích ứng với thế giới La Mã (Panaetius, Posidonius), và Hậu kỳ tập trung vào thực hành đạo đức (Seneca, Epictetus, Marcus Aurelius). Điều đáng chú ý là các triết gia khắc kỉ đến từ mọi tầng lớp xã hội – từ nô lệ (Epictetus) đến hoàng đế (Marcus Aurelius) – chứng minh tính phổ quát của triết học này.
