\section{Mở đầu}

\subsection{Tính cấp thiết của đề tài}
Trong lịch sử tư tưởng phương Tây, Chủ nghĩa khắc kỉ (Stoicism) không chỉ là một trường phái triết học mà còn là một hệ tư tưởng có ảnh hưởng sâu rộng đến văn hóa, tôn giáo và chính trị của nền văn minh Hy-La. Ra đời trong bối cảnh hỗn loạn của thời kỳ Hy Lạp hóa (Hellenistic period), khi các cấu trúc xã hội truyền thống (Polis) sụp đổ, chủ nghĩa khắc kỉ đã cung cấp một điểm tựa tinh thần vững chắc cho con người, chuyển trọng tâm từ đời sống chính trị công cộng sang sự bình an nội tại.

Nghiên cứu chủ nghĩa khắc kỉ không chỉ có giá trị khảo cổ học về tư tưởng mà còn mang tính thời đại sâu sắc. Trong thế kỷ 21, khi con người đối mặt với những khủng hoảng hiện sinh mới, các khái niệm cốt lõi của Stoicism như "quyền kiểm soát", "sống thuận tự nhiên" và "chủ nghĩa thế giới" (cosmopolitanism) đang được tái khám phá mạnh mẽ như một liều thuốc tinh thần cho xã hội hiện đại.

\subsection{Mục tiêu và Nhiệm vụ nghiên cứu}
Bài tiểu luận này nhằm mục đích làm sáng tỏ bản chất hệ thống triết học khắc kỉ thông qua các nhiệm vụ cụ thể:
\begin{itemize}
    \item Phân tích bối cảnh lịch sử và sự phát triển của chủ nghĩa khắc kỉ qua ba giai đoạn: Sơ kỳ, Trung kỳ và Hậu kỳ.
    \item Hệ thống hóa ba bộ phận cấu thành nên triết học khắc kỉ: Vật lý học, Logic học và Đạo đức học, đồng thời chỉ ra mối quan hệ hữu cơ giữa chúng.
    \item Làm rõ các khái niệm triết học nền tảng bằng thuật ngữ gốc (Hy Lạp/Latin) như \textit{Logos}, \textit{Pneuma}, \textit{Oikeiosis}, v.v.
    \item Đánh giá vai trò và sự chuyển biến của chủ nghĩa khắc kỉ từ lý thuyết siêu hình sang thực hành đạo đức sống.
\end{itemize}

\subsection{Phương pháp nghiên cứu}
Đề tài sử dụng phương pháp duy vật biện chứng và duy vật lịch sử để xem xét sự ra đời của chủ nghĩa khắc kỉ trong mối liên hệ với điều kiện kinh tế - xã hội thời kỳ Hy Lạp hóa. Ngoài ra, phương pháp phân tích - tổng hợp và so sánh văn bản cũng được sử dụng để đối chiếu quan điểm giữa các triết gia khắc kỉ (như sự khác biệt giữa tính khắc khổ của Zeno và tính thực tiễn của Seneca).

\subsection{Cấu trúc của báo cáo}
Ngoài phần Mở đầu, Kết luận và Danh mục tài liệu tham khảo, nội dung chính của báo cáo gồm 2 chương:
\begin{itemize}
    \item \textbf{Chương 1: Lịch sử hình thành và phát triển của Chủ nghĩa khắc kỉ}: Trình bày tiến trình lịch sử và các đại diện tiêu biểu.
    \item \textbf{Chương 2: Hệ thống tư tưởng của Chủ nghĩa khắc kỉ}: Đi sâu phân tích ba trụ cột Logic học, Vật lý học và Đạo đức học.
\end{itemize}
