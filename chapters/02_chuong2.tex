% ============ CHƯƠNG 2 ============
\section{NỘI DUNG TƯ TƯỞNG CƠ BẢN CỦA CHỦ NGHĨA KHẮC KỈ}

Hệ thống triết học khắc kỉ được các triết gia sáng lập ví như một khu vườn: Vật lý học là đất đai và cây cối, Logic học là hàng rào bảo vệ, và Đạo đức học là hoa trái – mục đích cuối cùng của toàn bộ hệ thống. Trong chương này, chúng ta sẽ đi sâu phân tích ba bộ phận này theo thứ tự: Bản thể luận (Vật lý học), Nhận thức luận (Logic học) và Nhân bản luận (Đạo đức học)\footnote{Inwood, B. (Ed.) (2003), \textit{The Cambridge Companion to the Stoics}, Cambridge University Press, Cambridge, tr. 85-100.}.

\subsection{Bản thể luận của chủ nghĩa khắc kỉ}

\subsubsection{Quan niệm về Logos trong lịch sử triết học}

Khái niệm Logos (λόγος) là một trong những khái niệm phong phú nhất trong triết học Hy Lạp, với nhiều tầng nghĩa: lời nói, lý lẽ, lý tính, quy luật, tỷ lệ. Heraclitus (khoảng 535-475 TCN) là người đầu tiên sử dụng Logos như một thuật ngữ triết học để chỉ nguyên lý lý tính chi phối vũ trụ. Ông viết: \textit{"Logos này tồn tại mãi mãi, nhưng con người không hiểu được nó, dù trước khi nghe hay sau khi đã nghe."}

Các triết gia khắc kỉ đã kế thừa và phát triển khái niệm Logos của Heraclitus thành trụ cột trung tâm của hệ thống bản thể luận. Đối với họ, Logos không chỉ là một nguyên lý trừu tượng mà là một thực thể vật chất, sống động, thâm nhập vào mọi ngóc ngách của vũ trụ\footnote{Long, A. A. \& Sedley, D. N. (1987), \textit{The Hellenistic Philosophers, Volume 1}, Cambridge University Press, Cambridge, tr. 265-290.}.

\subsubsection{Thế giới như là Logos và Pneuma}

Chủ nghĩa khắc kỉ xác lập lập trường duy vật (Materialism – chủ nghĩa duy vật), khẳng định rằng chỉ có vật chất mới thực sự tồn tại, bởi lẽ chỉ vật chất mới có khả năng tác động và bị tác động. Tuy nhiên, vật chất trong quan niệm của họ không phải là vật chất vô tri mà bao hàm hai nguyên lý không thể tách rời:

\textbf{Nguyên lý thụ động (Passive principle):} Đây là vật chất vô hình, vô tính, như một chất nền chờ được nhào nặn. Plato gọi nó là "chất liệu" (hyle), chủ nghĩa khắc kỉ gọi là \textit{"vật chất không có tính chất"} (apoios ousia).

\textbf{Nguyên lý chủ động (Active principle):} Đây chính là Logos hay Thượng đế – lý tính sáng tạo và tổ chức vũ trụ. Nguyên lý chủ động này mang bản chất vật chất, được gọi là \textit{Pneuma} (Khí/Hơi thở). Pneuma là sự pha trộn của hai nguyên tố: Lửa (mang tính nóng, hoạt động) và Không khí (mang tính chuyển động, linh hoạt).

Pneuma thâm nhập vào mọi vật chất thụ động và tạo ra sự "căng" (tonos – sức căng) – một lực mang tính kép hướng đồng thời vừa ra ngoài vừa vào trong. Chính sự căng này tạo nên hình dạng, tính chất và sự kết dính của các vật thể:
\begin{itemize}[leftmargin=1.5cm]
    \item Trong các vật vô cơ (đá, kim loại), Pneuma tạo nên \textit{hexis} (sự kết dính).
    \item Trong thực vật, Pneuma tạo nên \textit{phusis} (bản tính tự nhiên).
    \item Trong động vật, Pneuma tạo nên \textit{psyche} (linh hồn).
    \item Trong con người, Pneuma tạo nên \textit{nous} (lý trí).
\end{itemize}

Như vậy, vũ trụ trong quan niệm khắc kỉ là một cơ thể sống thống nhất, nơi tất cả các bộ phận đều liên kết với nhau thông qua Pneuma. Đây là chủ nghĩa vật hoạt luận (Hylozoism) – quan niệm rằng vật chất tự thân mang tính sống và lý tính.

\subsubsection{Thuyết định mệnh và Thuyết tương thích}

Vì Logos (lý tính vũ trụ) chi phối tất cả, mọi sự kiện trong vũ trụ đều diễn ra theo chuỗi nhân quả tất yếu. Các triết gia khắc kỉ gọi đây là \textit{"số phận"} (heimarmene) hoặc \textit{"quan phòng"} (pronoia). Chrysippus viết: \textit{"Không có gì xảy ra mà không có nguyên nhân, cũng không có gì vượt ra ngoài bản chất và lý tính vũ trụ."}

Các triết gia khắc kỉ giải quyết vấn đề tự do bằng \textbf{thuyết tương thích} (Compatibilism) – cho rằng tự do và định mệnh có thể cùng tồn tại. Họ phân biệt giữa hai loại nguyên nhân:
\begin{itemize}[leftmargin=1.5cm]
    \item \textit{Nguyên nhân bên ngoài (proximate cause – nguyên nhân kích hoạt):} Các sự kiện thế giới tác động lên ta.
    \item \textit{Nguyên nhân bên trong (principal cause – nguyên nhân chính yếu):} Phản ứng của tâm trí ta đối với các sự kiện đó.
\end{itemize}

Chrysippus dùng hình ảnh một cái ống lăn: nếu bạn đẩy một cái ống hình trụ, nó sẽ lăn. Cái đẩy là nguyên nhân bên ngoài, nhưng việc nó lăn (thay vì trượt, như khối vuông) là do bản tính bên trong của nó. Tương tự, sự kiện bên ngoài kích hoạt phản ứng của ta, nhưng \textit{cách} ta phản ứng phụ thuộc vào tính cách và phán đoán của chính ta. Tự do nằm ở chỗ ta có quyền \textit{đồng ý} hoặc \textit{từ chối} các ấn tượng theo lý trí.

Một ẩn dụ nổi tiếng khác là hình ảnh con chó bị buộc vào cỗ xe ngựa: nếu con chó chạy theo xe thì cuộc hành trình sẽ êm ả; nếu nó chống cự thì vẫn bị kéo đi nhưng đau đớn. Kết quả cuối cùng là như nhau, nhưng trải nghiệm hoàn toàn khác biệt. Sự khôn ngoan nằm ở việc \textit{tự nguyện hòa hợp} với trật tự vũ trụ thay vì chống cự vô vọng\footnote{Long, A. A. (1986), \textit{Hellenistic Philosophy: Stoics, Epicureans, Sceptics}, University of California Press, Berkeley, tr. 160-175.}.

\subsubsection{Chu kỳ vũ trụ (Ekpyrosis)}

Một quan điểm độc đáo của chủ nghĩa khắc kỉ sơ kỳ là thuyết \textit{Ekpyrosis} (Đại hỏa tai). Theo đó, vũ trụ vận hành theo các chu kỳ vĩnh cửu gọi là \textit{"Đại năm"} (Great Year). Khi một Đại năm kết thúc, toàn bộ vũ trụ sẽ bùng cháy trong ngọn lửa thanh tẩy, mọi thứ trở về nguyên dạng lửa thuần khiết (tức Logos/Zeus – lý tính vũ trụ trong trạng thái thuần túy nhất). Sau đó, vũ trụ sẽ tái sinh từ ngọn lửa và lặp lại chính xác mọi sự kiện đã diễn ra trong chu kỳ trước.

\subsection{Nhận thức luận của chủ nghĩa khắc kỉ}

\subsubsection{Phantasia – Ấn tượng giác quan}

Quá trình nhận thức theo quan niệm khắc kỉ bắt đầu từ \textit{Phantasia} (ấn tượng giác quan). Khi một đối tượng bên ngoài tác động vào giác quan, nó để lại một "dấu ấn" trong tâm trí, giống như con dấu in lên sáp ong. Dấu ấn này gọi là ấn tượng (phantasia).

Tuy nhiên, không phải mọi ấn tượng (phantasia) đều đúng. Có những ấn tượng sai lệch do giác quan bị lừa (như nhìn cây gậy trong nước bị gãy khúc), do trạng thái tâm lý bất thường (như ảo giác trong cơn điên), hoặc do suy luận sai.

\subsubsection{Kataleptike Phantasia – Ấn tượng thấu hiểu}

Tiêu chuẩn của chân lý trong nhận thức luận khắc kỉ là \textit{Kataleptike Phantasia} (ấn tượng thấu hiểu/nắm bắt). Đây là một loại ấn tượng đặc biệt với các đặc điểm:
\begin{itemize}[leftmargin=1.5cm]
    \item Phát sinh từ một đối tượng thực sự tồn tại.
    \item Phản ánh chính xác đối tượng đó.
    \item Có tính rõ ràng, minh bạch đến mức tâm trí không thể chối từ.
\end{itemize}

\subsubsection{Sự phê chuẩn (Synkatathesis)}

Điểm then chốt trong nhận thức luận khắc kỉ là khái niệm \textit{synkatathesis} (sự phê chuẩn/đồng ý). Khi ấn tượng (phantasia) đến với tâm trí, ta có quyền tự do:
\begin{itemize}[leftmargin=1.5cm]
    \item \textbf{Đồng ý} (assent – phê chuẩn): Chấp nhận rằng ấn tượng này đúng.
    \item \textbf{Từ chối} (dissent – phủ nhận): Bác bỏ ấn tượng là sai.
    \item \textbf{Treo lơ lửng} (suspension – đình chỉ phán đoán): Không đưa ra phán đoán, chờ thêm bằng chứng.
\end{itemize}

Hegemonikon (phần chủ đạo của linh hồn, tức lý trí) là nơi đưa ra quyết định phê chuẩn. Đây chính là \textit{"ngai vàng"} của tự do con người. Sai lầm trong nhận thức (và theo đó, sai lầm đạo đức) xảy ra khi ta vội vàng đồng ý với những ấn tượng chưa được kiểm chứng kỹ lưỡng.

\subsubsection{Logic mệnh đề của chủ nghĩa khắc kỉ}

Chrysippus đã phát triển một hệ thống logic mệnh đề (propositional logic) phức tạp, khác biệt với logic hạn từ (term logic) của Aristotle. Logic khắc kỉ tập trung vào các mệnh đề hoàn chỉnh và mối quan hệ giữa chúng (nếu – thì, hoặc – hoặc, và, không).

Chrysippus đã xây dựng năm dạng suy luận hợp lệ cơ bản (indemonstrables – các suy luận tự hiển nhiên) mà từ đó có thể suy ra mọi suy luận hợp lệ khác:
\begin{enumerate}
    \item Nếu A thì B; A; vậy B (Modus ponens – Phương thức khẳng định)
    \item Nếu A thì B; không B; vậy không A (Modus tollens – Phương thức phủ định)
    \item Không (A và B); A; vậy không B
    \item A hoặc B; A; vậy không B
    \item A hoặc B; không A; vậy B (Disjunctive syllogism – Tam đoạn luận tuyển)
\end{enumerate}

Hệ thống logic này đã ảnh hưởng sâu sắc đến sự phát triển của logic học hiện đại vào thế kỷ XIX-XX\footnote{Sellars, J. (2006), \textit{Stoicism}, University of California Press, Berkeley, tr. 60-75.}.

Để làm rõ sự đóng góp của logic khắc kỉ, ta có thể so sánh ngắn gọn với logic của Aristotle:

\begin{table}[H]
    \centering
    \caption{So sánh Logic Aristotle và Logic Khắc kỷ}
    \begin{tabularx}{\linewidth}{|X|X|}
    \hline
    \textbf{Logic Aristotle (Logic hạn từ)} & \textbf{Logic Khắc kỷ (Logic mệnh đề)} \\
    \hline
    Cơ sở là khái niệm/hạn từ (terms). & Cơ sở là mệnh đề (propositions). \\
    \hline
    Ví dụ: Mọi A là B; C là A; Vậy C là B. & Ví dụ: Nếu trời mưa thì đất ướt; Trời mưa; Vậy đất ướt. \\
    \hline
    Nhấn mạnh quan hệ bao hàm giữa các lớp đối tượng. & Nhấn mạnh mối quan hệ nhân quả và điều kiện. \\
    \hline
    \end{tabularx}
\end{table}

\subsection{Nhân bản luận và Đạo đức học của chủ nghĩa khắc kỉ}

\subsubsection{Oikeiosis – Sự chiếm hữu/Thân thuộc}

Khái niệm nền tảng của đạo đức học khắc kỉ là \textit{Oikeiosis} (có thể dịch là "sự chiếm hữu", "sự thân thuộc", hay "sự nhận mình"). Đây là một quá trình tự nhiên, bẩm sinh, xảy ra ở mọi sinh vật.

\textbf{Oikeiosis cá nhân:} Ngay từ khi sinh ra, mọi sinh vật đều có bản năng tự bảo tồn. Đứa trẻ sơ sinh không cần được dạy vẫn biết tìm bầu sữa mẹ, tránh né nguy hiểm và đau đớn. Sinh vật coi bản thân và những gì thuộc về mình là \textit{"thân thuộc"} (oikeion), coi những gì đe dọa mình là \textit{"xa lạ"} (allotrion).

\textbf{Oikeiosis lý trí:} Khi con người trưởng thành, lý trí phát triển và thay đổi nội dung của oikeiosis (chiếm hữu). Con người nhận ra rằng điều quý giá nhất của mình không phải là thể xác mà là lý trí và đức hạnh. Do đó, mục tiêu của đời sống chuyển từ bảo tồn thể xác sang hoàn thiện đức hạnh.

\textbf{Oikeiosis xã hội:} Quá trình oikeiosis (chiếm hữu) cũng mở rộng ra bên ngoài, từ bản thân đến gia đình, bạn bè, cộng đồng và cuối cùng là toàn nhân loại. Hierocles (triết gia khắc kỉ thế kỷ II CN) mô tả điều này bằng hình ảnh các vòng tròn đồng tâm: vòng trong cùng là bản thân, các vòng tiếp theo là gia đình, họ hàng, láng giềng, đồng bào và cuối cùng là toàn bộ loài người. Nhiệm vụ của triết học là \textit{"kéo các vòng tròn về phía tâm"} – mở rộng sự quan tâm của ta đến tất cả mọi người như thể họ là người thân.

\begin{figure}[H]
    \centering
    \begin{tikzpicture}
        \draw (0,0) circle (0.5cm) node {\footnotesize Bản thân};
        \draw (0,0) circle (1.5cm) node[above=0.6cm] {\footnotesize Gia đình};
        \draw (0,0) circle (2.5cm) node[above=1.6cm] {\footnotesize Bạn bè/Xóm giềng};
        \draw (0,0) circle (3.5cm) node[above=2.6cm] {\footnotesize Đồng bào};
        \draw (0,0) circle (4.5cm) node[above=3.6cm] {\footnotesize Nhân loại};
        \draw[->, thick, bkblue] (4.5,0) -- (0.5,0) node[midway, below, fill=white] {\footnotesize Kéo các vòng tròn về tâm};
    \end{tikzpicture}
    \caption{Mô hình Oikeiosis (Vòng tròn đồng tâm) của Hierocles}
\end{figure}

Đây là cơ sở của tư tưởng \textbf{Chủ nghĩa thế giới} (Cosmopolitanism) – quan niệm rằng mọi người đều là công dân của một quốc gia chung: thế giới. Vì tất cả đều chia sẻ Logos (lý tính vũ trụ), nên không có sự phân biệt căn bản giữa người Hy Lạp và "man di", giữa tự do và nô lệ, giữa nam và nữ\footnote{Aurelius, M. (2006), \textit{Suy tưởng (Meditations)}, Nxb. Thế Giới, Hà Nội, Quyển VI.}.

\subsubsection{Đức hạnh là điều kiện cần và đủ cho hạnh phúc}

Các triết gia khắc kỉ phân chia mọi thứ thành ba loại:

\textbf{Tốt (Good):} Chỉ có đức hạnh (arete – đức hạnh) mới thực sự tốt. Bốn đức hạnh chính (cardinal virtues – tứ đức) là:
\begin{itemize}[leftmargin=1.5cm]
    \item \textit{Phronesis} (Trí tuệ thực tiễn): Biết cái gì đáng chọn và cái gì đáng tránh.
    \item \textit{Dikaiosyne} (Công bằng): Đối xử với mọi người theo phẩm giá của họ.
    \item \textit{Andreia} (Can đảm): Chịu đựng những gì phải chịu đựng.
    \item \textit{Sophrosyne} (Tiết độ): Kiểm soát ham muốn và xung động.
\end{itemize}
Các đức hạnh này không tách rời mà liên kết chặt chẽ với nhau – người có một đức hạnh phải có tất cả.

\textbf{Xấu (Bad):} Các thói xấu (vices – tật xấu) đối lập – ngu dốt, bất công, hèn nhát, phóng túng.

\textbf{Dửng dưng (Indifferents – Adiaphora):} Tất cả những thứ còn lại – sức khỏe, bệnh tật, giàu nghèo, danh tiếng, thậm chí sự sống và cái chết – đều không tốt cũng không xấu về mặt đạo đức. Chúng không ảnh hưởng đến hạnh phúc đích thực.

\subsubsection{Cảm xúc và Đam mê (Pathe)}

Chủ nghĩa khắc kỉ phân biệt rõ ràng giữa hai loại trạng thái cảm xúc:

\textbf{Pathe (Đam mê/Dục vọng):} Đây là những chuyển động phi lý trí của tâm hồn, được định nghĩa là \textit{"sự phán đoán sai lầm"} (false judgment). Có bốn loại pathe chính:
\begin{itemize}[leftmargin=1.5cm]
    \item \textit{Epithumia} (Ham muốn): Tin rằng điều gì đó tương lai là tốt trong khi thực ra nó chỉ là dửng dưng.
    \item \textit{Phobos} (Sợ hãi): Tin rằng điều gì đó tương lai là xấu.
    \item \textit{Hedone} (Khoái lạc): Tin rằng điều gì đó hiện tại là tốt trong khi không phải.
    \item \textit{Lupe} (Đau khổ): Tin rằng điều gì đó hiện tại là xấu.
\end{itemize}

Mục tiêu của người khắc kỉ là đạt được \textbf{Apatheia} (sự vắng mặt của các pathe – an nhiên). Từ "apatheia" không có nghĩa là "vô cảm" (apathy) như cách hiểu hiện đại, mà là sự giải thoát khỏi những cảm xúc phi lý trí, đau khổ không cần thiết.

\textbf{Eupatheia (Cảm xúc tốt):} Chỉ bậc hiền nhân mới có được những cảm xúc tích cực dựa trên phán đoán đúng đắn. Như vậy, người khắc kỉ không phải là hòn đá vô cảm, họ có cảm xúc nhưng là những cảm xúc được lý trí dẫn dắt.

\subsubsection{Quyền kiểm soát (Dichotomy of Control)}

Epictetus đã đúc kết toàn bộ đạo đức học khắc kỉ thành một nguyên tắc đơn giản nhưng sâu sắc trong câu mở đầu của \textit{Enchiridion} (Cẩm nang):

\textit{"Có những thứ phụ thuộc vào ta và có những thứ không phụ thuộc vào ta. Phụ thuộc vào ta là ý kiến, động lực, ham muốn, ghét bỏ – nói tóm lại, tất cả những gì là hành động của chính ta. Không phụ thuộc vào ta là thân thể, tài sản, danh tiếng, chức vụ – nói tóm lại, tất cả những gì không phải hành động của ta."}

Sự khôn ngoan nằm ở việc:
\begin{enumerate}
    \item Tập trung năng lượng vào những gì ta kiểm soát được.
    \item Buông bỏ lo lắng về những gì nằm ngoài tầm kiểm soát.
    \item Chấp nhận kết quả bên ngoài với sự bình thản.
\end{enumerate}

Epictetus dạy: \textit{"Đừng mong muốn sự vật diễn ra như bạn muốn, mà hãy muốn chúng diễn ra như chúng đang diễn ra, và bạn sẽ sống tốt."}\footnote{Epictetus (2019), \textit{Discourses, Fragments, Handbook}, Oxford University Press, Oxford, Enchiridion I-VIII.}

\subsection*{Tiểu kết Chương 2}
\addcontentsline{toc}{subsection}{Tiểu kết Chương 2}

Hệ thống tư tưởng của chủ nghĩa khắc kỉ thể hiện sự nhất quán chặt chẽ giữa ba bộ phận: Bản thể luận cung cấp nền tảng (vũ trụ được chi phối bởi Logos – lý tính vũ trụ), Nhận thức luận giải thích cơ chế (ta nắm bắt chân lý thông qua phantasia – ấn tượng và synkatathesis – sự phê chuẩn), và Đạo đức học đưa ra phương hướng sống (đức hạnh là điều tốt duy nhất, ta cần phân biệt những gì ta kiểm soát được).

Điểm độc đáo của chủ nghĩa khắc kỉ là sự kết hợp giữa chủ nghĩa duy vật nhất nguyên (monist materialism), thuyết định mệnh (determinism) và khẳng định tự do đạo đức (moral freedom). Họ cho rằng tự do không nằm ở việc thay đổi thế giới bên ngoài mà ở việc làm chủ thế giới bên trong – phán đoán, giá trị và phản ứng của chính ta.
