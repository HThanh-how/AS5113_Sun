\section{Lịch sử hình thành và phát triển}

Lịch sử chủ nghĩa khắc kỉ trải dài hơn 500 năm, từ thế kỷ 3 TCN đến thế kỷ 2 CN, thường được chia thành ba giai đoạn phát triển liên tục nhưng có những sắc thái riêng biệt.

\subsection{Sơ kỳ (Early Stoa)}
Giai đoạn này đặt nền móng lý thuyết cho toàn bộ hệ thống.
\begin{itemize}
    \item \textbf{Zeno thành Citium (334–262 TCN)}: Khởi đầu từ trường phái Cynic của Crates, Zeno đã tách ra và sáng lập trường phái riêng tại Stoa Poikile. Ông định nghĩa lại mục đích của triết học là "sống thuận theo tự nhiên". Zeno là người đầu tiên phân chia triết học thành ba phần: Logic, Vật lý và Đạo đức.
    \item \textbf{Cleanthes (331–232 TCN)}: Ông giữ gìn các học thuyết của Zeno với lòng sùng kính tôn giáo, nhấn mạnh khía cạnh thần học của Vật lý học khắc kỉ (coi Vũ trụ là hiện thân của Thượng đế).
    \item \textbf{Chrysippus (280–207 TCN)}: Được mệnh danh là "Bộ não của Stoa". Ông đã viết hơn 700 tác phẩm (hầu hết đã thất lạc) để hệ thống hóa, bảo vệ học thuyết khắc kỉ trước sự tấn công của Viện Hàn lâm (Academy). Ông phát triển logic mệnh đề (propositional logic) đến mức hoàn thiện, tạo nên xương sống lý luận cho trường phái.
\end{itemize}

\subsection{Trung kỳ (Middle Stoa)}
Giai đoạn này đánh dấu sự chuyển mình của chủ nghĩa khắc kỉ để thích nghi với thế giới La Mã. Hai đại diện chính là \textbf{Panaetius} và \textbf{Posidonius}. Họ đã làm mềm hóa tính khắc khổ cực đoan của Sơ kỳ, đưa vào các yếu tố của Plato và Aristotle, và tập trung nhiều hơn vào nghĩa vụ thực tiễn của con người trong xã hội.

\subsection{Hậu kỳ (Late/Roman Stoa)}
Đây là giai đoạn chủ nghĩa khắc kỉ trở thành hệ tư tưởng thống trị trong giới tinh hoa La Mã, tập trung chủ yếu vào Đạo đức học thực hành.
\begin{itemize}
    \item \textbf{Seneca (4 TCN–65 CN)}: Nhấn mạnh vào việc kiểm soát cơn giận, sử dụng thời gian và đối diện cái chết. Văn phong của ông giàu tính hùng biện và tâm lý học.
    \item \textbf{Epictetus (55–135 CN)}: Tập trung tuyệt đối vào tự do nội tại và sự phân biệt quyền kiểm soát (\textit{Dichotomy of Control}).
    \item \textbf{Marcus Aurelius (121–180 CN)}: Vị "Vua Triết gia", người thực hành khắc kỉ như một phương pháp rèn luyện tâm linh để giữ vững phẩm giá giữa quyền lực tối thượng.
\end{itemize}

\section{Hệ thống tư tưởng cốt lõi}
Các triết gia khắc kỉ so sánh triết học của họ như một sinh thể sống, trong đó ba bộ phận không thể tách rời: Vật lý học (xương thịt), Logic học (hệ thần kinh) và Đạo đức học (linh hồn).

\subsection{Vật lý học (Stoic Physics)}
Vật lý học khắc kỉ là nền tảng bản thể luận cho đạo đức học.
\subsubsection{Chủ nghĩa duy vật và Nhất nguyên luận}
Khắc kỉ giáo theo chủ nghĩa duy vật (Materialism), cho rằng chỉ có vật chất mới có khả năng tác động và bị tác động. Tuy nhiên, vật chất bao gồm hai nguyên lý:
\begin{enumerate}
    \item \textbf{Nguyên lý thụ động (Passive principle)}: Vật chất vô tri, đất đá.
    \item \textbf{Nguyên lý chủ động (Active principle)}: Gọi là \textbf{\textit{Logos}} (Lý tính) hay Thượng đế. Đây là ngọn lửa thiêng liêng (\textit{Pneuma} - Khí) thâm nhập vào mọi vật chất thụ động để tạo hình và duy trì sự sống.
\end{enumerate}

\subsubsection{Thuyết định mệnh và Thuyết tương thích}
Vì \textit{Logos} chi phối tất cả, mọi sự kiện trong vũ trụ đều diễn ra theo luật nhân quả tất yếu. Đây là thuyết định mệnh (Determinism). Tuy nhiên, khắc kỉ giáo chủ trương thuyết tương thích (Compatibilism): Tự do của con người không phải là thay đổi định mệnh, mà là khả năng lý trí \textit{chấp nhận} và \textit{hòa hợp} với định mệnh đó (như hình ảnh con chó bị buộc vào cỗ xe ngựa: nếu nó chạy theo xe thì hành trình êm ả, nếu nó chống cự thì vẫn bị kéo đi nhưng đau đớn).

\subsection{Logic học và Nhận thức luận}
\subsubsection{Phantasia và Kataleptike}
Quá trình nhận thức bắt đầu từ \textbf{\textit{Phantasia}} (ấn tượng giác quan). Tuy nhiên, không phải ấn tượng nào cũng đúng. Tiêu chuẩn của chân lý là \textbf{\textit{Kataleptike Phantasia}} (ấn tượng thấu hiểu/nắm bắt) - một ấn tượng rõ ràng, minh bạch đến mức tâm trí không thể chối từ.
\subsubsection{Sự phê chuẩn (Assent)}
Lý trí (\textit{Hegemonikon}) có quyền năng tối thượng: quyền \textbf{phê chuẩn} (assent) hoặc từ chối các ấn tượng. Sai lầm đạo đức thực chất là sai lầm trong nhận thức (phê chuẩn một ấn tượng sai lệch).

\subsection{Đạo đức học (Stoic Ethics)}
\subsubsection{Oikeiosis - Sự chiếm hữu/Thân thuộc}
Khái niệm nền tảng của đạo đức học khắc kỉ là \textbf{\textit{Oikeiosis}}. Ngay từ khi sinh ra, mọi sinh vật đều có bản năng tự bảo tồn và coi bản thân là "thân thuộc" với chính mình. Khi trưởng thành, lý trí phát triển, vòng tròn thân thuộc này mở rộng ra: từ bản thân đến gia đình, bè bạn, công dân và cuối cùng là toàn nhân loại. Đây là cơ sở của \textbf{Chủ nghĩa thế giới (Cosmopolitanism)}.

\subsubsection{Đức hạnh và Những điều dửng dưng}
Khắc kỉ giáo phân chia mọi thứ thành ba loại:
\begin{itemize}
    \item \textbf{Tốt (Good)}: Các đức hạnh (Trí tuệ, Công bằng, Can đảm, Tiết độ). Chỉ có cái tốt mới mang lại hạnh phúc.
    \item \textbf{Xấu (Bad)}: Các thói xấu (Ngu dốt, Bất công, Hèn nhát, Phóng túng).
    \item \textbf{Dửng dưng (Indifferents - \textit{Adiaphora})}: Bao gồm sức khỏe, tiền bạc, danh vọng, bệnh tật, nghèo đói. Chúng không ảnh hưởng đến hạnh phúc đích thực. Tuy nhiên, chúng được chia nhỏ thành:
    \begin{itemize}
        \item \textbf{Dửng dưng được ưu tiên (Preferred Indifferents)}: Sức khỏe, của cải (tự nhiên ta nên chọn chúng nếu không vi phạm đức hạnh).
        \item \textbf{Dửng dưng không được ưu tiên (Dispreferred Indifferents)}: Bệnh tật, cái chết.
    \end{itemize}
\end{itemize}

\subsubsection{Trạng thái Apatheia}
Mục tiêu của hiền nhân khắc kỉ là đạt được \textbf{\textit{Apatheia}}. Từ này không có nghĩa là "vô cảm" (apathy) như cách hiểu hiện đại, mà là \textbf{sự vắng mặt của những đam mê phi lý trí} (passions) như sợ hãi, ham muốn, đau khổ. Hiền nhân vẫn có những cảm xúc tốt đẹp (\textit{eupatheia}) như niềm vui, sự thận trọng và lòng từ tâm.
