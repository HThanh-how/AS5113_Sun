\section{Lịch sử hình thành và Các đại diện tiêu biểu}

Lịch sử của Chủ nghĩa khắc kỉ không chỉ là lịch sử của các tư tưởng mà là lịch sử của những con người đã sống và chết vì lý tưởng đó. Dưới đây là khảo cứu chi tiết về cuộc đời và đóng góp của các triết gia khắc kỉ qua các thời kỳ.

\subsection{Thời kỳ Sơ kỳ (The Early Stoa)}

\subsubsection{Zeno thành Citium (334–262 TCN) - Người sáng lập}
Zeno sinh ra tại Citium, đảo Cyprus, vốn là một thương gia giàu có. Bước ngoặt cuộc đời ông diễn ra sau một vụ đắm tàu kinh hoàng khiến ông mất toàn bộ tài sản khi đến Athens. Tại đây, ông tình cờ đọc được cuốn *Memorabilia* của Xenophon tại một hiệu sách và hỏi người chủ: "Tôi có thể tìm những người như thế này ở đâu?". Người chủ chỉ tay về phía Crates thành Thebes, một triết gia phái Khuyển nho (Cynic).

Zeno theo học Crates nhưng cảm thấy sự phô trương và thiếu tế nhị của phái Khuyển nho không phù hợp với mình. Ông tiếp tục nghiên cứu với các triết gia thuộc Học viện (Academy) và trường phái Megarian. Cuối cùng, vào khoảng năm 300 TCN, ông bắt đầu giảng dạy tại Stoa Poikile.
Tư tưởng của Zeno là sự tổng hòa giữa đạo đức học Khuyển nho (sống thuận tự nhiên) và vật lý học của Heraclitus (ngọn lửa vũ trụ). Ông được mô tả là người có lối sống khắc khổ, làn da ngăm đen và tính cách nghiêm nghị. Khi vua Antigonus của Macedonia mời ông làm cố vấn, Zeno từ chối nhưng cử học trò đi thay, thể hiện sự độc lập của triết học trước quyền lực.

\subsubsection{Cleanthes thành Assos (331–232 TCN) - Người mộ đạo}
Cleanthes là một võ sĩ quyền anh trước khi đến với triết học. Ông nổi tiếng với sự cần cù và chịu khó. Ban ngày ông đi gánh nước thuê cho các khu vườn để kiếm sống, ban đêm ông học triết. Chính sự lao động chân tay này đã khiến ông bị gọi ra tòa vì nghi ngờ làm sao một người không có tài sản lại có thể sống khỏe mạnh như vậy. Khi ông chứng minh mình sống bằng lao động lương thiện, tòa án Athens cảm phục định thưởng tiền nhưng Zeno khuyên ông từ chối.
Đóng góp lớn nhất của Cleanthes là bài thơ *Hymn to Zeus* (Thánh ca dâng Zeus), một kiệt tác văn học - triết học, khẳng định niềm tin tuyệt đối vào sự sắp đặt thiêng liêng của vũ trụ.

\subsubsection{Chrysippus thành Soli (280–207 TCN) - Nhà hệ thống hóa}
Nếu không có Chrysippus, sẽ không có Stoa. Ông là một thiên tài logic học và là người viết nhiều nhất trong số các triết gia cổ đại (hơn 700 cuộn sách). Ông đã bảo vệ học thuyết Stoa khỏi sự tấn công dữ dội từ Arcesilaus và Viện Hàn lâm Hoài nghi. Chrysippus đã xây dựng hệ thống logic mệnh đề phức tạp, giải quyết các nghịch lý (như nghịch lý Người nói dối) và củng cố vững chắc thuyết định mệnh bằng các lập luận về nhân quả. Cái chết của ông cũng rất kỳ lạ (cười đến chết khi thấy một con lừa ăn quả vả), phản ánh tính cách lập dị của một thiên tài.

\subsection{Thời kỳ Trung kỳ (The Middle Stoa)}

Giai đoạn này đánh dấu sự chuyển giao quyền lực từ Athens sang Rome. Các triết gia như \textbf{Panaetius} và \textbf{Posidonius} đã đưa triết học Stoa thâm nhập vào giới quý tộc La Mã.
\begin{itemize}
    \item \textbf{Panaetius (185–110 TCN)}: Ông là người bạn thân thiết của Scipio Aemilianus, vị tướng La Mã lừng danh. Panaetius đã lược bỏ các yếu tố vật lý học khô khan để tập trung vào nghĩa vụ (offiicium) của người công dân. Tác phẩm của ông ảnh hưởng trực tiếp đến cuốn *De Officiis* của Cicero sau này.
    \item \textbf{Posidonius (135–51 TCN)}: Ông là một nhà bác học đa tài, nghiên cứu cả thiên văn, địa lý và lịch sử. Posidonius am hiểu tâm lý học con người sâu sắc hơn các bậc tiền bối, thừa nhận vai trò của cảm xúc phi lý trí và tìm cách dung hòa chúng.
\end{itemize}

\subsection{Thời kỳ Hậu kỳ (The Late Stoa) - Thời đại La Mã}

Đây là thời kỳ rực rỡ nhất với những tác phẩm còn nguyên vẹn đến ngày nay, tập trung vào việc áp dụng triết học vào đời sống hàng ngày để mưu cầu hạnh phúc.

\subsubsection{Seneca (4 TCN – 65 CN) - Chính khách và Nhà văn}
Lucius Annaeus Seneca là một trong những nhân vật quyền lực và giàu có nhất đế chế La Mã. Ông là gia sư và sau là cố vấn cho bạo chúa Nero. Cuộc đời ông đầy rẫy mâu thuẫn giữa lý tưởng triết học về sự giản đơn và thực tế cuộc sống xa hoa chốn cung đình. Tuy nhiên, chính trong hoàn cảnh đó, ông viết nên những dòng triết học sâu sắc nhất về sự vô thường của của cải.
Trong *Những bức thư đạo đức gửi Lucilius*, Seneca bàn về mọi khía cạnh của đời sống: tình bạn, nô lệ, tuổi già, cái chết và sự lưu vong. Ông dạy rằng: "Không phải người có quá ít, mà người mong muốn nhiều hơn mới là người nghèo". Cuối cùng, ông bị Nero ép tự sát. Cái chết bình thản của Seneca được ví như cái chết của Socrates.

\subsubsection{Epictetus (55 – 135 CN) - Người nộ lệ giải phóng}
Trái ngược với Seneca, Epictetus sinh ra là một nô lệ tại Hierapolis (Thổ Nhĩ Kỳ ngày nay). Tên "Epictetus" trong tiếng Hy Lạp nghĩa là "người được mua về". Ông bị chủ đánh gãy một chân, trở thành người tàn tật suốt đời. Sau khi được giải phóng, ông mở trường dạy triết học tại Nicopolis.
Triết lý của Epictetus cực kỳ thực tế và "cơ bắp". Ông không viết sách; những lời dạy của ông được học trò Arrian ghi lại. Ông nhấn mạnh tuyệt đối vào quyền tự chủ ý chí. Câu nói nổi tiếng của ông: "Đau đớn là không thể tránh khỏi, nhưng đau khổ là sự lựa chọn".

\subsubsection{Marcus Aurelius (121 – 180 CN) - Vị vua triết gia}
Marcus Aurelius là hoàng đế của Đế chế La Mã hùng mạnh, người nắm trong tay quyền sinh sát cả thế giới. Tuy nhiên, cuộc đời ông là chuỗi ngày dài của chiến tranh biên ải, dịch bệnh và sự phản bội. Ông viết tác phẩm *Suy tưởng* (Meditations) không phải để xuất bản, mà như một cuốn nhật ký rèn luyện tinh thần cho chính mình giữa chiến trường.
Tác phẩm là minh chứng sống động nhất cho việc một người có thể giữ gìn phẩm hạnh và lòng nhân ái ngay cả khi nắm giữ quyền lực tuyệt đối. Ông thường tự nhắc nhở: "Sáng sớm thức dậy, hãy tự nhủ: hôm nay ta sẽ gặp những kẻ tọc mạch, vô ơn, kiêu ngạo, lừa lọc, đố kỵ... Nhưng họ như vậy vì họ không phân biệt được thiện ác. Ta cũng không thể giận họ, vì ta và họ cùng chung một bản thể".

\section{Kết luận chương}
Từ một thương gia mất sạch tài sản, một nô lệ tàn tật, đến một hoàng đế quyền uy, các đại diện của chủ nghĩa khắc kỉ chứng minh rằng triết học này dành cho tất cả mọi người, bất kể địa vị hay hoàn cảnh. Họ chia sẻ một niềm tin chung: đức hạnh là đủ cho hạnh phúc.
