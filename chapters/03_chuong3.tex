% ============ CHƯƠNG 3 ============
\section{NHỮNG GIÁ TRỊ, HẠN CHẾ VÀ VAI TRÒ CỦA CHỦ NGHĨA KHẮC KỈ ĐỐI VỚI ĐỜI SỐNG XÃ HỘI}

\subsection{Những giá trị và hạn chế của chủ nghĩa khắc kỉ}

\subsubsection{Những giá trị của chủ nghĩa khắc kỉ}

\textbf{Thứ nhất, giá trị về phương diện bản thể luận và thế giới quan:}

Chủ nghĩa khắc kỉ đã xây dựng một thế giới quan nhất quán và hợp lý trong bối cảnh lịch sử của nó. Quan niệm về một vũ trụ thống nhất, được chi phối bởi Logos (lý tính vũ trụ), đã cung cấp cho con người một cảm giác về trật tự và ý nghĩa giữa sự hỗn loạn của thế giới Hy Lạp hóa. Thuyết phiếm thần nội tại (Immanent pantheism) – quan niệm rằng Thượng đế hiện diện trong mọi sự vật – đã khắc phục được khoảng cách giữa thế giới thần linh và thế giới trần tục trong tôn giáo truyền thống.

Khái niệm Logos như một lý tính vũ trụ khách quan đã đặt nền móng cho tư duy khoa học sau này. Niềm tin rằng vũ trụ vận hành theo quy luật có thể nhận thức được đã khuyến khích con người tìm hiểu tự nhiên thay vì sợ hãi nó như sản phẩm của ý chí thất thường của thần linh\footnote{Long, A. A. (1986), \textit{Hellenistic Philosophy: Stoics, Epicureans, Sceptics}, University of California Press, Berkeley, tr. 200-220.}.

\textbf{Thứ hai, giá trị về phương diện logic học và nhận thức luận:}

Logic mệnh đề của Chrysippus là một đóng góp to lớn cho lịch sử logic học nhân loại. Trong gần 2000 năm, logic Aristotle thống trị thế giới phương Tây, nhưng logic khắc kỉ đã được tái phát hiện và đánh giá cao vào thế kỷ XIX-XX khi Gottlob Frege và Bertrand Russell phát triển logic hiện đại. Nhiều học giả cho rằng logic mệnh đề của chủ nghĩa khắc kỉ là tiền thân trực tiếp của logic ký hiệu (symbolic logic) hiện đại.

Lý thuyết về phantasia (ấn tượng) và synkatathesis (sự phê chuẩn) cũng rất quan trọng. Việc khẳng định rằng ta có quyền phê chuẩn hoặc từ chối các ấn tượng đã đặt nền móng cho quan niệm về chủ thể nhận thức có tính chủ động, không chỉ là tấm gương thụ động phản chiếu thế giới.

\textbf{Thứ ba, giá trị về phương diện đạo đức học:}

Đây là lĩnh vực mà chủ nghĩa khắc kỉ để lại di sản lớn nhất. Các giá trị đạo đức bao gồm:

\begin{itemize}[leftmargin=1.5cm]
    \item \textit{Tính phổ quát:} Quan niệm về sự bình đẳng căn bản của mọi người dựa trên việc tất cả đều chia sẻ Logos (lý tính vũ trụ) đã phá vỡ các rào cản giai cấp, dân tộc và giới tính của xã hội cổ đại. Tư tưởng \textit{"công dân thế giới"} (cosmopolitanism – chủ nghĩa thế giới) là tiền thân của các tuyên ngôn nhân quyền hiện đại.
    
    \item \textit{Tự do nội tại:} Quan niệm rằng tự do đích thực nằm ở bên trong, không phụ thuộc vào hoàn cảnh bên ngoài, đã mang lại hy vọng cho những người bị áp bức (nô lệ, tù nhân, người nghèo). Epictetus, bản thân là nô lệ, đã chứng minh rằng kẻ làm chủ có thể là nô lệ của ham muốn trong khi người nô lệ có thể là người tự do nhất.
    
    \item \textit{Trách nhiệm cá nhân:} Bằng cách đặt quyền phê chuẩn vào tay cá nhân, chủ nghĩa khắc kỉ đã nhấn mạnh rằng mỗi người phải chịu trách nhiệm về phản ứng của chính mình.
    
    \item \textit{Kiên cường (Resilience):} Các bài tập thực hành như \textit{"Premeditatio Malorum"} (Suy ngẫm trước về điều xấu), \textit{"Memento Mori"} (Hãy nhớ rằng bạn sẽ chết), hay \textit{"View from Above"} (Nhìn từ trên cao) đã cung cấp những công cụ cụ thể để rèn luyện sự kiên cường trước nghịch cảnh.
\end{itemize}

\textbf{Thứ tư, giá trị về phương diện thực hành:}

Khác với nhiều trường phái triết học trừu tượng, chủ nghĩa khắc kỉ coi triết học là một \textit{"nghệ thuật sống"} (techne peri ton bion). Các triết gia như Marcus Aurelius đã thực hành những gì họ giảng dạy – viết nhật ký để tự rèn luyện, đối xử nhân từ với kẻ thù, giữ vững phẩm hạnh giữa quyền lực tối thượng\footnote{Irvine, W. B. (2008), \textit{A Guide to the Good Life: The Ancient Art of Stoic Joy}, Oxford University Press, New York, tr. 180-210.}.

\subsubsection{Những hạn chế của chủ nghĩa khắc kỉ}

\textbf{Thứ nhất, hạn chế về phương diện bản thể luận:}

Chủ nghĩa duy vật của chủ nghĩa khắc kỉ, mặc dù tiến bộ so với nhị nguyên Plato, vẫn gặp khó khăn trong việc giải thích các hiện tượng tinh thần phức tạp. Việc quy giản mọi thứ (kể cả tâm hồn, đức hạnh, cảm xúc) thành vật chất (Pneuma) đã đơn giản hóa quá mức bản chất của thực tại.

Thuyết hồi quy vĩnh cửu (Ekpyrosis) – quan niệm rằng vũ trụ sẽ lặp lại chính xác mọi sự kiện – tuy hấp dẫn về mặt triết học nhưng thiếu cơ sở thực nghiệm và mang tính bi quan: nếu mọi thứ đã được định sẵn và sẽ lặp lại mãi mãi, thì nỗ lực của con người dường như vô nghĩa.

\textbf{Thứ hai, hạn chế về vấn đề định mệnh và tự do:}

Mặc dù các triết gia khắc kỉ cố gắng dung hòa định mệnh và tự do thông qua thuyết tương thích, nhưng lập luận của họ không hoàn toàn thuyết phục. Nếu mọi sự kiện (bao gồm cả suy nghĩ và quyết định của ta) đều được quy định bởi chuỗi nhân quả từ trước, thì \textit{"tự do"} mà họ đề cập có phải là tự do đích thực hay chỉ là ảo tưởng?

Các nhà phê bình (đặc biệt là Epicurus và Alexander of Aphrodisias) đã chỉ ra rằng thuyết định mệnh cứng của chủ nghĩa khắc kỉ làm suy yếu cơ sở của trách nhiệm đạo đức và khiến việc khen thưởng hay trừng phạt trở nên vô nghĩa.

\textbf{Thứ ba, hạn chế về quan niệm cảm xúc:}

Quan điểm cho rằng mọi pathe (đam mê, cảm xúc mãnh liệt) đều là \textit{"sai lầm"} cần loại bỏ đã bị nhiều triết gia sau này chỉ trích là thiếu tính nhân bản. Có những cảm xúc (như đau buồn khi mất người thân, phẫn nộ trước bất công) dường như là phản ứng tự nhiên và thậm chí cần thiết của con người.

\textbf{Thứ tư, hạn chế về tính thụ động chính trị:}

Mặc dù chủ nghĩa khắc kỉ khuyến khích tham gia đời sống công cộng và thực hiện nghĩa vụ xã hội, nhưng thuyết định mệnh và sự nhấn mạnh vào việc chấp nhận những gì nằm ngoài tầm kiểm soát đôi khi có thể dẫn đến thái độ thụ động trước bất công xã hội\footnote{Nguyễn Thị Minh Hương (2016), Cái nhìn duy ý chí của A.Schopenhauer về con người, \textit{Tạp chí Triết học}, 9 (304), Viện Triết học – Hà Nội, trang 55-62.}.

\subsection{Vai trò của chủ nghĩa khắc kỉ đối với triết học phương Tây hiện đại}

\subsubsection{Chủ nghĩa khắc kỉ với Kitô giáo sơ khai}

Mối quan hệ giữa chủ nghĩa khắc kỉ và Kitô giáo là một trong những chủ đề hấp dẫn nhất trong lịch sử tư tưởng. Khi Kitô giáo hình thành và phát triển trong thế giới Hy-La, nó đã hấp thụ nhiều thuật ngữ và tư tưởng từ chủ nghĩa khắc kỉ:

\begin{itemize}[leftmargin=1.5cm]
    \item \textit{Khái niệm Logos:} Tin Mừng theo Thánh John mở đầu: \textit{"Ban đầu có Ngôi Lời (Logos), Ngôi Lời ở cùng Thiên Chúa, và Ngôi Lời là Thiên Chúa."} Mặc dù Logos Kitô giáo có ý nghĩa khác (đây là Đức Kitô), nhưng việc sử dụng thuật ngữ này cho thấy ảnh hưởng của triết học Hy Lạp.
    
    \item \textit{Đạo đức khắc kỉ và đạo đức Kitô giáo:} Nhiều đức tính được chủ nghĩa khắc kỉ đề cao (tiết độ, can đảm, công bằng, trí tuệ) cũng là các đức tính cơ bản của Kitô giáo. Thánh Paul khi thuyết giáo ở Athens đã trích dẫn các nhà thơ khắc kỉ (Công vụ 17:28).
    
    \item \textit{Chủ nghĩa thế giới:} Quan niệm khắc kỉ về một cộng đồng nhân loại không phân biệt dân tộc, giai cấp phù hợp với thông điệp phổ quát của Kitô giáo: \textit{"Không còn Do Thái hay Hy Lạp, nô lệ hay tự do, nam hay nữ"} (Galatians 3:28).
\end{itemize}

Tuy nhiên, cũng có những khác biệt căn bản: chủ nghĩa khắc kỉ theo thuyết phiếm thần (Thượng đế = Vũ trụ) trong khi Kitô giáo theo thuyết hữu thần (Thượng đế siêu việt tạo dựng vũ trụ); chủ nghĩa khắc kỉ nhấn mạnh nỗ lực cá nhân đạt được đức hạnh trong khi Kitô giáo nhấn mạnh ân sủng thần linh.

\subsubsection{Chủ nghĩa khắc kỉ với phân tâm học của S. Freud}

Mặc dù Sigmund Freud (1856-1939) không trực tiếp kế thừa chủ nghĩa khắc kỉ, nhưng có những điểm giao thoa đáng chú ý:

\begin{itemize}[leftmargin=1.5cm]
    \item Cả hai đều quan tâm đến cơ chế nội tâm của con người và mối quan hệ giữa lý trí và các xung động phi lý trí.
    \item Chủ nghĩa khắc kỉ phân biệt giữa hegemonikon (phần chủ đạo của lý trí) và các pathe (đam mê); Freud phân biệt giữa Ego (Bản ngã), Id (Bản năng) và Super-ego (Siêu ngã).
    \item Cả hai đều cho rằng việc hiểu và quản lý các xung động nội tâm là chìa khóa của sức khỏe tâm thần.
\end{itemize}

Tuy nhiên, Freud bi quan hơn về khả năng của lý trí trong việc kiểm soát vô thức, trong khi chủ nghĩa khắc kỉ tin tưởng mạnh mẽ vào sức mạnh của logos (lý tính) cá nhân.

\subsubsection{Chủ nghĩa khắc kỉ xây dựng triết học tâm trạng khơi dòng cho sự ra đời và phát triển của chủ nghĩa hiện sinh}

Chủ nghĩa hiện sinh (Existentialism) thế kỷ XX (Kierkegaard, Heidegger, Sartre, Camus) có nhiều điểm giao thoa với chủ nghĩa khắc kỉ:

\begin{itemize}[leftmargin=1.5cm]
    \item \textit{Quan tâm đến ý nghĩa cuộc sống:} Cả hai đều đặt câu hỏi trung tâm: Làm thế nào để sống một cuộc đời có ý nghĩa giữa một thế giới có vẻ như phi lý hoặc thờ ơ với con người?
    
    \item \textit{Trách nhiệm cá nhân:} Sartre nói \textit{"Con người bị kết án phải tự do"} – ta buộc phải chọn lựa và chịu trách nhiệm. Điều này tương đồng với quan niệm khắc kỉ về synkatathesis (sự phê chuẩn).
    
    \item \textit{Đối mặt với cái chết:} Heidegger nói về \textit{"Sein-zum-Tode"} (Hữu-hướng-về-chết); Marcus Aurelius thường xuyên suy ngẫm về sự vô thường và cái chết như một phần của thực hành triết học.
    
    \item \textit{Amor Fati (Yêu số phận):} Khái niệm \textit{"yêu số phận"} của Nietzsche có nguồn gốc trực tiếp từ chủ nghĩa khắc kỉ – chấp nhận và thậm chí yêu thương mọi điều xảy ra như một phần của trật tự vũ trụ.
\end{itemize}

Tuy nhiên, chủ nghĩa khắc kỉ lạc quan hơn: họ tin rằng vũ trụ có lý tính và ý nghĩa (Logos), trong khi nhiều triết gia hiện sinh (đặc biệt Sartre và Camus) cho rằng vũ trụ là phi lý (absurd – vô nghĩa) và con người phải tự tạo ra ý nghĩa.

\subsubsection{Chủ nghĩa khắc kỉ với Liệu pháp Nhận thức Hành vi (CBT) và tâm lý học hiện đại}

Ảnh hưởng rõ ràng và quan trọng nhất của chủ nghĩa khắc kỉ trong thời đại hiện đại là trong lĩnh vực tâm lý trị liệu. Albert Ellis (sáng lập Rational Emotive Behavior Therapy - REBT vào năm 1955) và Aaron T. Beck (sáng lập Cognitive Therapy vào năm 1960s) đều thừa nhận sự kế thừa trực tiếp từ tư tưởng khắc kỉ, đặc biệt là Epictetus.

Nguyên lý cốt lõi của CBT phản ánh triết học khắc kỉ:
\begin{itemize}[leftmargin=1.5cm]
    \item \textbf{Epictetus}: \textit{"Con người không bị quấy nhiễu bởi sự việc, mà bởi quan điểm của họ về sự việc đó."}
    \item \textbf{CBT}: Sự kiện kích hoạt (A) không trực tiếp gây ra phản ứng cảm xúc (C). Chính niềm tin (B) của ta về sự kiện mới là nguyên nhân. Thay đổi niềm tin phi lý, ta sẽ thay đổi cảm xúc.
\end{itemize}

Kỹ thuật \textit{"cognitive restructuring"} (tái cấu trúc nhận thức) trong CBT chính là sự áp dụng hiện đại của kỷ luật phê chuẩn (synkatathesis) của chủ nghĩa khắc kỉ. CBT hiện là một trong những phương pháp tâm lý trị liệu được nghiên cứu kỹ lưỡng nhất và có hiệu quả được chứng minh trong điều trị trầm cảm, lo âu và nhiều vấn đề tâm lý khác\footnote{Robertson, D. (2010), \textit{The Philosophy of Cognitive-Behavioural Therapy (CBT): Stoic Philosophy as Rational and Cognitive Psychotherapy}, Karnac Books, London, tr. 55-80.}.

\subsubsection{Một số quan điểm cơ bản rút ra từ việc nghiên cứu chủ nghĩa khắc kỉ}

Qua việc phân tích ảnh hưởng của chủ nghĩa khắc kỉ, có thể rút ra một số quan điểm cơ bản:

\begin{enumerate}[leftmargin=1.5cm]
    \item \textbf{Tính liên tục của truyền thống triết học:} Các tư tưởng triết học không biến mất mà được tái khám phá và tái diễn giải qua các thời đại. Chủ nghĩa khắc kỉ \textit{"sống lại"} trong CBT là minh chứng cho sức sống vượt thời gian của triết học cổ đại.
    
    \item \textbf{Tính thực tiễn của triết học:} Chủ nghĩa khắc kỉ chứng minh rằng triết học không chỉ là tư biện trừu tượng mà có thể là công cụ cụ thể để cải thiện chất lượng cuộc sống.
    
    \item \textbf{Sự giao thoa Đông-Tây:} Nhiều nguyên lý khắc kỉ (buông bỏ, chấp nhận, tập trung vào hiện tại) có điểm tương đồng với Phật giáo và Đạo gia, cho thấy tính phổ quát của một số chân lý về đời sống con người.
    
    \item \textbf{Giới hạn của lý tính:} Các phê phán của Freud và Hiện sinh chủ nghĩa cho thấy niềm tin tuyệt đối vào lý tính của chủ nghĩa khắc kỉ cần được điều chỉnh – con người không hoàn toàn là sinh vật lý tính.
\end{enumerate}

\subsection{Vai trò của chủ nghĩa khắc kỉ trong lịch sử xây dựng và bảo vệ đất nước của các dân tộc}

\subsubsection{Sức mạnh của tinh thần khắc kỉ trong lịch sử thế giới}

Trong suốt lịch sử nhân loại, tinh thần khắc kỉ – dù có được gọi tên như vậy hay không – đã đóng vai trò quan trọng trong việc xây dựng và bảo vệ các quốc gia:

\textbf{La Mã cổ đại:} Chính sự kỷ luật và tinh thần khắc kỉ của quân đội La Mã đã giúp họ xây dựng một đế chế rộng lớn nhất trong lịch sử. Các vị tướng và hoàng đế theo chủ nghĩa khắc kỉ (như Marcus Aurelius) đã thể hiện lý tưởng \textit{"vua triết gia"} – lãnh đạo bằng đức hạnh và lý trí thay vì chỉ bằng vũ lực.

\textbf{Nước Anh thời Victoria:} Tinh thần \textit{"stiff upper lip"} (giữ vững môi trên – tức không biểu lộ cảm xúc trước nghịch cảnh) của người Anh có nguồn gốc sâu xa từ ảnh hưởng của chủ nghĩa khắc kỉ qua hệ thống giáo dục cổ điển. Tinh thần này đã giúp nước Anh vượt qua hai cuộc Thế chiến.

\textbf{Hoa Kỳ:} Nhiều Founding Fathers (như George Washington, Thomas Jefferson) đã nghiên cứu và ngưỡng mộ các triết gia khắc kỉ. Tinh thần tự lực, kiên cường trước nghịch cảnh đã trở thành một phần của \textit{"American Spirit"}. Đô đốc James Stockdale, tù binh chiến tranh Việt Nam trong 7 năm, đã sống sót nhờ áp dụng triết học của Epictetus và sau đó viết sách về điều này.

\textbf{Nhật Bản:} Tinh thần Bushido (Võ sĩ đạo) của các Samurai có nhiều điểm tương đồng với chủ nghĩa khắc kỉ: kỷ luật tự thân, chấp nhận cái chết, trách nhiệm với danh dự. Tinh thần này đã định hình nước Nhật qua nhiều thế kỷ và góp phần vào sự phục hồi thần kỳ sau Thế chiến II\footnote{Holiday, R. \& Hanselman, S. (2016), \textit{The Daily Stoic: 366 Meditations on Wisdom, Perseverance, and the Art of Living}, Portfolio, New York.}.

\subsubsection{Sức mạnh của tinh thần khắc kỉ trong lịch sử Việt Nam}

Mặc dù chủ nghĩa khắc kỉ là sản phẩm của văn hóa Hy-La và không trực tiếp ảnh hưởng đến Việt Nam, nhưng tinh thần tương đồng đã hiện diện xuyên suốt lịch sử dân tộc:

\textbf{Tinh thần kiên cường trước nghịch cảnh:}

Lịch sử Việt Nam là lịch sử của hàng nghìn năm chống ngoại xâm. Các anh hùng dân tộc đã thể hiện tinh thần kiên cường đáng kinh ngạc:
\begin{itemize}[leftmargin=1.5cm]
    \item Hai Bà Trưng, Bà Triệu đã phất cờ khởi nghĩa dù biết lực lượng yếu hơn địch gấp nhiều lần.
    \item Trần Hưng Đạo ba lần đánh bại quân Nguyên Mông – đế quốc hùng mạnh nhất thế giới đương thời – bằng chiến lược \textit{"lấy yếu thắng mạnh, lấy ít địch nhiều"}.
    \item Lê Lợi và Nguyễn Trãi kiên trì kháng chiến 10 năm trong điều kiện vô cùng khó khăn trước khi giành độc lập.
\end{itemize}

\textbf{Phân biệt những gì có thể và không thể kiểm soát:}

Tư tưởng về sự phân biệt giữa những gì có thể và không thể kiểm soát có thể thấy trong cách người Việt Nam đối mặt với thiên tai, chiến tranh. Thay vì than vãn số phận, người Việt tập trung vào những gì họ có thể làm: xây dựng lại sau bão lũ, tìm cách đánh địch phù hợp với điều kiện thực tế.

Câu tục ngữ \textit{"Còn da lông mọc, còn chồi nảy cây"} thể hiện sự kiên cường điển hình – chấp nhận mất mát nhưng không đầu hàng, tập trung vào việc xây dựng lại từ những gì còn lại.

\textbf{Tinh thần khắc kỉ trong tư tưởng Hồ Chí Minh:}

Chủ tịch Hồ Chí Minh – dù không nghiên cứu trực tiếp chủ nghĩa khắc kỉ – đã thể hiện nhiều phẩm chất tương đồng:
\begin{itemize}[leftmargin=1.5cm]
    \item Lối sống giản dị, coi thường vật chất – tương tự với quan niệm khắc kỉ về \textit{"adiaphora"} (những điều dửng dưng).
    \item Kiên định trước nghịch cảnh trong suốt 30 năm bôn ba tìm đường cứu nước.
    \item Tự chủ về tinh thần: trong nhà tù của Tưởng Giới Thạch, Người vẫn viết thơ và giữ vững tinh thần lạc quan.
    \item Quan niệm về đức hạnh cách mạng: Cần, Kiệm, Liêm, Chính – tương đồng với các đức hạnh căn bản của chủ nghĩa khắc kỉ.
\end{itemize}

\textbf{Ý nghĩa đối với Việt Nam đương đại:}

Trong bối cảnh hiện nay, các nguyên lý của chủ nghĩa khắc kỉ có thể đóng góp vào việc:
\begin{itemize}[leftmargin=1.5cm]
    \item Xây dựng văn hóa liêm chính và trách nhiệm cá nhân trong bối cảnh phòng chống tham nhũng.
    \item Rèn luyện sự kiên cường cho thế hệ trẻ trước áp lực học tập, cạnh tranh và thách thức của toàn cầu hóa.
    \item Hỗ trợ sức khỏe tâm thần trong bối cảnh tỷ lệ lo âu, trầm cảm gia tăng.
    \item Khuyến khích lối sống giản dị, bền vững, không chạy theo chủ nghĩa tiêu dùng cực đoan.
    \item Xây dựng tinh thần tự chủ, không phụ thuộc, phù hợp với đường lối đối ngoại độc lập của Việt Nam.
\end{itemize}

\subsection*{Tiểu kết Chương 3}
\addcontentsline{toc}{subsection}{Tiểu kết Chương 3}

Chủ nghĩa khắc kỉ để lại một di sản phong phú với nhiều giá trị vượt thời gian: quan niệm về sự bình đẳng của mọi người, tự do nội tại, trách nhiệm cá nhân, và các bài tập thực hành rèn luyện sự kiên cường. Tuy nhiên, nó cũng có những hạn chế lịch sử: thuyết định mệnh khó dung hòa với tự do ý chí, quan niệm về cảm xúc đôi khi thiếu tính nhân bản, và nguy cơ bị lợi dụng để biện minh cho sự thụ động.

Ảnh hưởng của chủ nghĩa khắc kỉ lan tỏa khắp lịch sử tư tưởng phương Tây: từ Kitô giáo sơ khai, qua phân tâm học của Freud, đến chủ nghĩa hiện sinh và Liệu pháp Nhận thức Hành vi hiện đại. Tinh thần khắc kỉ cũng đã đóng vai trò quan trọng trong việc xây dựng và bảo vệ đất nước của nhiều dân tộc trên thế giới\footnote{Nguyễn Hữu Vui (1998), \textit{Lịch sử triết học}, Nxb. Chính trị quốc gia, Hà Nội, tr. 150-180.}.

Đối với Việt Nam, mặc dù không có ảnh hưởng trực tiếp, nhưng tinh thần tương đồng với chủ nghĩa khắc kỉ đã hiện diện trong truyền thống kiên cường, bất khuất của dân tộc. Các nguyên lý của chủ nghĩa khắc kỉ có thể bổ sung cho truyền thống tư tưởng Việt Nam, góp phần vào việc xây dựng con người mới có bản lĩnh, có trách nhiệm, và có khả năng thích ứng với những thách thức của thời đại toàn cầu hóa.
