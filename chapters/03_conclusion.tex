\section{Kết luận}

\subsection{Đánh giá phê phán}
Chủ nghĩa khắc kỉ là một hệ thống triết học đồ sộ và nhất quán bậc nhất trong thế giới cổ đại. Ưu điểm lớn nhất của nó là tính thực tiễn và khả năng giải phóng con người khỏi khổ đau tinh thần thông qua sự tự chủ nội tại. Bằng cách định nghĩa hạnh phúc dựa trên đức hạnh thay vì may mắn, khắc kỉ giáo trao cho con người một "thành trì bất khả xâm phạm" trước nghịch cảnh.

Tuy nhiên, học thuyết này cũng vấp phải những giới hạn lịch sử và các chỉ trích:
\begin{itemize}
    \item \textbf{Tính cứng nhắc (Rigidity)}: Quan điểm "tất cả hoặc không gì cả" (một người không thể có một chút đức hạnh; nếu không phải là hiền nhân hoàn hảo, anh ta vẫn là kẻ điên rồ) bị coi là thiếu thực tế.
    \item \textbf{Vấn đề cảm xúc}: Mặc dù không cổ vũ sự vô cảm, nhưng việc coi các cảm xúc mãnh liệt (đau buồn khi mất người thân) là "phi lý trí" đôi khi bị xem là thiếu tính nhân bản.
    \item \textbf{Tính thụ động chính trị}: Mặc dù khuyến khích nghĩa vụ công dân, nhưng thuyết định mệnh đôi khi có thể dẫn đến thái độ chấp nhận bất công xã hội thay vì đấu tranh thay đổi nó.
\end{itemize}

\subsection{Di sản và Ảnh hưởng hiện đại}
Bất chấp những hạn chế, di sản của chủ nghĩa khắc kỉ vẫn sống động một cách đáng kinh ngạc.
\subsubsection{Cơ sở cho Tâm lý học hiện đại}
Ảnh hưởng rõ rệt nhất của chủ nghĩa khắc kỉ là trong lĩnh vực tâm lý trị liệu. Albert Ellis (sáng lập REBT) và Aaron T. Beck (sáng lập CBT - Liệu pháp Nhận thức Hành vi) đều thừa nhận sự kế thừa trực tiếp từ tư tưởng của Epictetus: \textit{"Con người không bị quấy nhiễu bởi sự việc, mà bởi quan điểm của họ về sự việc đó"}. Các kỹ thuật của CBT như tái cấu trúc nhận thức (cognitive restructuring) chính là sự áp dụng hiện đại của kỷ luật "Phê chuẩn" (Assent).

\subsubsection{Chủ nghĩa khắc kỉ hiện đại (Modern Stoicism)}
Ngày nay, chủ nghĩa khắc kỉ đang hồi sinh mạnh mẽ trong thung lũng Silicon, giới doanh nhân và thể thao đỉnh cao như một "hệ điều hành cho tâm trí" (operating system for the mind) để đối phó với áp lực cao và sự bất định. Những tác giả như William B. Irvine, Ryan Holiday hay Massimo Pigliucci đã đưa triết học này trở lại đời sống đại chúng, chứng minh sức sống vĩnh cửu của trí tuệ cổ xưa.

\section{Tài liệu tham khảo}
\begin{enumerate}
    \item Long, A. A., \& Sedley, D. N. (1987). \textit{The Hellenistic Philosophers} (Vol. 1). Cambridge University Press.
    \item Sellars, J. (2006). \textit{Stoicism}. University of California Press.
    \item Inwood, B. (Ed.). (2003). \textit{The Cambridge Companion to the Stoics}. Cambridge University Press.
    \item Aurelius, M. (2006). \textit{Suy tưởng}. (Nguyễn Duy Nhiên dịch). NXB Thế Giới.
    \item Epictetus. (2019). \textit{Giáo khoa thư (Discourses)}. NXB Tri Thức.
    \item Seneca. (2018). \textit{Những bức thư đạo đức (Moral Letters)}. NXB Tri Thức.
    \item Robertson, D. (2010). \textit{The Philosophy of Cognitive-Behavioural Therapy (CBT): Stoic Philosophy as Rational and Cognitive Psychotherapy}. Karnac Books.
\end{enumerate}
