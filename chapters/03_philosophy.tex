\section{Hệ thống học thuyết (Doctrine)}

Hệ thống triết học khắc kỉ được xây dựng chặt chẽ như một tòa nhà kiên cố, nơi Vật lý học cung cấp nền móng, Logic học cung cấp khung sườn và Đạo đức học là không gian sống. Trong chương này, chúng ta sẽ đi sâu vào phân tích các khái niệm kỹ thuật của từng bộ phận.

\subsection{Vật lý học (Stoic Physics)}
Vật lý học của Stoicism không chỉ là nghiên cứu về thế giới tự nhiên mà còn là nghiên cứu về bản chất thần thánh của thực tại.

\subsubsection{Nguyên lý Logos và Pneuma}
Người khắc kỉ tin rằng vũ trụ là một thể thống nhất hữu cơ (organic unity). Mọi vật chất đều được thấm đẫm bởi một "hơi thở" hay "tinh thần" gọi là \textbf{\textit{Pneuma}}.
\begin{itemize}
    \item \textbf{Pneuma} là sự kết hợp của Lửa (nhiệt) và Không khí (chuyển động). Nó tạo ra "tensão" (sức căng) giữ cho các vật thể có hình dạng và tính chất.
    \item \textbf{Logos} (Lý tính vũ trụ) là khía cạnh lý trí của Pneuma. Logos hoạt động như quy luật sắp xếp trật tự thế giới. Trong con người, một phần nhỏ của Logos trở thành lý trí cá nhân. Do đó, sống theo lý trí chính là sống thuận theo tự nhiên (vũ trụ).
\end{itemize}

\subsubsection{Chu kỳ Vũ trụ (Ekpyrosis)}
Một quan điểm độc đáo của Stoicism là thuyết \textbf{\textit{Ekpyrosis}} (Đại hỏa tai). Các triết gia sơ kỳ tin rằng vũ trụ vận hành theo chu kỳ vĩnh cửu. Sau một khoảng thời gian dài (Đại năm), vũ trụ sẽ bùng cháy trong ngọn lửa thanh tẩy, mọi thứ trở về nguyên dạng lửa ban đầu. Sau đó, vũ trụ sẽ tái sinh và lặp lại chính xác mọi sự kiện đã diễn ra trong chu kỳ trước (Thuyết hồi quy vĩnh cửu - Eternal Recurrence). Điều này củng cố quan điểm định mệnh: mọi thứ đã, đang và sẽ diễn ra đúng như nó phải thế.

\subsection{Logic học và Nhận thức luận}
Logic học Stoa được đánh giá cao và phức tạp hơn logic học của Aristotle ở chỗ nó là logic học mệnh đề (propositional logic) thay vì logic học hạn từ (term logic).

\subsubsection{Quy trình nhận thức}
Người khắc kỉ mô tả quá trình tâm trí nắm bắt sự thật qua hình ảnh bàn tay:
\begin{enumerate}
    \item \textbf{Phantasia (Ấn tượng)}: Xòe bàn tay ra. Các giác quan tiếp nhận hình ảnh từ thế giới bên ngoài.
    \item \textbf{Assent (Sự phê chuẩn)}: Hơi co ngón tay lại. Tâm trí xem xét ấn tượng và đồng ý rằng nó đúng. Đây là bước quan trọng nhất vì nó nằm trong quyền kiểm soát của ta.
    \item \textbf{Katalepsis (Sự nắm bắt)}: Nắm chặt tay lại. Sự hiểu biết chắc chắn về một đối tượng cụ thể.
    \item \textbf{Knowledge (Tri thức khoa học)}: Nắm chặt tay kia bao lấy nắm tay này. Một hệ thống các tri thức không thể lay chuyển, chỉ có ở bậc Hiền nhân (Sage).
\end{enumerate}

\subsubsection{Các nghịch lý (Paradoxes)}
Chrysippus đã nghiên cứu nhiều nghịch lý nổi tiếng, ví dụ như nghịch lý "Sorites" (Đống cát): Bỏ một hạt cát đi thì vẫn là đống cát, vậy bỏ đến hạt thứ bao nhiêu thì nó không còn là đống cát? Điều này cho thấy sự mơ hồ của ngôn ngữ và khái niệm.

\subsection{Đạo đức học (Stoic Ethics)}
Đạo đức học là mục tiêu cuối cùng của triết học Stoa.

\subsubsection{Oikeiosis (Sự chiếm hữu/Thân thuộc)}
Đây là nền tảng sinh học của đạo đức Stoa.
\begin{itemize}
    \item Giai đoạn đầu: Đứa trẻ sinh ra có bản năng tự bảo tồn, nó coi bản thân là cái gì đó "thân thuộc" và quý giá với chính nó.
    \item Giai đoạn sau: Khi lý trí phát triển, con người nhận ra rằng lý trí và đức hạnh mới là cái quý giá nhất của bản thân.
    \item Giai đoạn xã hội: Con người nhận ra những người khác cũng chia sẻ chung lý trí (Logos) với mình. Do đó, người khắc kỉ mở rộng vòng tròn quan tâm (circles of concern) từ bản thân -> gia đình -> hàng xóm -> công dân -> toàn nhân loại. Đây là nguồn gốc của tư tưởng "Công dân thế giới".
\end{itemize}

\subsubsection{Đức hạnh là điều kiện cần và đủ}
Các trường phái khác như Peripatetic (Aristotle) cho rằng hạnh phúc cần cả đức hạnh VÀ các điều kiện bên ngoài (giàu có, sức khỏe). Stoicism phản đối kịch liệt. Họ khẳng định "Đức hạnh là đủ" (Virtue is sufficient for happiness). Thậm chí khi bị tra tấn, bỏ tù, bệnh tật, người Hiền nhân vẫn có thể hạnh phúc (Eudaimonia) vì phẩm giá đạo đức của ông ta không bị tổn hại.

\subsubsection{Cảm xúc và Đam mê (Pathe)}
Stoicism phân biệt hai loại trạng thái cảm xúc:
\begin{itemize}
    \item \textbf{Pathe (Đam mê)}: Là những chuyển động phi lý trí của tâm hồn, là sai lầm trong phán đoán. Có 4 loại chính: Khát khao (Lust), Sợ hãi (Fear), Vui sướng (Delight - vui vì cái xấu), Đau khổ (Distress). Mục tiêu là loại bỏ chúng.
    \item \textbf{Eupatheia (Cảm xúc tốt)}: Chỉ hiền nhân mới có. Bao gồm: Mong muốn (Wish - muốn điều tốt), Thận trọng (Caution - tránh điều xấu), Vui mừng (Joy - vui vì điều đức hạnh).
\end{itemize}
Như vậy, người khắc kỉ không phải là hòn đá vô cảm, họ chỉ loại bỏ những cảm xúc độc hại và nuôi dưỡng những cảm xúc tích cực dựa trên lý trí.
