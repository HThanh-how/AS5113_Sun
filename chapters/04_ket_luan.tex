% ============ KẾT LUẬN ============

Qua quá trình nghiên cứu đề tài "Lịch sử triết học phương Tây: Chủ nghĩa khắc kỉ và vai trò của nó đối với đời sống xã hội", chúng tôi rút ra những kết luận sau:

\textbf{Thứ nhất,} chủ nghĩa khắc kỉ là một trong những trường phái triết học quan trọng nhất của thế giới Hy-La, ra đời trong bối cảnh khủng hoảng của thời kỳ Hy Lạp hóa khi các cấu trúc xã hội truyền thống sụp đổ và con người buộc phải tìm kiếm ý nghĩa cuộc sống trong nội tâm [5], [7]. Trường phái này kế thừa sáng tạo các tư tưởng của Socrates, phái Khuyển nho và Heraclitus [8], đồng thời phát triển thành một hệ thống triết học toàn diện trải dài hơn 500 năm với ba giai đoạn: Sơ kỳ (Zeno, Chrysippus), Trung kỳ (Panaetius, Posidonius) và Hậu kỳ (Seneca, Epictetus, Marcus Aurelius) [13].

\textbf{Thứ hai,} hệ thống tư tưởng của chủ nghĩa khắc kỉ thể hiện sự nhất quán chặt chẽ giữa ba bộ phận [5]: Vật lý học xây dựng thế giới quan duy vật nhất nguyên với Logos là lý tính vũ trụ chi phối mọi sự vật [8]; Logic học phát triển lý thuyết về phantasia (ấn tượng) và synkatathesis (phê chuẩn) cùng với hệ thống logic mệnh đề tiên tiến [13]; Đạo đức học đề cao đức hạnh như điều tốt duy nhất, phân biệt những gì ta có thể kiểm soát (thế giới nội tâm) với những gì nằm ngoài tầm kiểm soát (thế giới bên ngoài) [2], [4]. Khái niệm Oikeiosis giải thích nguồn gốc tự nhiên của đạo đức và mở đường cho tư tưởng "công dân thế giới" (cosmopolitanism) [1], [5].

\textbf{Thứ ba,} chủ nghĩa khắc kỉ có những giá trị vượt thời gian đáng được ghi nhận: quan niệm về sự bình đẳng căn bản của mọi người [1], tự do nội tại không phụ thuộc hoàn cảnh bên ngoài [2], trách nhiệm cá nhân đối với phản ứng của chính mình [4], và các bài tập thực hành rèn luyện sự kiên cường [6], [11]. Tuy nhiên, nó cũng có những hạn chế lịch sử: thuyết định mệnh khó dung hòa hoàn toàn với tự do ý chí [7], quan niệm về cảm xúc đôi khi thiếu tính nhân bản [10], và tiêu chuẩn "hiền nhân hoàn hảo" quá xa vời với đời sống thực [6].

\textbf{Thứ tư,} ảnh hưởng của chủ nghĩa khắc kỉ lan tỏa khắp lịch sử tư tưởng phương Tây [5]. Kitô giáo sơ khai đã hấp thụ nhiều khái niệm và đức tính Stoic [5], [8]. Triết học hiện đại (đặc biệt là Chủ nghĩa hiện sinh) chia sẻ một số mối quan tâm của Stoicism về ý nghĩa cuộc sống và trách nhiệm cá nhân [10]. Quan trọng nhất, Liệu pháp Nhận thức Hành vi (CBT) – một trong những phương pháp tâm lý trị liệu hiệu quả nhất hiện nay – thừa nhận sự kế thừa trực tiếp từ tư tưởng của Epictetus [2], [12].

\textbf{Thứ năm,} trong bối cảnh thế kỷ XXI đầy biến động với khủng hoảng khí hậu, đại dịch, bất ổn xã hội và quá tải thông tin, các nguyên lý của chủ nghĩa khắc kỉ đang được tái khám phá như một phương thức sống hiệu quả [4], [11]. Sự phục hưng của "Stoicism hiện đại" cho thấy các triết gia cổ đại vẫn có nhiều điều để dạy chúng ta về cách giữ vững phẩm giá và sự bình an giữa một thế giới bất định [6].

Đối với Việt Nam, mặc dù chủ nghĩa khắc kỉ không có ảnh hưởng trực tiếp, nhưng tinh thần tương đồng với Stoicism – sự kiên cường, bất khuất, tập trung vào những gì có thể kiểm soát – đã hiện diện trong truyền thống dân tộc [3], [9]. Các nguyên lý khắc kỉ có thể bổ sung cho truyền thống tư tưởng Việt Nam, góp phần xây dựng con người mới có bản lĩnh, có trách nhiệm, và có khả năng thích ứng với những thách thức của thời đại toàn cầu hóa [3], [9].

Kết quả nghiên cứu cho thấy nhiệm vụ đặt ra của đề tài đã được hoàn thành. Chúng tôi hy vọng tiểu luận này góp phần làm phong phú thêm hiểu biết về lịch sử triết học phương Tây và cung cấp những gợi ý có giá trị cho việc xây dựng lối sống có ý nghĩa trong thời đại ngày nay.
