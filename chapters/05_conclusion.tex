\section{Kết luận và Di sản}

\subsection{So sánh Chủ nghĩa Khắc kỉ và Chủ nghĩa Khoái lạc (Epicureanism)}
Để hiểu rõ vị trí của Stoicism, ta cần đặt nó cạnh đối thủ cạnh tranh lớn nhất thời bấy giờ: Chủ nghĩa Khoái lạc của Epicurus.
\begin{table}[H]
\centering
\begin{tabularx}{\textwidth}{|X|X|X|}
\hline
\textbf{Tiêu chí} & \textbf{Chủ nghĩa Khắc kỉ (Stoicism)} & \textbf{Chủ nghĩa Khoái lạc (Epicureanism)} \\
\hline
\textbf{Mục đích tối hậu} & Đức hạnh (Virtue). Sống thuận theo tự nhiên và lý trí. & Khoái lạc (Pleasure) - được hiểu là sự vắng mặt của đau đớn (Aponia) và tĩnh tâm (Ataraxia). \\
\hline
\textbf{Quan niệm về Thần linh} & Thượng đế/Logos hiện hữu và chi phối mọi thứ (Chủ nghĩa quan phòng). Vũ trụ có mục đích. & Thần linh tồn tại nhưng không can thiệp vào đời sống con người. Vũ trụ là ngẫu nhiên của các nguyên tử. \\
\hline
\textbf{Thái độ xã hội} & Tham gia tích cực vào đời sống chính trị (để thực hiện nghĩa vụ công dân). & Ẩn dật, tránh xa chính trị ("Hãy sống ẩn dật") để giữ tâm hồn thanh thản. \\
\hline
\textbf{Cảm xúc} & Loại bỏ đam mê phi lý trí. Kiểm soát cảm xúc bằng lý trí. & Tiết chế ham muốn. Tìm kiếm những niềm vui giản đơn, tự nhiên. \\
\hline
\end{tabularx}
\caption{So sánh Stoicism và Epicureanism}
\end{table}
Mặc dù khác biệt về xuất phát điểm và phương pháp, cả hai trường phái đều hướng tới mục tiêu chung của triết học Hy Lạp hóa: sự bình an trong tâm hồn (Ataraxia).

\subsection{Đánh giá phê phán}
Chủ nghĩa khắc kỉ là một hệ thống triết học đồ sộ và nhất quán bậc nhất trong thế giới cổ đại. Ưu điểm lớn nhất của nó là tính thực tiễn và khả năng giải phóng con người khỏi khổ đau tinh thần thông qua sự tự chủ nội tại.
Tuy nhiên, học thuyết này cũng vấp phải những giới hạn lịch sử và các chỉ trích:
\begin{itemize}
    \item \textbf{Tính cứng nhắc (Rigidity)}: Quan điểm "tất cả hoặc không gì cả" (hiền nhân thì hoàn hảo, người thường thì điên rồ) bị coi là thiếu thực tế.
    \item \textbf{Vấn đề cảm xúc}: Việc coi đau buồn khi mất người thân là "phi lý trí" đôi khi bị xem là thiếu tính nhân bản.
    \item \textbf{Tính thụ động chính trị}: Thuyết định mệnh đôi khi bị lợi dụng để biện minh cho sự tuân phục cường quyền (dù các Stoic như Cato hay Marcus Aurelius là những người hành động mạnh mẽ).
\end{itemize}

\subsection{Di sản và Ảnh hưởng hiện đại}
Bất chấp những hạn chế, di sản của chủ nghĩa khắc kỉ vẫn sống động một cách đáng kinh ngạc.

\subsubsection{Ảnh hưởng lên Kito giáo}
Khi Kito giáo hình thành, nó đã hấp thụ nhiều thuật ngữ và tư tưởng đạo đức của Stoicism. Khái niệm "Logos" trong Tin Mừng John ("Ban đầu có Ngôi Lời...") có sự tương đồng với Logos của Stoic. Thánh Paul khi thuyết giáo ở Athens cũng đã trích dẫn các nhà thơ Khắc kỉ. Các đức tính như khắc khổ, chấp nhận ý Chúa cũng rất gần gũi với việc chấp nhận Định mệnh.

\subsubsection{Cơ sở cho Tâm lý học hiện đại}
Ảnh hưởng rõ rệt nhất của chủ nghĩa khắc kỉ là trong lĩnh vực tâm lý trị liệu. Albert Ellis (sáng lập REBT) và Aaron T. Beck (sáng lập CBT - Liệu pháp Nhận thức Hành vi) đều thừa nhận sự kế thừa trực tiếp từ tư tưởng của Epictetus. Phương pháp CBT điều trị trầm cảm và lo âu dựa trên nguyên tắc cốt lõi của Stoa: thay đổi niềm tin/nhận thức sai lệch để thay đổi cảm xúc.

\subsubsection{Sự phục hưng của Stoicism hiện đại}
Ngày nay, chủ nghĩa khắc kỉ đang trở lại mạnh mẽ như một "hệ điều hành cho tâm trí" trong thế giới hiện đại đầy biến động. Nó cung cấp cho chúng ta một bộ công cụ để:
\begin{itemize}
    \item Phân biệt những gì ta có thể và không thể thay đổi (Lời nguyện thanh thản).
    \item Giữ vững sự bình tĩnh trước áp lực công việc và khủng hoảng.
    \item Xây dựng một cuộc sống ý nghĩa dựa trên giá trị tự thân thay vì chạy theo những hào nhoáng bên ngoài.
\end{itemize}

\section{Tài liệu tham khảo}
\begin{enumerate}
    \item Long, A. A., \& Sedley, D. N. (1987). \textit{The Hellenistic Philosophers} (Vol. 1). Cambridge University Press.
    \item Sellars, J. (2006). \textit{Stoicism}. University of California Press.
    \item Inwood, B. (Ed.). (2003). \textit{The Cambridge Companion to the Stoics}. Cambridge University Press.
    \item Aurelius, M. (2006). \textit{Suy tưởng (Meditations)}. (Nguyễn Duy Nhiên dịch). NXB Thế Giới.
    \item Epictetus. (2019). \textit{Giáo khoa thư (Discourses)}. NXB Tri Thức.
    \item Seneca. (2018). \textit{Những bức thư đạo đức (Moral Letters)}. NXB Tri Thức.
    \item Robertson, D. (2010). \textit{The Philosophy of Cognitive-Behavioural Therapy (CBT): Stoic Philosophy as Rational and Cognitive Psychotherapy}. Karnac Books.
    \item Irvine, W. B. (2008). \textit{A Guide to the Good Life: The Ancient Art of Stoic Joy}. Oxford University Press.
    \item Holiday, R., \& Hanselman, S. (2016). \textit{The Daily Stoic}. Portfolio.
\end{enumerate}
