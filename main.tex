\documentclass[14pt,a4paper]{extarticle}

% Packages
\usepackage{geometry}
\geometry{
 a4paper,
 left=2.5cm,
 right=2.5cm,
 top=2cm,
 bottom=2cm,
}
\usepackage{fontspec}
\usepackage[vietnamese]{babel}
\usepackage{graphicx}
\usepackage{float}
\usepackage{tabularx}
\usepackage{hyperref}
\usepackage{listings}
\usepackage{titlesec}
\usepackage{setspace}
\usepackage{indentfirst}
\usepackage{enumitem}
\usepackage{fancyhdr}
\usepackage{tocloft}
\usepackage{array}
\usepackage{xcolor}
\usepackage{tikz}
\usepackage{unicode-math}

% Define colors
\definecolor{bkblue}{RGB}{0,102,153}

% Formatting - Use TeX Gyre Termes (Linux-compatible Times New Roman clone)
\setmainfont{TeX Gyre Termes}
\setsansfont{TeX Gyre Heros}
\setmonofont{TeX Gyre Cursor}

% Set font size to 13pt
\renewcommand{\normalsize}{\fontsize{13pt}{15.6pt}\selectfont}
\normalsize

% Paragraph formatting per regulations
\setlength{\parindent}{1cm}
\setlength{\parskip}{6pt}
\linespread{1.3}

% Section formatting
\titleformat{\section}
  {\centering\normalfont\fontsize{14}{17}\bfseries\MakeUppercase}
  {CHƯƠNG \thesection.\ }{0em}{}
\titlespacing*{\section}{0pt}{12pt}{6pt}

\titleformat{\subsection}
  {\normalfont\fontsize{13}{16}\bfseries}
  {\thesubsection.}{0.5em}{}

\titleformat{\subsubsection}
  {\normalfont\fontsize{13}{16}\itshape}
  {\thesubsubsection.}{0.5em}{}

% Page numbering
\pagestyle{fancy}
\fancyhf{}
\fancyfoot[C]{\thepage}
\renewcommand{\headrulewidth}{0pt}

\begin{document}

% ============ TRANG BÌA (THEO MẪU CHÍNH THỨC) ============
\begin{titlepage}
    \newgeometry{left=1.5cm, right=1.5cm, top=1cm, bottom=1cm}
    
    % Khung viền xanh kép
    \begin{tikzpicture}[remember picture, overlay]
        \draw[bkblue, line width=2pt] 
            ([shift={(0.8cm,-0.8cm)}]current page.north west) 
            rectangle 
            ([shift={(-0.8cm,0.8cm)}]current page.south east);
        \draw[bkblue, line width=0.5pt] 
            ([shift={(1cm,-1cm)}]current page.north west) 
            rectangle 
            ([shift={(-1cm,1cm)}]current page.south east);
    \end{tikzpicture}
    
    \centering
    
    \vspace*{0.5cm}
    
    % Tên trường
    {\fontsize{14}{17}\selectfont \textbf{TRƯỜNG ĐẠI HỌC BÁCH KHOA}}
    
    \vspace{0.2cm}
    
    {\fontsize{14}{17}\selectfont \textbf{ĐẠI HỌC QUỐC GIA – THÀNH PHỐ HỒ CHÍ MINH}}
    
    \vspace{0.3cm}
    
    % Decorative symbol
    {\color{bkblue} \textbf{--- $\diamond$ ---}}
    
    \vspace{0.8cm}
    
    % Logo BK (sử dụng file thật)
    \includegraphics[height=2.5cm]{images/logo_bk.png}
    
    \vspace{1cm}
    
    % Tiêu đề bài tiểu luận
    {\fontsize{14}{17}\selectfont \textbf{BÀI TIỂU LUẬN TRIẾT HỌC (MM: AS 5113)}}
    
    \vspace{0.3cm}
    
    {\fontsize{14}{17}\selectfont \color{bkblue} \textbf{ĐỀ TÀI:}}
    
    \vspace{0.2cm}
    
    {\fontsize{14}{17}\selectfont \textbf{LỊCH SỬ TRIẾT HỌC PHƯƠNG TÂY: CHỦ NGHĨA KHẮC KỈ}}
    
    {\fontsize{14}{17}\selectfont \textbf{VÀ VAI TRÒ CỦA NÓ ĐỐI VỚI ĐỜI SỐNG XÃ HỘI}}
    
    \vspace{1.2cm}
    
    % Thông tin lớp, nhóm
    {\fontsize{13}{16}\selectfont \textbf{LỚP} \underline{\hspace{1cm}CH03\hspace{1cm}} \textbf{NHÓM} \underline{\hspace{1cm}7\hspace{1cm}} \textbf{HK 252}}
    
    \vspace{0.3cm}
    
    {\fontsize{13}{16}\selectfont \textbf{NGÀY NỘP} \underline{\hspace{4cm}}}
    
    \vspace{0.3cm}
    
    {\fontsize{13}{16}\selectfont \textbf{Giảng viên hướng dẫn:} \underline{\hspace{0.3cm}TS. NGUYỄN THỊ MINH HƯƠNG\hspace{0.3cm}}}
    
    \vspace{0.8cm}
    
    % Bảng thành viên nhóm (6 cột theo mẫu)
    \begin{center}
    \renewcommand{\arraystretch}{1.2}
    \begin{tabular}{|c|l|c|c|c|c|}
        \hline
        \textbf{Stt} & \textbf{Học viên thực hiện} & \textbf{ID} & \textbf{Xếp loại} & \textbf{Nhiệm vụ} & \textbf{Điểm} \\
        & & & \textbf{(... \%)} & & \textbf{số} \\
        \hline
        1 & Nguyễn Thế Dân & 2470158 & & & \\
        \hline
        2 & Nguyễn Tấn Phát & 2570289 & & & \\
        \hline
        3 & Huỳnh Thị Như Quỳnh & 2570311 & & & \\
        \hline
        4 & Phan Đức Minh & 2570254 & & & \\
        \hline
        5 & Nguyễn Chí Thành & 2570324 & & & \\
        \hline
        6 & Phạm Huy Thanh & 2570317 & & & \\
        \hline
    \end{tabular}
    \end{center}
    
    \vfill
    
    % Footer
    {\fontsize{13}{16}\selectfont \textit{Thành phố Hồ Chí Minh – 2026}}
    
    \vspace{0.5cm}
    
    \restoregeometry
\end{titlepage}

% ============ MẪU 02: BÁO CÁO KẾT QUẢ LÀM VIỆC NHÓM ============
\newpage
\thispagestyle{empty}

\begin{flushleft}
{\fontsize{11}{13}\selectfont \textbf{TRƯỜNG ĐẠI HỌC BÁCH KHOA}} \hfill {\fontsize{11}{13}\selectfont \textbf{Mẫu 02}}\\
{\fontsize{10}{12}\selectfont KHOA KHOA HỌC ỨNG DỤNG}\\
{\fontsize{10}{12}\selectfont \textbf{BỘ MÔN LÝ LUẬN CHÍNH TRỊ}}
\end{flushleft}

\vspace{0.5cm}

\begin{center}
{\fontsize{13}{16}\selectfont \textbf{BÁO CÁO KẾT QUẢ LÀM VIỆC NHÓM VÀ BẢNG ĐIỂM BÀI TIỂU LUẬN}}\\
\vspace{0.2cm}
{\fontsize{11}{13}\selectfont Học phần: \textbf{TRIẾT HỌC (SDH)} \textit{(MSMH: AS 5113)}}
\end{center}

\vspace{0.3cm}

{\fontsize{11}{13}\selectfont 
Lớp: \underline{\hspace{0.5cm}CH03\hspace{0.5cm}} \hspace{0.3cm}
Nhóm: \underline{\hspace{0.5cm}7\hspace{0.5cm}} \hspace{0.3cm}
HK: \underline{\hspace{0.3cm}252\hspace{0.3cm}} \hspace{0.3cm}
Năm học: \underline{\hspace{0.3cm}2025-2026\hspace{0.3cm}}
}

\vspace{0.3cm}

{\fontsize{11}{13}\selectfont 
Đề tài: Lịch sử triết học phương Tây: Chủ nghĩa khắc kỉ và vai trò của nó đối với đời sống xã hội
}

\vspace{0.5cm}

\begin{center}
\renewcommand{\arraystretch}{1.3}
\fontsize{9}{11}\selectfont
\begin{tabular}{|c|c|p{1.8cm}|p{1.5cm}|p{3.5cm}|c|c|c|}
    \hline
    \textbf{STT} & \textbf{Mã số} & \textbf{Họ} & \textbf{Tên} & \textbf{Nhiệm vụ} & \textbf{Nhóm} & \textbf{Điểm} & \textbf{Ký} \\
    & \textbf{HV} & & & \textbf{phân công} & \textbf{đánh giá} & \textbf{GV} & \textbf{tên} \\
    \hline
    1 & 2470158 & Nguyễn Thế & Dân & & & & \\
    \hline
    2 & 2570289 & Nguyễn Tấn & Phát & & & & \\
    \hline
    3 & 2570311 & Huỳnh T. Như & Quỳnh & & & & \\
    \hline
    4 & 2570254 & Phan Đức & Minh & & & & \\
    \hline
    5 & 2570324 & Nguyễn Chí & Thành & & & & \\
    \hline
    6 & 2570317 & Phạm Huy & Thanh & & & & \\
    \hline
\end{tabular}
\end{center}

\vspace{0.5cm}

{\fontsize{10}{12}\selectfont 
Họ và tên nhóm trưởng: \underline{\hspace{0.3cm}Nguyễn Thế Dân\hspace{0.3cm}}, Số ĐT: \underline{\hspace{2cm}}, Email: \underline{\hspace{3cm}}
}

\vspace{0.3cm}

{\fontsize{10}{12}\selectfont 
Nhận xét của GV: \dotfill
}

\vspace{1cm}

\begin{center}
\begin{tabular}{cc}
\textbf{GIẢNG VIÊN} & \textbf{NHÓM TRƯỞNG} \\
\textit{(Ký và ghi rõ họ, tên)} & \textit{(Ký và ghi rõ họ, tên)} \\
& \\
& \\
& \\
TS. Nguyễn Thị Minh Hương & Nguyễn Thế Dân \\
\end{tabular}
\end{center}

\vspace{0.5cm}

{\fontsize{8}{10}\selectfont \textit{
HV in 01 tờ rời và nộp báo cáo cho GV khi nộp BTL; GV đánh giá, chấm điểm và nộp báo cáo về Bộ môn.\\
Nhóm điền vào tất cả các cột, trừ cột số 7 (Điểm GV chấm): Thầy/ Cô sẽ điền.\\
Cột 6: Nhóm tự đánh giá các thành viên theo loại tốt (T), khá (K), trung bình (TB), Yếu (Y) hoặc theo \% (100\%, 75\%, 50\%, 25\%, 0\%).
}}

% ============ MẪU 01: BẢNG ĐIỂM KHÁC 20% ============
\newpage
\thispagestyle{empty}

\begin{flushright}
{\fontsize{12}{14}\selectfont \textbf{Mẫu 01}}
\end{flushright}

\begin{center}
{\fontsize{13}{16}\selectfont \textbf{BẢNG ĐIỂM KHÁC (chiếm 20\% điểm học phần)}}\\
\vspace{0.2cm}
{\fontsize{12}{14}\selectfont \textbf{HỌC PHẦN: TRIẾT HỌC MM: AS 5113}}
\end{center}

\vspace{0.5cm}

{\fontsize{11}{13}\selectfont 
\textbf{Lớp:} \underline{\hspace{1cm}CH03\hspace{1cm}} \hspace{1cm} \textbf{Nhóm:} \underline{\hspace{1cm}7\hspace{1cm}} \hspace{1cm} \textbf{Học kỳ:} \underline{\hspace{1cm}252\hspace{1cm}}
}

\vspace{0.3cm}

{\fontsize{11}{13}\selectfont 
\textbf{Nhóm trưởng:} \underline{\hspace{0.3cm}Nguyễn Thế Dân\hspace{0.3cm}} \hspace{0.5cm} \textbf{Số di động:} \underline{\hspace{2cm}} \hspace{0.5cm} \textbf{Email:} \underline{\hspace{3cm}}
}

\vspace{0.5cm}

\begin{center}
\renewcommand{\arraystretch}{1.3}
\fontsize{10}{12}\selectfont
\begin{tabular}{|c|c|p{3cm}|p{1.5cm}|c|c|c|c|}
    \hline
    \textbf{Stt} & \textbf{ID} & \textbf{Họ và lót} & \textbf{Tên} & \textbf{Bài tập quá} & \textbf{Xây dựng bài,} & \textbf{Tổng} & \textbf{Ký tên} \\
    & & & & \textbf{trình trên BK\_} & \textbf{Chuyên cần} & \textbf{Khác} & \\
    & & & & \textbf{LMS} & & \textbf{20\%} & \\
    & & & & \textbf{(5,0 đ)} & \textbf{(5,0đ)} & & \\
    \textbf{(1)} & \textbf{(2)} & \textbf{(3)} & \textbf{(4)} & \textbf{(5)} & \textbf{(6)} & \textbf{(7)} & \textbf{(8)} \\
    \hline
    1 & 2470158 & Nguyễn Thế & Dân & & & & \\
    \hline
    2 & 2570289 & Nguyễn Tấn & Phát & & & & \\
    \hline
    3 & 2570311 & Huỳnh Thị Như & Quỳnh & & & & \\
    \hline
    4 & 2570254 & Phan Đức & Minh & & & & \\
    \hline
    5 & 2570324 & Nguyễn Chí & Thành & & & & \\
    \hline
    6 & 2570317 & Phạm Huy & Thanh & & & & \\
    \hline
\end{tabular}
\end{center}

\vspace{0.3cm}

{\fontsize{9}{11}\selectfont 
\textit{Ghi chú:} Bảng điểm này có 06 học viên\\
Cột 5: Làm các bài tập theo quy định của giảng viên trên trang BK\_LMS.\\
Cột 6: Đi học đầy đủ, xây dựng bài tích cực và thực hiện được các yêu cầu của buổi học. Vắng 1 buổi học trừ 1 điểm, vắng 05 buổi cấm thi.\\
Cột 7 = (5) + (6)
}

\vspace{1cm}

\begin{flushright}
{\fontsize{11}{13}\selectfont 
\textit{Ngày} \underline{\hspace{1cm}} \textit{tháng} \underline{\hspace{1cm}} \textit{năm} \underline{\hspace{1.5cm}}
}

\vspace{0.5cm}

{\fontsize{11}{13}\selectfont \textbf{Nhóm trưởng}}\\
{\fontsize{10}{12}\selectfont \textit{(Họ tên và chữ ký)}}

\vspace{1.5cm}

{\fontsize{11}{13}\selectfont Nguyễn Thế Dân}
\end{flushright}

% ============ DANH MỤC VIẾT TẮT ============
\newpage
\section*{DANH MỤC CÁC TỪ VIẾT TẮT}
\addcontentsline{toc}{section}{DANH MỤC CÁC TỪ VIẾT TẮT}

\begin{tabular}{l l}
    \textbf{CNKK} & Chủ nghĩa khắc kỉ \\
    \textbf{TCN} & Trước Công nguyên \\
    \textbf{CN} & Công nguyên \\
    \textbf{Nxb} & Nhà xuất bản \\
    \textbf{TPHCM} & Thành phố Hồ Chí Minh \\
    \textbf{CBT} & Cognitive Behavioral Therapy (Liệu pháp Nhận thức Hành vi) \\
\end{tabular}

% ============ MỤC LỤC ============
\newpage
\tableofcontents

% ============ MỞ ĐẦU ============
\newpage
\section*{MỞ ĐẦU}
\addcontentsline{toc}{section}{MỞ ĐẦU}
% ============ MỞ ĐẦU ============

\subsection*{1. Tính cấp thiết của đề tài}
\addcontentsline{toc}{subsection}{1. Tính cấp thiết của đề tài}

Trong lịch sử tư tưởng phương Tây, triết học Hy Lạp cổ đại được coi là cái nôi của nền văn minh lý tính. Trong số các trường phái triết học Hy Lạp hóa (Hellenistic philosophy), chủ nghĩa khắc kỉ nổi lên như một hệ tư tưởng có sức sống mãnh liệt nhất, kéo dài hơn 500 năm từ thế kỷ III TCN đến thế kỷ II CN và để lại dấu ấn sâu đậm trong văn hóa, tôn giáo và đời sống chính trị của Đế chế La Mã\footnote{Long, A. A. (1986), \textit{Hellenistic Philosophy: Stoics, Epicureans, Sceptics}, University of California Press, Berkeley.}.

Về bản chất, \textbf{chủ nghĩa khắc kỉ} được hiểu là một hệ thống triết học thực hành, quan niệm rằng hạnh phúc đích thực không đến từ của cải hay danh vọng bên ngoài, mà từ việc rèn luyện đức hạnh và sống thuận theo tự nhiên. Cốt lõi của nó là thái độ bình thản chấp nhận những gì không thể thay đổi và nỗ lực làm chủ những gì nằm trong tầm kiểm soát của bản thân\footnote{Inwood, B. (Ed.) (2003), \textit{The Cambridge Companion to the Stoics}, Cambridge University Press, Cambridge.}.

Chủ nghĩa khắc kỉ ra đời trong bối cảnh khủng hoảng của thế giới Hy Lạp sau cái chết của Alexander Đại đế (323 TCN). Khi các thành bang Hy Lạp (Polis) sụp đổ, con người mất đi điểm tựa cộng đồng truyền thống và rơi vào trạng thái bất an hiện sinh. Trong hoàn cảnh đó, các triết gia khắc kỉ đã xây dựng một hệ thống triết học toàn diện nhằm giúp cá nhân tìm kiếm sự bình an nội tại bất chấp hoàn cảnh bên ngoài\footnote{Sellars, J. (2006), \textit{Stoicism}, University of California Press, Berkeley.}.

Nghiên cứu chủ nghĩa khắc kỉ không chỉ có ý nghĩa khảo cổ học về tư tưởng mà còn mang tính thời đại sâu sắc. Trong thế kỷ XXI, khi con người đối mặt với những khủng hoảng hiện sinh mới như biến đổi khí hậu, đại dịch toàn cầu, và sự bất ổn xã hội, các nguyên lý của chủ nghĩa khắc kỉ về việc phân biệt những gì ta có thể kiểm soát và những gì nằm ngoài tầm kiểm soát đang được tái khám phá như một phương thức sống hiệu quả. Đặc biệt, chủ nghĩa khắc kỉ đã trở thành nền tảng lý thuyết cho Liệu pháp Nhận thức Hành vi (CBT) – một trong những phương pháp tâm lý trị liệu hiệu quả nhất hiện nay\footnote{Robertson, D. (2010), \textit{The Philosophy of Cognitive-Behavioural Therapy (CBT)}, Karnac Books, London.}.

Do đó, việc nghiên cứu \textit{"Lịch sử triết học phương Tây: Chủ nghĩa khắc kỷ, những giá trị, hạn chế và vai trò đối với đời sống xã hội"} là cần thiết về cả phương diện lý luận lẫn thực tiễn.

\subsection*{2. Mục đích và nhiệm vụ nghiên cứu}
\addcontentsline{toc}{subsection}{2. Mục đích và nhiệm vụ nghiên cứu}

\textbf{Mục đích nghiên cứu:} Làm sáng tỏ quá trình hình thành, phát triển và nội dung tư tưởng cơ bản của chủ nghĩa khắc kỉ; đồng thời đánh giá những giá trị, hạn chế và vai trò của trường phái này đối với lịch sử tư tưởng phương Tây và đời sống xã hội hiện đại.

\textbf{Nhiệm vụ nghiên cứu:}
\begin{itemize}[leftmargin=1.5cm]
    \item Phân tích bối cảnh lịch sử và các tiền đề lý luận dẫn đến sự ra đời của chủ nghĩa khắc kỉ.
    \item Hệ thống hóa quá trình phát triển của chủ nghĩa khắc kỉ qua ba giai đoạn: Sơ kỳ, Trung kỳ và Hậu kỳ.
    \item Trình bày và phân tích nội dung tư tưởng cơ bản của chủ nghĩa khắc kỉ trên ba phương diện: Bản thể luận (Vật lý học), Nhận thức luận (Logic học) và Nhân bản luận (Đạo đức học).
    \item Đánh giá những giá trị tích cực và những hạn chế lịch sử của chủ nghĩa khắc kỉ.
    \item Làm rõ vai trò và ảnh hưởng của chủ nghĩa khắc kỉ đối với triết học phương Tây hiện đại và đời sống xã hội đương đại.
\end{itemize}

\subsection*{3. Đối tượng và phạm vi nghiên cứu}
\addcontentsline{toc}{subsection}{3. Đối tượng và phạm vi nghiên cứu}

\textbf{Đối tượng nghiên cứu:} Hệ thống tư tưởng triết học của chủ nghĩa khắc kỉ, bao gồm các quan điểm về vật lý học, logic học và đạo đức học của các triết gia đại diện như Zeno thành Citium, Cleanthes, Chrysippus, Seneca, Epictetus và Marcus Aurelius.

\textbf{Phạm vi nghiên cứu:}
\begin{itemize}[leftmargin=1.5cm]
    \item Về thời gian: Tập trung vào giai đoạn từ thế kỷ III TCN (khi Zeno sáng lập trường phái) đến thế kỷ II CN (thời kỳ Marcus Aurelius). Đồng thời, mở rộng xem xét ảnh hưởng của chủ nghĩa khắc kỉ đến triết học và tâm lý học hiện đại.
    \item Về không gian: Thế giới Hy-La (Hy Lạp cổ đại và Đế chế La Mã) cùng với những liên hệ đến bối cảnh toàn cầu đương đại.
    \item Về nội dung: Tập trung phân tích ba bộ phận cấu thành chủ nghĩa khắc kỉ (Logic, Vật lý, Đạo đức) và mối quan hệ biện chứng giữa chúng.
\end{itemize}

\subsection*{4. Cơ sở lý luận và phương pháp nghiên cứu}
\addcontentsline{toc}{subsection}{4. Cơ sở lý luận và phương pháp nghiên cứu}

\textbf{Cơ sở lý luận:}
\begin{itemize}[leftmargin=1.5cm]
    \item Quan điểm duy vật biện chứng và duy vật lịch sử của triết học Mác - Lênin về sự phát triển của tư tưởng triết học trong mối liên hệ với điều kiện kinh tế - xã hội\footnote{Hội đồng Trung ương (1999), \textit{Giáo trình triết học Mác – Lênin}, Nxb. Chính trị quốc gia, Hà Nội.}.
    \item Các tác phẩm gốc của triết gia khắc kỉ: \textit{Suy tưởng} (Marcus Aurelius), \textit{Giáo khoa thư} và \textit{Cẩm nang} (Epictetus), \textit{Những bức thư đạo đức} (Seneca).
    \item Các công trình nghiên cứu chuyên sâu về lịch sử triết học Hy Lạp và La Mã của các học giả trong và ngoài nước.
\end{itemize}

\textbf{Phương pháp nghiên cứu:}
\begin{itemize}[leftmargin=1.5cm]
    \item \textit{Phương pháp lịch sử - logic:} Xem xét sự hình thành và phát triển của chủ nghĩa khắc kỉ trong tiến trình lịch sử, đồng thời rút ra những nội dung lý luận cốt lõi.
    \item \textit{Phương pháp phân tích - tổng hợp:} Phân tích các khái niệm, phạm trù triết học cụ thể và tổng hợp thành hệ thống tư tưởng nhất quán.
    \item \textit{Phương pháp so sánh:} Đối chiếu chủ nghĩa khắc kỉ với các trường phái triết học cùng thời (Epicurus, Hoài nghi) và các trào lưu tư tưởng hiện đại (Chủ nghĩa hiện sinh, CBT).
    \item \textit{Phương pháp liên ngành:} Kết hợp triết học với lịch sử, tâm lý học và xã hội học để có cái nhìn toàn diện.
\end{itemize}

\subsection*{5. Ý nghĩa lý luận và thực tiễn}
\addcontentsline{toc}{subsection}{5. Ý nghĩa lý luận và thực tiễn}

\textbf{Ý nghĩa lý luận:}
\begin{itemize}[leftmargin=1.5cm]
    \item Góp phần làm phong phú thêm hiểu biết về lịch sử triết học phương Tây, đặc biệt là giai đoạn triết học Hy Lạp hóa – một giai đoạn ít được nghiên cứu sâu tại Việt Nam so với triết học cổ điển (Socrates, Plato, Aristotle)\footnote{Nguyễn Hữu Vui (1998), \textit{Lịch sử triết học}, Nxb. Chính trị quốc gia, Hà Nội.}.
    \item Cung cấp một góc nhìn mới về mối quan hệ giữa triết học cổ đại và tư tưởng hiện đại, chứng minh sức sống vượt thời gian của di sản triết học.
    \item Làm rõ những đóng góp của chủ nghĩa khắc kỉ vào sự phát triển của logic học, đạo đức học và nhận thức luận trong lịch sử tư tưởng nhân loại.
\end{itemize}

\textbf{Ý nghĩa thực tiễn:}
\begin{itemize}[leftmargin=1.5cm]
    \item Cung cấp cơ sở triết học cho việc rèn luyện bản lĩnh, sự kiên định và khả năng đối mặt với nghịch cảnh trong cuộc sống hiện đại\footnote{Irvine, W. B. (2008), \textit{A Guide to the Good Life: The Ancient Art of Stoic Joy}, Oxford University Press, New York.}.
    \item Giúp người đọc nhận thức rõ hơn về mối quan hệ biện chứng giữa thế giới nội tâm và thế giới bên ngoài, từ đó xây dựng lối sống cân bằng và ý nghĩa hơn.
    \item Có thể ứng dụng các nguyên lý khắc kỉ trong lĩnh vực giáo dục, tâm lý trị liệu và phát triển bản thân.
\end{itemize}

\subsection*{6. Kết cấu của tiểu luận}
\addcontentsline{toc}{subsection}{6. Kết cấu của tiểu luận}

Ngoài phần Mở đầu, Kết luận và Danh mục tài liệu tham khảo, nội dung chính của tiểu luận được trình bày trong 3 chương:

\textbf{Chương 1: Những cơ sở, tiền đề và quá trình hình thành, phát triển của chủ nghĩa khắc kỉ.}

Chương này trình bày bối cảnh lịch sử - xã hội của thế giới Hy Lạp thời kỳ Hy Lạp hóa, các tiền đề tư tưởng từ triết học Hy Lạp cổ điển (Socrates, phái Khuyển nho, Heraclitus), và quá trình hình thành, phát triển của chủ nghĩa khắc kỉ qua ba giai đoạn: Sơ kỳ (Zeno, Cleanthes, Chrysippus), Trung kỳ (Panaetius, Posidonius) và Hậu kỳ (Seneca, Epictetus, Marcus Aurelius).

\textbf{Chương 2: Nội dung tư tưởng cơ bản của chủ nghĩa khắc kỉ.}

Chương này đi sâu phân tích ba bộ phận cấu thành hệ thống triết học khắc kỉ: Bản thể luận (khái niệm Logos – lý tính vũ trụ, Pneuma – khí/hơi thở, thuyết định mệnh và tương thích); Nhận thức luận (lý thuyết về phantasia – ấn tượng, katalepsis – nắm bắt, và synkatathesis – sự phê chuẩn); Nhân bản luận và Đạo đức học (khái niệm oikeiosis – sự chiếm hữu, phân loại các điều dửng dưng, tư tưởng về đức hạnh và sự tự do nội tại).

\textbf{Chương 3: Những giá trị, hạn chế và vai trò của chủ nghĩa khắc kỉ đối với đời sống xã hội.}

Chương này đánh giá những giá trị tích cực và hạn chế lịch sử của chủ nghĩa khắc kỉ; phân tích vai trò của nó đối với triết học phương Tây hiện đại (ảnh hưởng đến Kitô giáo sơ khai, triết học hiện sinh, CBT); và liên hệ với thực tiễn đời sống xã hội đương đại, đặc biệt trong bối cảnh Việt Nam.


% ============ NỘI DUNG ============
\newpage
% ============ CHƯƠNG 1 ============
\section{NHỮNG CƠ SỞ, TIỀN ĐỀ VÀ QUÁ TRÌNH HÌNH THÀNH, PHÁT TRIỂN CỦA CHỦ NGHĨA KHẮC KỈ}

\subsection{Những cơ sở hình thành và phát triển Chủ nghĩa khắc kỉ}

\subsubsection{Bối cảnh lịch sử của thế giới Hy Lạp thời kỳ Hy Lạp hóa}

Chủ nghĩa khắc kỉ ra đời trong một giai đoạn chuyển biến lớn của lịch sử phương Tây – thời kỳ Hy Lạp hóa (Hellenistic period), kéo dài từ cái chết của Alexander Đại đế năm 323 TCN đến khi Đế chế La Mã sáp nhập Ai Cập năm 30 TCN. Đây là thời kỳ mà thế giới Hy Lạp trải qua những biến động chính trị, xã hội và văn hóa sâu sắc.

Trước đó, trong thời kỳ cổ điển, đời sống của người Hy Lạp gắn bó chặt chẽ với thành bang (Polis). Công dân Hy Lạp tìm thấy ý nghĩa cuộc sống trong việc tham gia vào các hoạt động chính trị, tôn giáo và văn hóa của cộng đồng. Triết học của Plato và Aristotle đều lấy Polis làm trung tâm, coi việc xây dựng một nhà nước lý tưởng là mục tiêu tối thượng của triết học chính trị.

Tuy nhiên, các cuộc chinh phạt của Alexander Đại đế đã phá vỡ cấu trúc thành bang truyền thống. Các vương quốc Hy Lạp hóa rộng lớn (Ptolemy ở Ai Cập, Seleucid ở Syria, Antigonid ở Macedonia) thay thế các thành bang nhỏ bé. Trong các đế chế đa sắc tộc này, cá nhân trở nên nhỏ bé và bất lực trước bộ máy quyền lực khổng lồ. Họ không còn có tiếng nói trong các quyết định chính trị như trước.

Sự sụp đổ của Polis kéo theo một cuộc khủng hoảng hiện sinh sâu sắc. Khi cá nhân không còn tìm thấy ý nghĩa cuộc sống trong đời sống công cộng, họ buộc phải quay vào bên trong, tìm kiếm sự bình an trong chính tâm hồn mình. Đây chính là mảnh đất màu mỡ cho sự nảy nở của các trường phái triết học đạo đức mới như Chủ nghĩa Khoái lạc (Epicureanism), Chủ nghĩa Hoài nghi (Skepticism) và Chủ nghĩa Khắc kỉ (Stoicism). Các trường phái này có chung mục tiêu: giúp cá nhân đạt được trạng thái bình an tâm hồn (Ataraxia) bất chấp sự hỗn loạn của thế giới bên ngoài.

\subsubsection{Bối cảnh kinh tế - xã hội}

Thời kỳ Hy Lạp hóa chứng kiến sự phát triển mạnh mẽ của thương mại và giao lưu văn hóa. Các tuyến đường buôn bán kết nối Hy Lạp với Ấn Độ, Trung Á và Ai Cập. Các thành phố quốc tế lớn như Alexandria, Antioch và Pergamon trở thành trung tâm văn hóa đa dạng, nơi các tư tưởng từ phương Đông và phương Tây giao thoa.

Sự phân hóa giàu nghèo ngày càng sâu sắc. Tầng lớp thương nhân giàu có nổi lên, trong khi nhiều nông dân mất đất và trở thành lao động làm thuê. Nô lệ chiếm một tỷ lệ lớn trong dân số. Trong bối cảnh này, triết học khắc kỉ với quan điểm về sự bình đẳng căn bản của mọi người (dựa trên việc tất cả đều chia sẻ Logos) đã thu hút được nhiều tầng lớp xã hội khác nhau, từ nô lệ (Epictetus) đến hoàng đế (Marcus Aurelius).

\subsection{Những tiền đề hình thành và phát triển Chủ nghĩa khắc kỉ}

\subsubsection{Tiền đề lý luận từ triết học Hy Lạp cổ điển}

Chủ nghĩa khắc kỉ không phải là một sáng tạo hoàn toàn mới mà là sự kế thừa và phát triển sáng tạo các di sản tư tưởng của triết học Hy Lạp cổ điển.

\textbf{Ảnh hưởng của Socrates:} Socrates (470-399 TCN) được coi là "ông tổ" tinh thần của chủ nghĩa khắc kỉ. Các triết gia khắc kỉ thường xuyên trích dẫn Socrates như một hình mẫu của hiền nhân (Sage). Những đóng góp của Socrates ảnh hưởng đến Stoicism bao gồm:
\begin{itemize}[leftmargin=1.5cm]
    \item Quan niệm rằng đức hạnh là tri thức, tội lỗi là sự ngu dốt (intellectualism đạo đức).
    \item Khẳng định rằng linh hồn quan trọng hơn thể xác, đời sống nội tâm quan trọng hơn của cải bên ngoài.
    \item Thái độ bình thản trước cái chết (thể hiện qua cái chết của Socrates khi uống thuốc độc).
    \item Phương pháp đối thoại (dialectic) để tìm kiếm chân lý.
\end{itemize}

\textbf{Ảnh hưởng của phái Khuyển nho (Cynicism):} Trường phái Khuyển nho do Antisthenes (học trò của Socrates) sáng lập và được Diogenes thành Sinope đưa đến cực đoan. Zeno, người sáng lập chủ nghĩa khắc kỉ, đã từng theo học Crates – một triết gia Khuyển nho. Những ảnh hưởng từ Cynicism bao gồm:
\begin{itemize}[leftmargin=1.5cm]
    \item Lý tưởng sống thuận theo tự nhiên (kata phusin).
    \item Coi thường của cải vật chất, danh vọng và khoái lạc.
    \item Nhấn mạnh sự tự túc (autarkeia) của cá nhân.
\end{itemize}
Tuy nhiên, Zeno đã làm mềm hóa tính cực đoan và phản xã hội của Cynicism, xây dựng một hệ thống triết học có tính hệ thống hơn và chấp nhận được hơn với xã hội thượng lưu.

\textbf{Ảnh hưởng của Heraclitus:} Heraclitus (khoảng 535-475 TCN) đã đề xuất khái niệm Logos như một nguyên lý lý tính chi phối vũ trụ. Ông cũng cho rằng lửa là nguyên tố căn bản của vũ trụ. Các triết gia khắc kỉ đã kế thừa:
\begin{itemize}[leftmargin=1.5cm]
    \item Khái niệm Logos như lý tính vũ trụ.
    \item Quan niệm vũ trụ vận hành theo quy luật tất yếu.
    \item Tư tưởng về sự thống nhất của các mặt đối lập.
    \item Hình ảnh ngọn lửa như nguyên lý chủ động của vũ trụ.
\end{itemize}

\subsubsection{Tiền đề lý luận từ triết học phương Đông}

Mặc dù vấn đề này còn đang được tranh luận, nhiều học giả cho rằng chủ nghĩa khắc kỉ có thể đã chịu ảnh hưởng gián tiếp từ các tư tưởng phương Đông thông qua các tuyến đường thương mại và giao lưu văn hóa thời Hy Lạp hóa.

Một số điểm tương đồng đáng chú ý giữa Stoicism và các triết học phương Đông:
\begin{itemize}[leftmargin=1.5cm]
    \item Với Đạo gia: Khái niệm Logos có nhiều điểm tương đồng với khái niệm Đạo (Tao) của Lão Tử – một nguyên lý vũ trụ vô hình nhưng chi phối mọi sự vật.
    \item Với Phật giáo: Quan niệm về sự vô thường của vạn vật, sự cần thiết phải buông bỏ các chấp trước để đạt được sự an lạc.
    \item Với Nho giáo: Nhấn mạnh nghĩa vụ xã hội và tu dưỡng đạo đức cá nhân.
\end{itemize}

\subsection{Quá trình hình thành và phát triển của Chủ nghĩa khắc kỉ}

\subsubsection{Thời kỳ Sơ kỳ (Early Stoa): Thế kỷ III-II TCN}

\textbf{Zeno thành Citium (334-262 TCN) – Người sáng lập:}

Zeno sinh ra tại Citium, một thành phố trên đảo Cyprus có nguồn gốc Phoenicia. Ông ban đầu là một thương gia giàu có. Bước ngoặt cuộc đời ông diễn ra khi tàu buôn của ông bị đắm và ông mất toàn bộ tài sản. Đến Athens, ông tình cờ đọc được cuốn \textit{Memorabilia} của Xenophon về Socrates tại một hiệu sách. Ông hỏi người chủ: "Tôi có thể tìm những người như thế này ở đâu?" Người chủ chỉ về phía Crates thành Thebes đang đi ngang qua. Từ đó, Zeno bắt đầu con đường triết học.

Zeno theo học Crates (phái Khuyển nho) nhưng cảm thấy sự phô trương và thiếu tế nhị của phái này không phù hợp với mình. Ông tiếp tục nghiên cứu với các triết gia thuộc Học viện (Academy) và trường phái Megarian. Khoảng năm 300 TCN, ông bắt đầu giảng dạy tại Stoa Poikile (Hành lang sơn) ở Athens – từ đó có tên gọi "Stoicism".

Zeno được mô tả là người có lối sống khắc khổ, làn da ngăm đen và tính cách nghiêm nghị. Ông là người đầu tiên phân chia triết học thành ba phần: Logic, Vật lý và Đạo đức. Ông định nghĩa mục tiêu cuối cùng của cuộc sống là "sống thuận theo tự nhiên" (homologoumenos te phusei zen).

\textbf{Cleanthes thành Assos (331-232 TCN) – Người kế nhiệm:}

Cleanthes là một võ sĩ quyền anh trước khi đến với triết học. Ông nổi tiếng với sự cần cù và lòng sùng kính. Ban ngày ông đi gánh nước thuê để kiếm sống, ban đêm ông học triết. Đóng góp lớn nhất của Cleanthes là bài thơ \textit{Thánh ca dâng Zeus} (Hymn to Zeus), một kiệt tác văn học - triết học, khẳng định niềm tin tuyệt đối vào sự sắp đặt thiêng liêng của vũ trụ và kêu gọi con người hòa hợp với ý chí của Zeus (tức Logos).

\textbf{Chrysippus thành Soli (280-207 TCN) – Người hệ thống hóa:}

Chrysippus được mệnh danh là "Bộ não của Stoa" hay "Người sáng lập thứ hai của chủ nghĩa khắc kỉ". Ông là một thiên tài logic học và là tác gia đa sản nhất trong số các triết gia cổ đại, viết hơn 700 cuốn sách (hầu hết đã thất lạc). Chrysippus đã:
\begin{itemize}[leftmargin=1.5cm]
    \item Bảo vệ học thuyết Stoa khỏi sự tấn công dữ dội từ Arcesilaus và Viện Hàn lâm Hoài nghi.
    \item Phát triển logic mệnh đề (propositional logic) đến mức hoàn thiện, tạo nên xương sống lý luận cho trường phái.
    \item Củng cố vững chắc thuyết định mệnh bằng các lập luận về nhân quả.
    \item Giải quyết nhiều nghịch lý logic như nghịch lý Người nói dối.
\end{itemize}
Có thể nói, phần lớn hệ thống triết học khắc kỉ như chúng ta biết ngày nay là do công lao của Chrysippus.

\subsubsection{Thời kỳ Trung kỳ (Middle Stoa): Thế kỷ II-I TCN}

Giai đoạn này đánh dấu sự chuyển giao của chủ nghĩa khắc kỉ từ Hy Lạp sang La Mã và sự điều chỉnh học thuyết để phù hợp với tư duy thực tiễn của người La Mã.

\textbf{Panaetius (185-110 TCN):} Ông là người bạn thân thiết của Scipio Aemilianus, vị tướng La Mã lừng danh đã tiêu diệt Carthage. Panaetius đã:
\begin{itemize}[leftmargin=1.5cm]
    \item Làm mềm hóa tính khắc khổ cực đoan của Sơ kỳ.
    \item Đưa vào các yếu tố của Plato và Aristotle.
    \item Tập trung nhiều hơn vào nghĩa vụ thực tiễn (officium) của con người trong xã hội.
    \item Tác phẩm của ông ảnh hưởng trực tiếp đến cuốn \textit{De Officiis} của Cicero.
\end{itemize}

\textbf{Posidonius (135-51 TCN):} Ông là một nhà bác học đa tài, nghiên cứu cả thiên văn, địa lý và lịch sử. Posidonius am hiểu tâm lý học con người sâu sắc hơn các bậc tiền bối, thừa nhận vai trò của các yếu tố phi lý trí trong tâm hồn và tìm cách dung hòa chúng với học thuyết khắc kỉ.

\subsubsection{Thời kỳ Hậu kỳ (Late Stoa / Roman Stoa): Thế kỷ I-II CN}

Đây là thời kỳ rực rỡ nhất của chủ nghĩa khắc kỉ về mặt văn học và ảnh hưởng xã hội. Các triết gia thời kỳ này tập trung chủ yếu vào đạo đức học thực hành, coi triết học là một phương thức sống hơn là một hệ thống lý thuyết trừu tượng.

\textbf{Seneca (4 TCN – 65 CN) – Chính khách và Nhà văn:}

Lucius Annaeus Seneca là một trong những nhân vật quyền lực và giàu có nhất đế chế La Mã. Ông là gia sư và sau là cố vấn cho hoàng đế Nero. Cuộc đời ông đầy rẫy mâu thuẫn giữa lý tưởng triết học về sự giản đơn và thực tế cuộc sống xa hoa chốn cung đình. Tuy nhiên, chính trong hoàn cảnh đó, ông viết nên những tác phẩm triết học sâu sắc nhất về sự vô thường của của cải, cách quản lý cơn giận (De Ira), giá trị của thời gian (De Brevitate Vitae), và nghệ thuật đối diện với cái chết.

Trong các tác phẩm như \textit{Những bức thư đạo đức gửi Lucilius}, Seneca bàn về mọi khía cạnh của đời sống: tình bạn, thái độ với nô lệ, tuổi già, và sự lưu vong. Ông dạy rằng: "Không phải người có quá ít, mà người mong muốn nhiều hơn mới là người nghèo". Cuối cùng, ông bị Nero ép tự sát. Cái chết bình thản của Seneca được ví như cái chết của Socrates.

\textbf{Epictetus (55-135 CN) – Từ nô lệ đến bậc thầy:}

Trái ngược hoàn toàn với Seneca, Epictetus sinh ra là một nô lệ tại Hierapolis (Thổ Nhĩ Kỳ ngày nay). Tên "Epictetus" trong tiếng Hy Lạp có nghĩa là "người được mua về". Ông bị chủ đánh gãy một chân, trở thành người tàn tật suốt đời. Sau khi được giải phóng, ông mở trường dạy triết học tại Nicopolis.

Triết lý của Epictetus cực kỳ thực tế và "cơ bắp". Ông không viết sách; những lời dạy của ông được học trò Arrian ghi lại trong \textit{Discourses} (Giáo khoa thư) và \textit{Enchiridion} (Cẩm nang). Epictetus nhấn mạnh tuyệt đối vào sự phân biệt giữa những gì ta có thể kiểm soát (ý kiến, phán đoán, ham muốn) và những gì nằm ngoài tầm kiểm soát (thân thể, tài sản, danh tiếng). Câu nói nổi tiếng của ông: "Đau đớn là không thể tránh khỏi, nhưng đau khổ là sự lựa chọn."

\textbf{Marcus Aurelius (121-180 CN) – Vị vua triết gia:}

Marcus Aurelius là hoàng đế của Đế chế La Mã hùng mạnh, người nắm trong tay quyền sinh sát cả thế giới. Tuy nhiên, cuộc đời ông là chuỗi ngày dài của chiến tranh biên ải chống lại các bộ tộc German, dịch bệnh (dịch Antonine) và sự phản bội (cuộc nổi loạn của Cassius). Ông viết tác phẩm \textit{Suy tưởng} (Meditations) không phải để xuất bản, mà như một cuốn nhật ký rèn luyện tinh thần cho chính mình giữa chiến trường.

\textit{Suy tưởng} là minh chứng sống động nhất cho việc một người có thể giữ gìn phẩm hạnh và lòng nhân ái ngay cả khi nắm giữ quyền lực tuyệt đối. Ông thường tự nhắc nhở: "Sáng sớm thức dậy, hãy tự nhủ: hôm nay ta sẽ gặp những kẻ tọc mạch, vô ơn, kiêu ngạo, lừa lọc, đố kỵ... Nhưng ta không thể giận họ, vì ta và họ cùng chung một bản thể."

\subsection*{Tiểu kết Chương 1}
\addcontentsline{toc}{subsection}{Tiểu kết Chương 1}

Chủ nghĩa khắc kỉ ra đời trong bối cảnh khủng hoảng của thế giới Hy Lạp thời kỳ Hy Lạp hóa, khi cấu trúc thành bang truyền thống sụp đổ và con người buộc phải tìm kiếm ý nghĩa cuộc sống trong chính nội tâm mình. Trường phái này kế thừa sáng tạo các tư tưởng của Socrates (đạo đức học), phái Khuyển nho (lý tưởng sống thuận tự nhiên) và Heraclitus (khái niệm Logos).

Qua hơn 500 năm phát triển, chủ nghĩa khắc kỉ đã trải qua ba giai đoạn: Sơ kỳ đặt nền móng lý thuyết (Zeno, Chrysippus), Trung kỳ thích ứng với thế giới La Mã (Panaetius, Posidonius), và Hậu kỳ tập trung vào thực hành đạo đức (Seneca, Epictetus, Marcus Aurelius). Điều đáng chú ý là các triết gia khắc kỉ đến từ mọi tầng lớp xã hội – từ nô lệ (Epictetus) đến hoàng đế (Marcus Aurelius) – chứng minh tính phổ quát của triết học này.


\newpage
% ============ CHƯƠNG 2 ============
\section{NỘI DUNG TƯ TƯỞNG CƠ BẢN CỦA CHỦ NGHĨA KHẮC KỈ}

Hệ thống triết học khắc kỉ được các triết gia sáng lập ví như một khu vườn: Vật lý học là đất đai và cây cối, Logic học là hàng rào bảo vệ, và Đạo đức học là hoa trái – mục đích cuối cùng của toàn bộ hệ thống. Trong chương này, chúng ta sẽ đi sâu phân tích ba bộ phận này theo thứ tự: Bản thể luận (Vật lý học), Nhận thức luận (Logic học) và Nhân bản luận (Đạo đức học).

\subsection{Bản thể luận của Chủ nghĩa khắc kỉ}

\subsubsection{Quan niệm về Logos trong lịch sử triết học}

Khái niệm Logos (λόγος) là một trong những khái niệm phong phú nhất trong triết học Hy Lạp, với nhiều tầng nghĩa: lời nói, lý lẽ, lý tính, quy luật, tỷ lệ. Heraclitus (khoảng 535-475 TCN) là người đầu tiên sử dụng Logos như một thuật ngữ triết học để chỉ nguyên lý lý tính chi phối vũ trụ. Ông viết: "Logos này tồn tại mãi mãi, nhưng con người không hiểu được nó, dù trước khi nghe hay sau khi đã nghe."

Các triết gia khắc kỉ đã kế thừa và phát triển khái niệm Logos của Heraclitus thành trụ cột trung tâm của hệ thống bản thể luận. Đối với họ, Logos không chỉ là một nguyên lý trừu tượng mà là một thực thể vật chất, sống động, thâm nhập vào mọi ngóc ngách của vũ trụ.

\subsubsection{Thế giới như là Logos và Pneuma}

Người khắc kỉ theo chủ nghĩa duy vật (Materialism), cho rằng chỉ có vật chất mới thực sự tồn tại, vì chỉ vật chất mới có khả năng tác động và bị tác động. Tuy nhiên, vật chất của họ không phải là vật chất "chết" mà bao gồm hai nguyên lý không thể tách rời:

\textbf{Nguyên lý thụ động (Passive principle):} Đây là vật chất vô hình, vô tính, như một chất nền chờ được nhào nặn. Plato gọi nó là "chất liệu" (hyle), Stoicism gọi là "vật chất không có tính chất" (apoios ousia).

\textbf{Nguyên lý chủ động (Active principle):} Đây chính là Logos hay Thượng đế – lý tính sáng tạo và tổ chức vũ trụ. Nguyên lý chủ động này mang bản chất vật chất, được gọi là \textit{Pneuma} (Khí/Hơi thở). Pneuma là sự pha trộn của hai nguyên tố: Lửa (mang tính nóng, hoạt động) và Không khí (mang tính chuyển động, linh hoạt).

Pneuma thâm nhập vào mọi vật chất thụ động và tạo ra sự "căng" (tonos) – một lực mang tính kép hướng đồng thời vừa ra ngoài vừa vào trong. Chính sự căng này tạo nên hình dạng, tính chất và sự kết dính của các vật thể:
\begin{itemize}[leftmargin=1.5cm]
    \item Trong các vật vô cơ (đá, kim loại), Pneuma tạo nên \textit{hexis} (sự kết dính) – giữ cho vật thể có hình dạng ổn định.
    \item Trong thực vật, Pneuma tạo nên \textit{phusis} (bản tính tự nhiên) – cho phép sinh trưởng và sinh sản.
    \item Trong động vật, Pneuma tạo nên \textit{psyche} (linh hồn) – cho phép cảm giác và chuyển động.
    \item Trong con người, Pneuma tạo nên \textit{nous} (lý trí) – cho phép tư duy và đức hạnh.
\end{itemize}

Như vậy, vũ trụ trong quan niệm khắc kỉ là một cơ thể sống thống nhất, nơi tất cả các bộ phận đều liên kết với nhau thông qua Pneuma. Đây là chủ nghĩa vật hoạt luận (Hylozoism) – quan niệm rằng vật chất tự thân mang tính sống và lý tính.

\subsubsection{Thuyết định mệnh và Thuyết tương thích}

Vì Logos chi phối tất cả, mọi sự kiện trong vũ trụ đều diễn ra theo chuỗi nhân quả tất yếu. Các triết gia khắc kỉ gọi đây là "số phận" (heimarmene) hoặc "quan phòng" (pronoia). Chrysippus viết: "Không có gì xảy ra mà không có nguyên nhân, cũng không có gì vượt ra ngoài bản chất và lý tính vũ trụ."

Đây là \textbf{thuyết định mệnh cứng} (Hard Determinism): mọi thứ đã, đang và sẽ xảy ra đều được định sẵn từ trước bởi Logos. Tuy nhiên, điều này đặt ra một vấn đề hóc búa: nếu mọi thứ đều được định sẵn, thì trách nhiệm đạo đức ở đâu? Con người có thực sự tự do không?

Các triết gia khắc kỉ giải quyết vấn đề này bằng \textbf{thuyết tương thích} (Compatibilism) – cho rằng tự do và định mệnh có thể cùng tồn tại. Họ phân biệt giữa hai loại nguyên nhân:
\begin{itemize}[leftmargin=1.5cm]
    \item \textit{Nguyên nhân bên ngoài (proximate cause):} Các sự kiện thế giới tác động lên ta.
    \item \textit{Nguyên nhân bên trong (principal cause):} Phản ứng của tâm trí ta đối với các sự kiện đó.
\end{itemize}

Chrysippus dùng hình ảnh một cái ống lăn: nếu bạn đẩy một cái ống hình trụ, nó sẽ lăn. Cái đẩy là nguyên nhân bên ngoài, nhưng việc nó lăn (thay vì trượt, như khối vuông) là do bản tính bên trong của nó. Tương tự, sự kiện bên ngoài kích hoạt phản ứng của ta, nhưng \textit{cách} ta phản ứng phụ thuộc vào tính cách và phán đoán của chính ta. Tự do nằm ở chỗ ta có quyền \textit{đồng ý} hoặc \textit{từ chối} các ấn tượng theo lý trí.

Một ẩn dụ nổi tiếng khác là hình ảnh con chó bị buộc vào cỗ xe ngựa: nếu con chó chạy theo xe thì cuộc hành trình sẽ êm ả; nếu nó chống cự thì vẫn bị kéo đi nhưng đau đớn. Kết quả cuối cùng là như nhau, nhưng trải nghiệm hoàn toàn khác biệt. Sự khôn ngoan nằm ở việc \textit{tự nguyện hòa hợp} với trật tự vũ trụ thay vì chống cự vô vọng.

\subsubsection{Chu kỳ vũ trụ (Ekpyrosis)}

Một quan điểm độc đáo của chủ nghĩa khắc kỉ sơ kỳ là thuyết \textit{Ekpyrosis} (Đại hỏa tai). Theo đó, vũ trụ vận hành theo các chu kỳ vĩnh cửu gọi là "Đại năm" (Great Year). Khi một Đại năm kết thúc, toàn bộ vũ trụ sẽ bùng cháy trong ngọn lửa thanh tẩy, mọi thứ trở về nguyên dạng lửa thuần khiết (tức Logos/Zeus trong trạng thái thuần túy nhất). Sau đó, vũ trụ sẽ tái sinh từ ngọn lửa và lặp lại chính xác mọi sự kiện đã diễn ra trong chu kỳ trước – Socrates sẽ lại bị xử tử, Alexander sẽ lại chinh phục thế giới.

Quan niệm này (Thuyết hồi quy vĩnh cửu – Eternal Recurrence) sau này được Friedrich Nietzsche hồi sinh trong triết học hiện đại, mặc dù với ý nghĩa khác biệt.

\subsection{Nhận thức luận của Chủ nghĩa khắc kỉ}

\subsubsection{Phantasia – Ấn tượng giác quan}

Quá trình nhận thức theo quan niệm khắc kỉ bắt đầu từ \textit{Phantasia} (ấn tượng giác quan). Khi một đối tượng bên ngoài tác động vào giác quan, nó để lại một "dấu ấn" trong tâm trí, giống như con dấu in lên sáp ong. Dấu ấn này gọi là phantasia.

Tuy nhiên, không phải mọi phantasia đều đúng. Có những ấn tượng sai lệch do giác quan bị lừa (như nhìn cây gậy trong nước bị gãy khúc), do trạng thái tâm lý bất thường (như ảo giác trong cơn điên), hoặc do suy luận sai.

\subsubsection{Kataleptike Phantasia – Ấn tượng thấu hiểu}

Tiêu chuẩn của chân lý trong nhận thức luận khắc kỉ là \textit{Kataleptike Phantasia} (ấn tượng thấu hiểu/nắm bắt). Đây là một loại ấn tượng đặc biệt với các đặc điểm:
\begin{itemize}[leftmargin=1.5cm]
    \item Phát sinh từ một đối tượng thực sự tồn tại.
    \item Phản ánh chính xác đối tượng đó.
    \item Có tính rõ ràng, minh bạch đến mức tâm trí không thể chối từ (giống như ánh sáng mặt trời tự chứng minh sự tồn tại của mình).
\end{itemize}

Các triết gia Hoài nghi (Skeptics) tấn công dữ dội tiêu chuẩn này, cho rằng không thể phân biệt được ấn tượng thật với ảo giác. Chrysippus đáp trả bằng cách cho rằng có những trường hợp ấn tượng rõ ràng đến mức bất kỳ người lành mạnh nào cũng phải đồng ý.

\subsubsection{Sự phê chuẩn (Synkatathesis)}

Điểm then chốt trong nhận thức luận khắc kỉ là khái niệm \textit{synkatathesis} (sự phê chuẩn/đồng ý). Khi phantasia đến với tâm trí, ta có quyền tự do:
\begin{itemize}[leftmargin=1.5cm]
    \item \textbf{Đồng ý} (assent): Chấp nhận rằng ấn tượng này đúng.
    \item \textbf{Từ chối} (dissent): Bác bỏ ấn tượng là sai.
    \item \textbf{Treo lơ lửng} (suspension): Không đưa ra phán đoán, chờ thêm bằng chứng.
\end{itemize}

Hegemonikon (phần chủ đạo của linh hồn, tức lý trí) là nơi đưa ra quyết định phê chuẩn. Đây chính là "ngai vàng" của tự do con người. Sai lầm trong nhận thức (và theo đó, sai lầm đạo đức) xảy ra khi ta vội vàng đồng ý với những ấn tượng chưa được kiểm chứng kỹ lưỡng.

Người khắc kỉ sử dụng hình ảnh bàn tay để minh họa quá trình nhận thức:
\begin{enumerate}
    \item \textbf{Phantasia:} Xòe bàn tay ra – ấn tượng đến với tâm trí.
    \item \textbf{Synkatathesis:} Hơi co ngón tay lại – tâm trí xem xét và đồng ý.
    \item \textbf{Katalepsis:} Nắm chặt tay lại – sự nắm bắt chắc chắn.
    \item \textbf{Episteme:} Nắm chặt tay kia bao lấy – tri thức khoa học, chỉ có ở bậc hiền nhân.
\end{enumerate}

\subsubsection{Logic mệnh đề của Stoicism}

Chrysippus đã phát triển một hệ thống logic mệnh đề (propositional logic) phức tạp, khác biệt với logic hạn từ (term logic) của Aristotle. Logic Stoic tập trung vào các mệnh đề hoàn chỉnh và mối quan hệ giữa chúng (nếu – thì, hoặc – hoặc, và, không).

Chrysippus đã xây dựng năm dạng suy luận hợp lệ cơ bản (indemonstrables) mà từ đó có thể suy ra mọi suy luận hợp lệ khác:
\begin{enumerate}
    \item Nếu A thì B; A; vậy B (Modus ponens)
    \item Nếu A thì B; không B; vậy không A (Modus tollens)
    \item Không (A và B); A; vậy không B
    \item A hoặc B; A; vậy không B
    \item A hoặc B; không A; vậy B (Disjunctive syllogism)
\end{enumerate}

Hệ thống logic này đã ảnh hưởng sâu sắc đến sự phát triển của logic học hiện đại vào thế kỷ XIX-XX.

\subsection{Nhân bản luận và Đạo đức học của Chủ nghĩa khắc kỉ}

\subsubsection{Oikeiosis – Sự chiếm hữu/Thân thuộc}

Khái niệm nền tảng của đạo đức học khắc kỉ là \textit{Oikeiosis} (có thể dịch là "sự chiếm hữu", "sự thân thuộc", hay "sự nhận mình"). Đây là một quá trình tự nhiên, bẩm sinh, xảy ra ở mọi sinh vật.

\textbf{Oikeiosis cá nhân:} Ngay từ khi sinh ra, mọi sinh vật đều có bản năng tự bảo tồn. Đứa trẻ sơ sinh không cần được dạy vẫn biết tìm bầu sữa mẹ, tránh né nguy hiểm và đau đớn. Sinh vật coi bản thân và những gì thuộc về mình là "thân thuộc" (oikeion), coi những gì đe dọa mình là "xa lạ" (allotrion). Các triết gia khắc kỉ dùng điều này để bác bỏ quan điểm của Epicurus rằng động lực gốc của sinh vật là khoái lạc – họ cho rằng khoái lạc chỉ là hệ quả của việc đạt được sự tự bảo tồn.

\textbf{Oikeiosis lý trí:} Khi con người trưởng thành, lý trí phát triển và thay đổi nội dung của oikeiosis. Con người nhận ra rằng điều quý giá nhất của mình không phải là thể xác mà là lý trí và đức hạnh. Do đó, mục tiêu của đời sống chuyển từ bảo tồn thể xác sang hoàn thiện đức hạnh.

\textbf{Oikeiosis xã hội:} Quá trình oikeiosis cũng mở rộng ra bên ngoài, từ bản thân đến gia đình, bạn bè, cộng đồng và cuối cùng là toàn nhân loại. Hierocles (triết gia khắc kỉ thế kỷ II CN) mô tả điều này bằng hình ảnh các vòng tròn đồng tâm: vòng trong cùng là bản thân, các vòng tiếp theo là gia đình, họ hàng, láng giềng, đồng bào và cuối cùng là toàn bộ loài người. Nhiệm vụ của triết học là "kéo các vòng tròn về phía tâm" – mở rộng sự quan tâm của ta đến tất cả mọi người như thể họ là người thân.

Đây là cơ sở của tư tưởng \textbf{Chủ nghĩa thế giới (Cosmopolitanism)} – quan niệm rằng mọi người đều là công dân của một quốc gia chung: thế giới. Vì tất cả đều chia sẻ Logos, nên không có sự phân biệt căn bản giữa người Hy Lạp và "man di", giữa tự do và nô lệ, giữa nam và nữ.

\subsubsection{Đức hạnh là điều kiện cần và đủ cho hạnh phúc}

Các triết gia khắc kỉ phân chia mọi thứ thành ba loại:

\textbf{Tốt (Good):} Chỉ có đức hạnh (arete) mới thực sự tốt. Bốn đức hạnh chính (cardinal virtues) là:
\begin{itemize}[leftmargin=1.5cm]
    \item \textit{Phronesis} (Trí tuệ thực tiễn): Biết cái gì đáng chọn và cái gì đáng tránh.
    \item \textit{Dikaiosyne} (Công bằng): Đối xử với mọi người theo phẩm giá của họ.
    \item \textit{Andreia} (Can đảm): Chịu đựng những gì phải chịu đựng.
    \item \textit{Sophrosyne} (Tiết độ): Kiểm soát ham muốn và xung động.
\end{itemize}
Các đức hạnh này không tách rời mà liên kết chặt chẽ với nhau – người có một đức hạnh phải có tất cả.

\textbf{Xấu (Bad):} Các thói xấu (vices) đối lập – ngu dốt, bất công, hèn nhát, phóng túng.

\textbf{Dửng dưng (Indifferents – Adiaphora):} Tất cả những thứ còn lại – sức khỏe, bệnh tật, giàu nghèo, danh tiếng, thậm chí sự sống và cái chết – đều không tốt cũng không xấu về mặt đạo đức. Chúng không ảnh hưởng đến hạnh phúc đích thực.

Tuy nhiên, các điều dửng dưng được chia thành:
\begin{itemize}[leftmargin=1.5cm]
    \item \textit{Dửng dưng được ưu tiên (Preferred indifferents):} Sức khỏe, của cải, danh tiếng – những thứ tự nhiên ta nên chọn nếu không vi phạm đức hạnh.
    \item \textit{Dửng dưng không được ưu tiên (Dispreferred indifferents):} Bệnh tật, nghèo khổ, cái chết.
    \item \textit{Dửng dưng tuyệt đối:} Số lượng tóc trên đầu, số sỏi dưới chân.
\end{itemize}

Điều quan trọng là: một người khắc kỉ có thể chọn sức khỏe hơn bệnh tật, giàu có hơn nghèo khổ, nhưng \textit{không được đánh đổi đức hạnh để có được chúng} và phải sẵn sàng từ bỏ chúng nếu cần thiết mà không đau khổ.

\subsubsection{Cảm xúc và Đam mê (Pathe)}

Chủ nghĩa khắc kỉ phân biệt rõ ràng giữa hai loại trạng thái cảm xúc:

\textbf{Pathe (Đam mê/Dục vọng):} Đây là những chuyển động phi lý trí của tâm hồn, được định nghĩa là "sự phán đoán sai lầm" (false judgment). Có bốn loại pathe chính:
\begin{itemize}[leftmargin=1.5cm]
    \item \textit{Epithumia} (Ham muốn): Tin rằng điều gì đó tương lai là tốt trong khi thực ra nó chỉ là dửng dưng.
    \item \textit{Phobos} (Sợ hãi): Tin rằng điều gì đó tương lai là xấu.
    \item \textit{Hedone} (Khoái lạc): Tin rằng điều gì đó hiện tại là tốt trong khi không phải.
    \item \textit{Lupe} (Đau khổ): Tin rằng điều gì đó hiện tại là xấu.
\end{itemize}

Mục tiêu của người khắc kỉ là đạt được \textbf{Apatheia} – sự vắng mặt của các pathe. Từ "apatheia" không có nghĩa là "vô cảm" (apathy) như cách hiểu hiện đại, mà là sự giải thoát khỏi những cảm xúc phi lý trí, đau khổ không cần thiết.

\textbf{Eupatheia (Cảm xúc tốt):} Chỉ bậc hiền nhân mới có được những cảm xúc tích cực dựa trên phán đoán đúng đắn:
\begin{itemize}[leftmargin=1.5cm]
    \item \textit{Boulesis} (Mong muốn): Muốn điều thực sự tốt (đức hạnh).
    \item \textit{Eulabeia} (Thận trọng): Tránh điều thực sự xấu (thói xấu).
    \item \textit{Chara} (Vui mừng): Vui vì điều thực sự tốt.
\end{itemize}

Như vậy, người khắc kỉ không phải là hòn đá vô cảm, họ có cảm xúc nhưng là những cảm xúc được lý trí dẫn dắt.

\subsubsection{Quyền kiểm soát (Dichotomy of Control)}

Epictetus đã đúc kết toàn bộ đạo đức học khắc kỉ thành một nguyên tắc đơn giản nhưng sâu sắc trong câu mở đầu của \textit{Enchiridion}:

"Có những thứ phụ thuộc vào ta và có những thứ không phụ thuộc vào ta. Phụ thuộc vào ta là ý kiến, động lực, ham muốn, ghét bỏ – nói tóm lại, tất cả những gì là hành động của chính ta. Không phụ thuộc vào ta là thân thể, tài sản, danh tiếng, chức vụ – nói tóm lại, tất cả những gì không phải hành động của ta."

\textbf{Những gì phụ thuộc vào ta (eph' hemin):} Thế giới nội tâm – suy nghĩ, phán đoán, giá trị, phản ứng của ta.

\textbf{Những gì không phụ thuộc vào ta (ouk eph' hemin):} Thế giới bên ngoài – sự kiện, hành động của người khác, thậm chí cơ thể ta (vì ta có thể bị bệnh, bị thương, bị giết).

Sự khôn ngoan nằm ở việc:
\begin{enumerate}
    \item Tập trung năng lượng vào những gì ta kiểm soát được.
    \item Buông bỏ lo lắng về những gì nằm ngoài tầm kiểm soát.
    \item Chấp nhận kết quả bên ngoài với sự bình thản.
\end{enumerate}

Epictetus dạy: "Đừng mong muốn sự vật diễn ra như bạn muốn, mà hãy muốn chúng diễn ra như chúng đang diễn ra, và bạn sẽ sống tốt."

\subsection*{Tiểu kết Chương 2}
\addcontentsline{toc}{subsection}{Tiểu kết Chương 2}

Hệ thống tư tưởng của chủ nghĩa khắc kỉ thể hiện sự nhất quán chặt chẽ giữa ba bộ phận: Bản thể luận cung cấp nền tảng (vũ trụ được chi phối bởi Logos lý tính), Nhận thức luận giải thích cơ chế (ta nắm bắt chân lý thông qua phantasia và synkatathesis), và Đạo đức học đưa ra phương hướng sống (đức hạnh là điều tốt duy nhất, ta cần phân biệt những gì ta kiểm soát được).

Điểm độc đáo của chủ nghĩa khắc kỉ là sự kết hợp giữa chủ nghĩa duy vật nhất nguyên (monist materialism), thuyết định mệnh (determinism) và khẳng định tự do đạo đức (moral freedom). Họ cho rằng tự do không nằm ở việc thay đổi thế giới bên ngoài mà ở việc làm chủ thế giới bên trong – phán đoán, giá trị và phản ứng của chính ta.


\newpage
% ============ CHƯƠNG 3 ============
\section{NHỮNG GIÁ TRỊ, HẠN CHẾ VÀ VAI TRÒ CỦA CHỦ NGHĨA KHẮC KỈ ĐỐI VỚI ĐỜI SỐNG XÃ HỘI}

\subsection{Những giá trị và hạn chế của chủ nghĩa khắc kỉ}

\subsubsection{Những giá trị của chủ nghĩa khắc kỉ}

\textbf{Thứ nhất, giá trị về phương diện bản thể luận và thế giới quan:}

Chủ nghĩa khắc kỉ đã xây dựng một thế giới quan nhất quán và hợp lý trong bối cảnh lịch sử của nó. Quan niệm về một vũ trụ thống nhất, được chi phối bởi Logos (lý tính vũ trụ), đã cung cấp cho con người một cảm giác về trật tự và ý nghĩa giữa sự hỗn loạn của thế giới Hy Lạp hóa. Thuyết phiếm thần nội tại (Immanent pantheism) – quan niệm rằng Thượng đế hiện diện trong mọi sự vật – đã khắc phục được khoảng cách giữa thế giới thần linh và thế giới trần tục trong tôn giáo truyền thống.

Khái niệm Logos như một lý tính vũ trụ khách quan đã đặt nền móng cho tư duy khoa học sau này. Niềm tin rằng vũ trụ vận hành theo quy luật có thể nhận thức được đã khuyến khích con người tìm hiểu tự nhiên thay vì sợ hãi nó như sản phẩm của ý chí thất thường của thần linh\footnote{Long, A. A. (1986), \textit{Hellenistic Philosophy: Stoics, Epicureans, Sceptics}, University of California Press, Berkeley, tr. 200-220.}.

\textbf{Thứ hai, giá trị về phương diện logic học và nhận thức luận:}

Logic mệnh đề của Chrysippus là một đóng góp to lớn cho lịch sử logic học nhân loại. Trong gần 2000 năm, logic Aristotle thống trị thế giới phương Tây, nhưng logic khắc kỉ đã được tái phát hiện và đánh giá cao vào thế kỷ XIX-XX khi Gottlob Frege và Bertrand Russell phát triển logic hiện đại. Nhiều học giả cho rằng logic mệnh đề của chủ nghĩa khắc kỉ là tiền thân trực tiếp của logic ký hiệu (symbolic logic) hiện đại.

Lý thuyết về phantasia (ấn tượng) và synkatathesis (sự phê chuẩn) cũng rất quan trọng. Việc khẳng định rằng ta có quyền phê chuẩn hoặc từ chối các ấn tượng đã đặt nền móng cho quan niệm về chủ thể nhận thức có tính chủ động, không chỉ là tấm gương thụ động phản chiếu thế giới.

\textbf{Thứ ba, giá trị về phương diện đạo đức học:}

Đây là lĩnh vực mà chủ nghĩa khắc kỉ để lại di sản lớn nhất. Các giá trị đạo đức bao gồm:

\begin{itemize}[leftmargin=1.5cm]
    \item \textit{Tính phổ quát:} Quan niệm về sự bình đẳng căn bản của mọi người dựa trên việc tất cả đều chia sẻ Logos (lý tính vũ trụ) đã phá vỡ các rào cản giai cấp, dân tộc và giới tính của xã hội cổ đại. Tư tưởng \textit{"công dân thế giới"} (cosmopolitanism – chủ nghĩa thế giới) là tiền thân của các tuyên ngôn nhân quyền hiện đại.
    
    \item \textit{Tự do nội tại:} Quan niệm rằng tự do đích thực nằm ở bên trong, không phụ thuộc vào hoàn cảnh bên ngoài, đã mang lại hy vọng cho những người bị áp bức (nô lệ, tù nhân, người nghèo). Epictetus, bản thân là nô lệ, đã chứng minh rằng kẻ làm chủ có thể là nô lệ của ham muốn trong khi người nô lệ có thể là người tự do nhất.
    
    \item \textit{Trách nhiệm cá nhân:} Bằng cách đặt quyền phê chuẩn vào tay cá nhân, chủ nghĩa khắc kỉ đã nhấn mạnh rằng mỗi người phải chịu trách nhiệm về phản ứng của chính mình.
    
    \item \textit{Kiên cường (Resilience):} Các bài tập thực hành như \textit{"Premeditatio Malorum"} (Suy ngẫm trước về điều xấu), \textit{"Memento Mori"} (Hãy nhớ rằng bạn sẽ chết), hay \textit{"View from Above"} (Nhìn từ trên cao) đã cung cấp những công cụ cụ thể để rèn luyện sự kiên cường trước nghịch cảnh.
\end{itemize}

\textbf{Thứ tư, giá trị về phương diện thực hành:}

Khác với nhiều trường phái triết học trừu tượng, chủ nghĩa khắc kỉ coi triết học là một \textit{"nghệ thuật sống"} (techne peri ton bion). Các triết gia như Marcus Aurelius đã thực hành những gì họ giảng dạy – viết nhật ký để tự rèn luyện, đối xử nhân từ với kẻ thù, giữ vững phẩm hạnh giữa quyền lực tối thượng\footnote{Irvine, W. B. (2008), \textit{A Guide to the Good Life: The Ancient Art of Stoic Joy}, Oxford University Press, New York, tr. 180-210.}.

\subsubsection{Những hạn chế của chủ nghĩa khắc kỉ}

\textbf{Thứ nhất, hạn chế về phương diện bản thể luận:}

Chủ nghĩa duy vật của chủ nghĩa khắc kỉ, mặc dù tiến bộ so với nhị nguyên Plato, vẫn gặp khó khăn trong việc giải thích các hiện tượng tinh thần phức tạp. Việc quy giản mọi thứ (kể cả tâm hồn, đức hạnh, cảm xúc) thành vật chất (Pneuma) đã đơn giản hóa quá mức bản chất của thực tại.

Thuyết hồi quy vĩnh cửu (Ekpyrosis) – quan niệm rằng vũ trụ sẽ lặp lại chính xác mọi sự kiện – tuy hấp dẫn về mặt triết học nhưng thiếu cơ sở thực nghiệm và mang tính bi quan: nếu mọi thứ đã được định sẵn và sẽ lặp lại mãi mãi, thì nỗ lực của con người dường như vô nghĩa.

\textbf{Thứ hai, hạn chế về vấn đề định mệnh và tự do:}

Mặc dù các triết gia khắc kỉ cố gắng dung hòa định mệnh và tự do thông qua thuyết tương thích, nhưng lập luận của họ không hoàn toàn thuyết phục. Nếu mọi sự kiện (bao gồm cả suy nghĩ và quyết định của ta) đều được quy định bởi chuỗi nhân quả từ trước, thì \textit{"tự do"} mà họ đề cập có phải là tự do đích thực hay chỉ là ảo tưởng?

Các nhà phê bình (đặc biệt là Epicurus và Alexander of Aphrodisias) đã chỉ ra rằng thuyết định mệnh cứng của chủ nghĩa khắc kỉ làm suy yếu cơ sở của trách nhiệm đạo đức và khiến việc khen thưởng hay trừng phạt trở nên vô nghĩa.

\textbf{Thứ ba, hạn chế về quan niệm cảm xúc:}

Quan điểm cho rằng mọi pathe (đam mê, cảm xúc mãnh liệt) đều là \textit{"sai lầm"} cần loại bỏ đã bị nhiều triết gia sau này chỉ trích là thiếu tính nhân bản. Có những cảm xúc (như đau buồn khi mất người thân, phẫn nộ trước bất công) dường như là phản ứng tự nhiên và thậm chí cần thiết của con người.

\textbf{Thứ tư, hạn chế về tính thụ động chính trị:}

Mặc dù chủ nghĩa khắc kỉ khuyến khích tham gia đời sống công cộng và thực hiện nghĩa vụ xã hội, nhưng thuyết định mệnh và sự nhấn mạnh vào việc chấp nhận những gì nằm ngoài tầm kiểm soát đôi khi có thể dẫn đến thái độ thụ động trước bất công xã hội\footnote{Nguyễn Thị Minh Hương (2016), Cái nhìn duy ý chí của A.Schopenhauer về con người, \textit{Tạp chí Triết học}, 9 (304), Viện Triết học – Hà Nội, trang 55-62.}.

\subsection{Vai trò của chủ nghĩa khắc kỉ đối với triết học phương Tây hiện đại}

\subsubsection{Chủ nghĩa khắc kỉ với Kitô giáo sơ khai}

Mối quan hệ giữa chủ nghĩa khắc kỉ và Kitô giáo là một trong những chủ đề hấp dẫn nhất trong lịch sử tư tưởng. Khi Kitô giáo hình thành và phát triển trong thế giới Hy-La, nó đã hấp thụ nhiều thuật ngữ và tư tưởng từ chủ nghĩa khắc kỉ:

\begin{itemize}[leftmargin=1.5cm]
    \item \textit{Khái niệm Logos:} Tin Mừng theo Thánh John mở đầu: \textit{"Ban đầu có Ngôi Lời (Logos), Ngôi Lời ở cùng Thiên Chúa, và Ngôi Lời là Thiên Chúa."} Mặc dù Logos Kitô giáo có ý nghĩa khác (đây là Đức Kitô), nhưng việc sử dụng thuật ngữ này cho thấy ảnh hưởng của triết học Hy Lạp.
    
    \item \textit{Đạo đức khắc kỉ và đạo đức Kitô giáo:} Nhiều đức tính được chủ nghĩa khắc kỉ đề cao (tiết độ, can đảm, công bằng, trí tuệ) cũng là các đức tính cơ bản của Kitô giáo. Thánh Paul khi thuyết giáo ở Athens đã trích dẫn các nhà thơ khắc kỉ (Công vụ 17:28).
    
    \item \textit{Chủ nghĩa thế giới:} Quan niệm khắc kỉ về một cộng đồng nhân loại không phân biệt dân tộc, giai cấp phù hợp với thông điệp phổ quát của Kitô giáo: \textit{"Không còn Do Thái hay Hy Lạp, nô lệ hay tự do, nam hay nữ"} (Galatians 3:28).
\end{itemize}

Tuy nhiên, cũng có những khác biệt căn bản: chủ nghĩa khắc kỉ theo thuyết phiếm thần (Thượng đế = Vũ trụ) trong khi Kitô giáo theo thuyết hữu thần (Thượng đế siêu việt tạo dựng vũ trụ); chủ nghĩa khắc kỉ nhấn mạnh nỗ lực cá nhân đạt được đức hạnh trong khi Kitô giáo nhấn mạnh ân sủng thần linh.

\subsubsection{Chủ nghĩa khắc kỉ với phân tâm học của S. Freud}

Mặc dù Sigmund Freud (1856-1939) không trực tiếp kế thừa chủ nghĩa khắc kỉ, nhưng có những điểm giao thoa đáng chú ý:

\begin{itemize}[leftmargin=1.5cm]
    \item Cả hai đều quan tâm đến cơ chế nội tâm của con người và mối quan hệ giữa lý trí và các xung động phi lý trí.
    \item Chủ nghĩa khắc kỉ phân biệt giữa hegemonikon (phần chủ đạo của lý trí) và các pathe (đam mê); Freud phân biệt giữa Ego (Bản ngã), Id (Bản năng) và Super-ego (Siêu ngã).
    \item Cả hai đều cho rằng việc hiểu và quản lý các xung động nội tâm là chìa khóa của sức khỏe tâm thần.
\end{itemize}

Tuy nhiên, Freud bi quan hơn về khả năng của lý trí trong việc kiểm soát vô thức, trong khi chủ nghĩa khắc kỉ tin tưởng mạnh mẽ vào sức mạnh của logos (lý tính) cá nhân.

\subsubsection{Chủ nghĩa khắc kỉ xây dựng triết học tâm trạng khơi dòng cho sự ra đời và phát triển của chủ nghĩa hiện sinh}

Chủ nghĩa hiện sinh (Existentialism) thế kỷ XX (Kierkegaard, Heidegger, Sartre, Camus) có nhiều điểm giao thoa với chủ nghĩa khắc kỉ:

\begin{itemize}[leftmargin=1.5cm]
    \item \textit{Quan tâm đến ý nghĩa cuộc sống:} Cả hai đều đặt câu hỏi trung tâm: Làm thế nào để sống một cuộc đời có ý nghĩa giữa một thế giới có vẻ như phi lý hoặc thờ ơ với con người?
    
    \item \textit{Trách nhiệm cá nhân:} Sartre nói \textit{"Con người bị kết án phải tự do"} – ta buộc phải chọn lựa và chịu trách nhiệm. Điều này tương đồng với quan niệm khắc kỉ về synkatathesis (sự phê chuẩn).
    
    \item \textit{Đối mặt với cái chết:} Heidegger nói về \textit{"Sein-zum-Tode"} (Hữu-hướng-về-chết); Marcus Aurelius thường xuyên suy ngẫm về sự vô thường và cái chết như một phần của thực hành triết học.
    
    \item \textit{Amor Fati (Yêu số phận):} Khái niệm \textit{"yêu số phận"} của Nietzsche có nguồn gốc trực tiếp từ chủ nghĩa khắc kỉ – chấp nhận và thậm chí yêu thương mọi điều xảy ra như một phần của trật tự vũ trụ.
\end{itemize}

Tuy nhiên, chủ nghĩa khắc kỉ lạc quan hơn: họ tin rằng vũ trụ có lý tính và ý nghĩa (Logos), trong khi nhiều triết gia hiện sinh (đặc biệt Sartre và Camus) cho rằng vũ trụ là phi lý (absurd – vô nghĩa) và con người phải tự tạo ra ý nghĩa.

\subsubsection{Chủ nghĩa khắc kỉ với Liệu pháp Nhận thức Hành vi (CBT) và tâm lý học hiện đại}

Ảnh hưởng rõ ràng và quan trọng nhất của chủ nghĩa khắc kỉ trong thời đại hiện đại là trong lĩnh vực tâm lý trị liệu. Albert Ellis (sáng lập Rational Emotive Behavior Therapy - REBT vào năm 1955) và Aaron T. Beck (sáng lập Cognitive Therapy vào năm 1960s) đều thừa nhận sự kế thừa trực tiếp từ tư tưởng khắc kỉ, đặc biệt là Epictetus.

Nguyên lý cốt lõi của CBT phản ánh triết học khắc kỉ:
\begin{itemize}[leftmargin=1.5cm]
    \item \textbf{Epictetus}: \textit{"Con người không bị quấy nhiễu bởi sự việc, mà bởi quan điểm của họ về sự việc đó."}
    \item \textbf{CBT}: Sự kiện kích hoạt (A) không trực tiếp gây ra phản ứng cảm xúc (C). Chính niềm tin (B) của ta về sự kiện mới là nguyên nhân. Thay đổi niềm tin phi lý, ta sẽ thay đổi cảm xúc.
\end{itemize}

Kỹ thuật \textit{"cognitive restructuring"} (tái cấu trúc nhận thức) trong CBT chính là sự áp dụng hiện đại của kỷ luật phê chuẩn (synkatathesis) của chủ nghĩa khắc kỉ. CBT hiện là một trong những phương pháp tâm lý trị liệu được nghiên cứu kỹ lưỡng nhất và có hiệu quả được chứng minh trong điều trị trầm cảm, lo âu và nhiều vấn đề tâm lý khác\footnote{Robertson, D. (2010), \textit{The Philosophy of Cognitive-Behavioural Therapy (CBT): Stoic Philosophy as Rational and Cognitive Psychotherapy}, Karnac Books, London, tr. 55-80.}.

\subsubsection{Một số quan điểm cơ bản rút ra từ việc nghiên cứu chủ nghĩa khắc kỉ}

Qua việc phân tích ảnh hưởng của chủ nghĩa khắc kỉ, có thể rút ra một số quan điểm cơ bản:

\begin{enumerate}[leftmargin=1.5cm]
    \item \textbf{Tính liên tục của truyền thống triết học:} Các tư tưởng triết học không biến mất mà được tái khám phá và tái diễn giải qua các thời đại. Chủ nghĩa khắc kỉ \textit{"sống lại"} trong CBT là minh chứng cho sức sống vượt thời gian của triết học cổ đại.
    
    \item \textbf{Tính thực tiễn của triết học:} Chủ nghĩa khắc kỉ chứng minh rằng triết học không chỉ là tư biện trừu tượng mà có thể là công cụ cụ thể để cải thiện chất lượng cuộc sống.
    
    \item \textbf{Sự giao thoa Đông-Tây:} Nhiều nguyên lý khắc kỉ (buông bỏ, chấp nhận, tập trung vào hiện tại) có điểm tương đồng với Phật giáo và Đạo gia, cho thấy tính phổ quát của một số chân lý về đời sống con người.
    
    \item \textbf{Giới hạn của lý tính:} Các phê phán của Freud và Hiện sinh chủ nghĩa cho thấy niềm tin tuyệt đối vào lý tính của chủ nghĩa khắc kỉ cần được điều chỉnh – con người không hoàn toàn là sinh vật lý tính.
\end{enumerate}

\subsection{Vai trò của chủ nghĩa khắc kỉ đối với việc hình thành nhân cách và đời sống cá nhân con người hiện đại}

\subsubsection{Chủ nghĩa khắc kỉ và quá trình tu dưỡng bản thân của cá nhân}

Đối với bản thân người viết, việc nghiên cứu chủ nghĩa khắc kỉ đã mang lại những bài học quý giá có thể áp dụng trực tiếp vào đời sống hàng ngày:

\textbf{Phân biệt những gì có thể và không thể kiểm soát:}

Trong quá trình học tập và nghiên cứu, thường có nhiều yếu tố nằm ngoài tầm kiểm soát như kết quả đánh giá từ giảng viên, phản hồi từ tạp chí khoa học, hay những biến động trong môi trường làm việc. Nguyên lý \textit{dichotomy of control} (phân chia quyền kiểm soát) của Epictetus nhắc nhở rằng thay vì lo lắng về những điều này, ta nên tập trung vào những gì mình có thể kiểm soát: nỗ lực nghiên cứu, chất lượng bài viết, và thái độ trước kết quả.

\textbf{Rèn luyện sự kiên cường trước áp lực:}

Áp lực học tập, thời hạn hoàn thành công việc, và kỳ vọng từ bản thân và người khác là điều không thể tránh khỏi trong đời sống sinh viên và nghiên cứu sinh. Thực hành \textit{Premeditatio Malorum} (suy ngẫm trước về điều xấu có thể xảy ra) giúp chuẩn bị tinh thần cho những khó khăn, biến chúng từ những cú sốc bất ngờ thành những thử thách có thể dự đoán và đối phó.

\textbf{Kiểm soát phản ứng cảm xúc:}

Nguyên lý cốt lõi của chủ nghĩa khắc kỉ -- rằng \textit{``con người không bị quấy nhiễu bởi sự việc, mà bởi quan điểm của họ về sự việc đó''} -- có thể áp dụng trong nhiều tình huống: khi nhận được phản hồi tiêu cực, khi đối mặt với thất bại, hay khi gặp mâu thuẫn với người khác. Thay vì phản ứng tức thì, ta học cách dừng lại, xem xét niềm tin của mình, và chọn cách phản ứng phù hợp hơn.

\textbf{Sống giản dị và biết đủ:}

Trong bối cảnh xã hội hiện đại với nhiều cám dỗ vật chất, quan niệm về \textit{adiaphora} (những điều dửng dưng) nhắc nhở rằng sự giàu có, danh tiếng hay tiện nghi vật chất không phải là điều kiện cần thiết cho hạnh phúc. Lối sống giản dị, tập trung vào phát triển đức hạnh và trí tuệ mang lại sự bình an bền vững hơn việc chạy theo những thứ phù phiếm.

\textbf{Trách nhiệm với cộng đồng:}

Chủ nghĩa khắc kỉ không khuyến khích sự lánh đời hay ích kỷ. Ngược lại, tinh thần \textit{cosmopolitanism} (công dân thế giới) nhấn mạnh trách nhiệm của mỗi cá nhân với cộng đồng rộng lớn hơn. Đối với người nghiên cứu, điều này có nghĩa là đóng góp kiến thức cho xã hội, chia sẻ với đồng nghiệp, và sử dụng chuyên môn để phục vụ cộng đồng.

\subsubsection{Giá trị của chủ nghĩa khắc kỉ đối với đời sống cá nhân trong bối cảnh xã hội hiện đại}

Trong bối cảnh xã hội hiện đại với nhiều thách thức về sức khỏe tâm thần, áp lực công việc và sự bất ổn kinh tế - xã hội, các nguyên lý của chủ nghĩa khắc kỉ đang được tái khám phá như những công cụ hữu ích giúp con người đối phó với cuộc sống:

\textbf{Đối phó với căng thẳng và lo âu:}

Theo số liệu của Tổ chức Y tế Thế giới, tỷ lệ người mắc các rối loạn lo âu và trầm cảm đang gia tăng trên toàn cầu, đặc biệt trong thế hệ trẻ. Nguyên lý cốt lõi của chủ nghĩa khắc kỉ -- phân biệt những gì có thể và không thể kiểm soát -- cung cấp một phương pháp thực tiễn để giảm bớt lo âu. Thay vì lo lắng về những điều nằm ngoài tầm kiểm soát như ý kiến của người khác, biến động kinh tế hay thiên tai, ta học cách tập trung năng lượng vào những gì mình có thể thay đổi: thái độ, nỗ lực và phản ứng của bản thân.

\textbf{Xây dựng sự kiên cường tâm lý:}

Xã hội hiện đại thường có xu hướng bảo vệ thế hệ trẻ khỏi mọi khó khăn, dẫn đến tình trạng thiếu kỹ năng đối phó với nghịch cảnh. Các bài tập thực hành của chủ nghĩa khắc kỉ như \textit{Premeditatio Malorum} (suy ngẫm trước về điều xấu) giúp chuẩn bị tinh thần cho những thử thách không thể tránh khỏi trong cuộc sống -- mất việc, chia tay, bệnh tật hay mất mát người thân. Khi đã suy ngẫm trước, những sự kiện này sẽ không còn là cú sốc mà trở thành những thử thách có thể đối mặt.

\textbf{Thoát khỏi chủ nghĩa tiêu dùng:}

Xã hội tiêu dùng hiện đại không ngừng tạo ra những nhu cầu mới, khiến con người luôn cảm thấy thiếu thốn và bất mãn. Quan niệm về \textit{adiaphora} (những điều dửng dưng) của chủ nghĩa khắc kỉ nhắc nhở rằng sự giàu có và tiện nghi vật chất không phải là điều kiện cần thiết cho hạnh phúc. Hạnh phúc đích thực đến từ bên trong -- từ đức hạnh, mối quan hệ tốt đẹp và sự bình an nội tại -- chứ không phải từ việc sở hữu ngày càng nhiều đồ vật.

\textbf{Cải thiện các mối quan hệ:}

Nguyên lý khắc kỉ về việc không để cảm xúc chi phối hành động giúp cải thiện các mối quan hệ cá nhân và nghề nghiệp. Khi đối mặt với xung đột, thay vì phản ứng tức thì theo cảm xúc, ta học cách dừng lại, xem xét niềm tin của mình, và chọn cách phản ứng mang tính xây dựng hơn. Điều này đặc biệt quan trọng trong thời đại mạng xã hội, nơi những phản ứng cảm xúc bốc đồng có thể gây ra hậu quả nghiêm trọng.

\textbf{Tìm kiếm ý nghĩa cuộc sống:}

Trong thời đại mà nhiều người cảm thấy mất phương hướng và thiếu ý nghĩa, chủ nghĩa khắc kỉ cung cấp một khuôn khổ để sống có mục đích. Tinh thần \textit{cosmopolitanism} (công dân thế giới) nhắc nhở rằng mỗi cá nhân là một phần của cộng đồng nhân loại lớn hơn, và có trách nhiệm đóng góp cho lợi ích chung. Ý nghĩa cuộc sống không chỉ đến từ thành tựu cá nhân mà còn từ việc phục vụ cộng đồng và sống theo các nguyên tắc đạo đức\footnote{Holiday, R. \& Hanselman, S. (2016), \textit{The Daily Stoic: 366 Meditations on Wisdom, Perseverance, and the Art of Living}, Portfolio, New York.}.

\subsection*{Tiểu kết Chương 3}
\addcontentsline{toc}{subsection}{Tiểu kết Chương 3}

Chủ nghĩa khắc kỉ để lại một di sản phong phú với nhiều giá trị vượt thời gian: quan niệm về sự bình đẳng của mọi người, tự do nội tại, trách nhiệm cá nhân, và các bài tập thực hành rèn luyện sự kiên cường. Tuy nhiên, nó cũng có những hạn chế lịch sử: thuyết định mệnh khó dung hòa với tự do ý chí, quan niệm về cảm xúc đôi khi thiếu tính nhân bản, và nguy cơ bị lợi dụng để biện minh cho sự thụ động.

Ảnh hưởng của chủ nghĩa khắc kỉ lan tỏa khắp lịch sử tư tưởng phương Tây: từ Kitô giáo sơ khai, qua phân tâm học của Freud, đến chủ nghĩa hiện sinh và Liệu pháp Nhận thức Hành vi hiện đại\footnote{Nguyễn Hữu Vui (1998), \textit{Lịch sử triết học}, Nxb. Chính trị quốc gia, Hà Nội, tr. 150-180.}.

Đối với đời sống cá nhân trong bối cảnh xã hội hiện đại, các nguyên lý của chủ nghĩa khắc kỉ cung cấp những công cụ hữu ích để tu dưỡng bản thân, đối phó với căng thẳng và lo âu, xây dựng sự kiên cường tâm lý, và tìm kiếm ý nghĩa cuộc sống. Triết học khắc kỉ chứng minh rằng những tư tưởng cổ đại vẫn có giá trị thực tiễn trong việc hình thành nhân cách con người hiện đại.


% ============ KẾT LUẬN ============
\newpage
\section*{KẾT LUẬN}
\addcontentsline{toc}{section}{KẾT LUẬN}
% ============ KẾT LUẬN ============

Qua quá trình nghiên cứu đề tài \textit{"Chủ nghĩa khắc kỷ, những giá trị, hạn chế và vai trò đối với đời sống xã hội"}, chúng tôi rút ra những kết luận sau:

\textbf{Thứ nhất,} chủ nghĩa khắc kỉ là một trong những trường phái triết học quan trọng nhất của thế giới Hy-La, ra đời trong bối cảnh khủng hoảng của thời kỳ Hy Lạp hóa khi các cấu trúc xã hội truyền thống sụp đổ và con người buộc phải tìm kiếm ý nghĩa cuộc sống trong nội tâm. Trường phái này kế thừa sáng tạo các tư tưởng của Socrates, phái Khuyển nho và Heraclitus, đồng thời phát triển thành một hệ thống triết học toàn diện trải dài hơn 500 năm với ba giai đoạn: Sơ kỳ (Zeno, Chrysippus), Trung kỳ (Panaetius, Posidonius) và Hậu kỳ (Seneca, Epictetus, Marcus Aurelius)\footnote{Long, A. A. (1986), \textit{Hellenistic Philosophy: Stoics, Epicureans, Sceptics}, University of California Press, Berkeley.}.

\textbf{Thứ hai,} hệ thống tư tưởng của chủ nghĩa khắc kỉ thể hiện sự nhất quán chặt chẽ giữa ba bộ phận: Vật lý học xây dựng thế giới quan duy vật nhất nguyên với Logos (lý tính vũ trụ) chi phối mọi sự vật; Logic học phát triển lý thuyết về phantasia (ấn tượng) và synkatathesis (sự phê chuẩn) cùng với hệ thống logic mệnh đề tiên tiến; Đạo đức học đề cao đức hạnh như điều tốt duy nhất, phân biệt những gì ta có thể kiểm soát (thế giới nội tâm) với những gì nằm ngoài tầm kiểm soát (thế giới bên ngoài). Khái niệm oikeiosis (sự chiếm hữu) giải thích nguồn gốc tự nhiên của đạo đức và mở đường cho tư tưởng \textit{"công dân thế giới"} (cosmopolitanism – chủ nghĩa thế giới).

\textbf{Thứ ba,} chủ nghĩa khắc kỉ có những giá trị vượt thời gian đáng được ghi nhận: quan niệm về sự bình đẳng căn bản của mọi người, tự do nội tại không phụ thuộc hoàn cảnh bên ngoài, trách nhiệm cá nhân đối với phản ứng của chính mình, và các bài tập thực hành rèn luyện sự kiên cường. Tuy nhiên, nó cũng có những hạn chế lịch sử: thuyết định mệnh khó dung hòa hoàn toàn với tự do ý chí, quan niệm về cảm xúc đôi khi thiếu tính nhân bản, và tiêu chuẩn \textit{"hiền nhân hoàn hảo"} quá xa vời với đời sống thực.

\textbf{Thứ tư,} ảnh hưởng của chủ nghĩa khắc kỉ lan tỏa khắp lịch sử tư tưởng phương Tây. Kitô giáo sơ khai đã hấp thụ nhiều khái niệm và đức tính khắc kỉ. Triết học hiện đại (đặc biệt là Chủ nghĩa hiện sinh) chia sẻ một số mối quan tâm của chủ nghĩa khắc kỉ về ý nghĩa cuộc sống và trách nhiệm cá nhân. Quan trọng nhất, Liệu pháp Nhận thức Hành vi (CBT) – một trong những phương pháp tâm lý trị liệu hiệu quả nhất hiện nay – thừa nhận sự kế thừa trực tiếp từ tư tưởng của Epictetus\footnote{Robertson, D. (2010), \textit{The Philosophy of Cognitive-Behavioural Therapy (CBT)}, Karnac Books, London.}.

\textbf{Thứ năm,} trong bối cảnh thế kỷ XXI đầy biến động với khủng hoảng khí hậu, đại dịch, bất ổn xã hội và quá tải thông tin, các nguyên lý của chủ nghĩa khắc kỉ đang được tái khám phá như một phương thức sống hiệu quả. Sự phục hưng của \textit{"chủ nghĩa khắc kỉ hiện đại"} cho thấy các triết gia cổ đại vẫn có nhiều điều để dạy chúng ta về cách giữ vững phẩm giá và sự bình an giữa một thế giới bất định.

Đối với Việt Nam, mặc dù chủ nghĩa khắc kỉ không có ảnh hưởng trực tiếp, nhưng tinh thần tương đồng – sự kiên cường, bất khuất, tập trung vào những gì có thể kiểm soát – đã hiện diện trong truyền thống dân tộc. Các nguyên lý khắc kỉ có thể bổ sung cho truyền thống tư tưởng Việt Nam, góp phần xây dựng con người mới có bản lĩnh, có trách nhiệm, và có khả năng thích ứng với những thách thức của thời đại toàn cầu hóa.

Kết quả nghiên cứu cho thấy nhiệm vụ đặt ra của đề tài đã được hoàn thành. Chúng tôi hy vọng tiểu luận này góp phần làm phong phú thêm hiểu biết về lịch sử triết học phương Tây và cung cấp những gợi ý có giá trị cho việc xây dựng lối sống có ý nghĩa trong thời đại ngày nay.


% ============ TÀI LIỆU THAM KHẢO ============
\newpage
\section*{TÀI LIỆU THAM KHẢO}
\addcontentsline{toc}{section}{TÀI LIỆU THAM KHẢO}
% ============ TÀI LIỆU THAM KHẢO ============

\begin{enumerate}[leftmargin=1.5cm]
    \item Aurelius, M. (2006), \textit{Suy tưởng (Meditations)}, (Nguyễn Duy Nhiên dịch), Nxb. Thế Giới, Hà Nội.
    
    \item Epictetus (2019), \textit{Discourses, Fragments, Handbook (Giáo khoa thư và Cẩm nang)}, (Robin Hard dịch), Oxford University Press, Oxford.
    
    \item Hội đồng Trung ương chỉ đạo biên soạn giáo trình quốc gia các môn khoa học Mác - Lênin, tư tưởng Hồ Chí Minh (1999), \textit{Giáo trình triết học Mác – Lênin}, Nxb. Chính trị quốc gia, Hà Nội.
    
    \item Holiday, R., \& Hanselman, S. (2016), \textit{The Daily Stoic: 366 Meditations on Wisdom, Perseverance, and the Art of Living}, Portfolio, New York.
    
    \item Inwood, B. (Ed.) (2003), \textit{The Cambridge Companion to the Stoics}, Cambridge University Press, Cambridge.
    
    \item Irvine, W. B. (2008), \textit{A Guide to the Good Life: The Ancient Art of Stoic Joy}, Oxford University Press, New York.
    
    \item Long, A. A. (1986), \textit{Hellenistic Philosophy: Stoics, Epicureans, Sceptics} (2nd ed.), University of California Press, Berkeley.
    
    \item Long, A. A., \& Sedley, D. N. (1987), \textit{The Hellenistic Philosophers, Volume 1: Translations of the Principal Sources with Philosophical Commentary}, Cambridge University Press, Cambridge.
    
    \item Nguyễn Hữu Vui (1998), \textit{Lịch sử triết học}, Nxb. Chính trị quốc gia, Hà Nội.
    
    \item Nguyễn Thị Minh Hương (2016), Cái nhìn duy ý chí của A.Schopenhauer về con người, \textit{Tạp chí Triết học}, 9 (304), Viện Triết học – Hà Nội, trang 55-62.
    
    \item Pigliucci, M. (2017), \textit{How to Be a Stoic: Using Ancient Philosophy to Live a Modern Life}, Basic Books, New York.
    
    \item Robertson, D. (2010), \textit{The Philosophy of Cognitive-Behavioural Therapy (CBT): Stoic Philosophy as Rational and Cognitive Psychotherapy}, Karnac Books, London.
    
    \item Sellars, J. (2006), \textit{Stoicism}, University of California Press, Berkeley.
    
    \item Seneca (2018), \textit{Những bức thư đạo đức gửi Lucilius (Letters from a Stoic)}, (Robin Campbell dịch), Penguin Classics, London.
    
    \item Trần Văn Phòng (2007), \textit{Triết học Hy Lạp cổ đại}, Nxb. Chính trị quốc gia, Hà Nội.
\end{enumerate}


\end{document}
